%!TEX root = P231_notes.tex

\subsection{Physics versus Mathematics}
\lecdate{lec~02}


Let’s make one point clear:
\begin{align}
  \text{Physics} \neq \text{Mathematics} \ .
\end{align}
This is a truth in many different respects\footnote{The astronomer Fritz Zwicky would perhaps call this a \emph{spherical truth}; no matter how you look at it, the statement is still true.}:
\begin{itemize}
	\item Physicists are rooted in experimental results\footnote{Even theorists? \emph{especially} theorists.}. 
	
	\item Physicists Taylor expand to their hearts’ content---sometimes even when the expansion is not formally justified\footnote{\url{https://johncarlosbaez.wordpress.com/2016/09/21/struggles-with-the-continuum-part-6/}}.

	\item Physicists use explicit coordinates, mathematicians abhor this.  Even worse, we pick a basis and decorate every tensor with indices\footnote{Those who are not trained may be intimidated by physics because of all the indices we use. Ironically, physicists are often intimidated by mathematics because of the conspicuous absence of any indices.}.

	\item Physicists seek to uncover a truth about \emph{this} universe.
\end{itemize}

% We use mathematical formalism to describe the universe. Sometimes the formalism that we need even spurs on developments in formal mathematics. However, let us be clear that our descriptions of the universe are \emph{mathematical models}. Our models have limits of validity: where they are tested, where they break down. 

\subsection{The most important binary relation}


When we write equations, the symbol that separates the left-hand side from the right-hand side is a binary relation. We use binary relations like $=$ or $\neq$. Sometimes to make a point we’ll write $\cong$ or $\equiv$ or $\dot =$ to mean something like `definition’ or `tautologically equivalent to’ or some other variant of \emph{even more equal than equal}. 

As physicists the most important binary relation is none of those things\footnote{I thank Yuval Grossman for emphasizing this. Unrelated: \url{https://xkcd.com/2343/}}. Usually what we really care about is in $\sim$.\footnote{I use this the same way as $\propto$, which is completely different from `approximately,’ $\approx$.} This tells how how something \emph{scales}. If I double a quantity on the right-hand side, how does the quantity on the left-hand side scale? Does it depend linearly? Quadratically? Non-linearly? The answer encodes something important about the underlying physics of the system. It's the reason why \emph{imagine the cow is a sphere} is a popular punchline in a joke about physicists. 


By the way, implicit in this is the idea that in this class, we will not care about stray factors of 2. As my adviser used to say, if you’re worried about a factor of 2, then your additional homework is to figure out that factor of 2.\footnote{That being said, you're reading these notes and find an error, do let me know about it.} 

\subsection{Units}

There is another way in which physics is different from mathematics. It is far more prosaic. \emph{Quantities in physics have units}. We don’t just deal with numbers, we deal with kilograms, electron volts, meters. It turns out that dimensional analysis is a big part of what we do as physicists. 
