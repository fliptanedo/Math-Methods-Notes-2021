%!TEX root = P231_notes.tex

\documentclass[12pt]{article}
%!TEX root = P231_notes.tex

%%%%%%%%%%%%%%%%%%%%%%%%%%
%%%  COMMON PACKAGES  %%%%
%%%%%%%%%%%%%%%%%%%%%%%%%%

\usepackage{amsmath}
\usepackage{amssymb}
\usepackage{amsfonts}
\usepackage{graphicx}
\usepackage[utf8]{inputenc}			% inspire bibs
\usepackage{aas_macros}				% ads bibs
%\usepackage{amsthm}

%%%%%%%%%%%%%%%%%%%%%%%%%%%%%%%%%
%%%  UNUSUAL PACKAGES        %%%%
%%%  Uncomment as necessary. %%%%
%%%%%%%%%%%%%%%%%%%%%%%%%%%%%%%%%

%% MATH AND PHYSICS SYMBOLS
%% ------------------------
\usepackage{slashed}				% \slashed{k}
\usepackage{mathrsfs}				% Weinberg-esque letters
\usepackage{bbm}					% \mathbbm{1} conflict: XeLaTeX 
\usepackage{cancel}					
%\usepackage[normalem]{ulem} 		% for \sout
%\usepackage{youngtab}	    		% Young Tableaux


%% CONTENT FORMAT AND DESIGN
%% -------------------------
\usepackage[dvipsnames]{xcolor}
\usepackage[hang,flushmargin]{footmisc} % no footnote indent

\usepackage{fancyhdr}		% preprint number
\usepackage{lipsum}			% block of text 
\usepackage{framed}			% boxed remarks
\usepackage{subcaption}		% subfigures
\usepackage{cite}			% group cites
\usepackage{xspace}			% macro spacing

\usepackage{marginnote} 


%% TABLES IN LaTeX
%% ---------------
\usepackage{booktabs}		% professional tables
%\usepackage{nicefrac}		% fractions in tables,
%\usepackage{multirow}		% multirow elements in a table
%\usepackage{arydshln}		% dashed lines in arrays

%% Other Packages and Notes
%% ------------------------
\usepackage[font=small]{caption} % caption font is small
\usepackage{float}         % for strict placement e.g. [H]
 



%%%%%%%%%%%%%%%%%%%%%%%%%%%%%%
%%%  DOCUMENT PROPERTIES  %%%%
%%%%%%%%%%%%%%%%%%%%%%%%%%%%%%

% \usepackage[margin=2cm]{geometry}   % margins
\usepackage[
  top=2.5cm, 
  bottom=2.5cm, 
  outer=2.5cm,            % size of margin
  inner=2.5cm,              % size of opposite margin
  heightrounded, 
  marginparwidth=1.5cm,   % less than outer
  marginparsep=.2cm%    % between text and margin note
  ]{geometry}
%% https://en.wikibooks.org/wiki/LaTeX/Footnotes_and_Margin_Notes#Margin_Notes
\graphicspath{{figures/}}			% figure folder
\numberwithin{equation}{section}    % set equation numbering


%% References in two columns, smaller
%% http://tex.stackexchange.com/questions/20758/
\usepackage{multicol}
\usepackage{etoolbox}
\usepackage{relsize}
\patchcmd{\thebibliography}
  {\list}
  {\begin{multicols}{2}\smaller\list}
  {}
  {}
\appto{\endthebibliography}{\end{multicols}}


% Change list spacing (instead of package paralist)
% from: http://en.wikibooks.org/wiki/LaTeX/List_Structures#Line_spacing
\let\oldenumerate\enumerate
\renewcommand{\enumerate}{
  \oldenumerate
  \setlength{\itemsep}{1pt}
  \setlength{\parskip}{0pt}
  \setlength{\parsep}{0pt}
}

\let\olditemize\itemize
\renewcommand{\itemize}{
  \olditemize
  \setlength{\itemsep}{1pt}
  \setlength{\parskip}{0pt}
  \setlength{\parsep}{0pt}
}


%%%%%%%%%%%%%%%%%%%%%%%%%%%
%%%  (RE)NEW COMMANDS  %%%%
%%%%%%%%%%%%%%%%%%%%%%%%%%%

%% FOR `NOT SHOUTING' CAPS (e.g. acronyms)
%% ---------------------------------------
\newcommand{\acro}[1]{\textsc{\MakeLowercase{#1}}}    
\usepackage{scalefnt}
\newcommand\acronum[2][0.85]{{\scalefont{#1}#2}} 

%% COMMON PHYSICS MACROS
%% ---------------------
\renewcommand{\tilde}{\widetilde}   % tilde over characters
\renewcommand{\vec}[1]{\mathbf{#1}} % vectors are boldface
\newcommand{\dbar}{d\mkern-6mu\mathchar'26}    % for d/2pi
\newcommand{\ket}[1]{\left|#1\right\rangle}    % <#1|
\newcommand{\bra}[1]{\left\langle#1\right|}    % |#1>

%% THEOREM ENVIRONMENT
%% -------------------
\newtheorem{exercise}{Exercise}[section]
\newtheorem{example}{Example}[section]

%% COMMANDS FOR TEMPORARY COMMENTS
%% -------------------------------
\newcommand{\comment}[2]{\textcolor{red}{[\textbf{#1} #2]}}
\newcommand{\flip}[1]{{
	\color{green!50!black} \footnotesize [\textbf{\textsf{Flip}}: \textsf{#1}]
	}}

\newcommand{\lecdate}[1]{\marginnote{\footnotesize\color{blue!50!black}{\textsf{#1}}}}


%% COMMANDS FOR TOP-MATTER
%% -----------------------
\newcommand{\email}[1]{\href{mailto:#1}{#1}}
\newenvironment{institutions}[1][2em]{\begin{list}{}{\setlength\leftmargin{#1}\setlength\rightmargin{#1}}\item[]}{\end{list}}


%% COMMANDS FOR LATEXDIFF
%% ----------------------
%% see http://bit.ly/1M74uwc
\providecommand{\DIFadd}[1]{{\protect\color{blue}#1}} %DIF PREAMBLE
\providecommand{\DIFdel}[1]{{\protect\color{red}\protect\scriptsize{#1}}}

%% REMARK: use latexdiff option --allow-spaces
%% for \frac, ref: http://bit.ly/1iFlujR


%%%%%%%%%%%%%%%%%%%
%%%  HYPERREF  %%%%
%%%%%%%%%%%%%%%%%%%

%% This package has to be at the end; can lead to conflicts
\usepackage[
	colorlinks=true,
	citecolor=green!50!black,
	linkcolor=NavyBlue!75!black,
	urlcolor=green!50!black,
	hypertexnames=false]{hyperref}


%%%%%%%%%%%%%%%%%%%%%
%%%  TITLE DATA  %%%%
%%%%%%%%%%%%%%%%%%%%%

%% PREPRINT NUMBER USING fancyhdr
%% Don't forget to set \thispagestyle{firststyle}
%% ----------------------------------------------
\renewcommand{\headrulewidth}{0pt} 	% no separator
\setlength{\headheight}{15pt} 		% min to avoid fancyhdr warning
\fancypagestyle{firststyle}{
	\rhead{\footnotesize%
	\texttt{UCR-P231-FLIP-F2020}%
	}}

%% TOC overwrites fancyhdr, here's a fix
%% http://tex.stackexchange.com/questions/167828/
\usepackage{etoc}
\renewcommand{\etocaftertitlehook}{\pagestyle{plain}}
\renewcommand{\etocaftertochook}{\thispagestyle{firststyle}}

\begin{document}

%\thispagestyle{empty}		% default if no preprint #
\thispagestyle{firststyle} 	% to include preprint
%% I think this may screw up pdfsync on the first page

\begin{center}
	
	{\large \acronum{P231:}}
    {\Large \bf Mathematical Methods in Graduate Physics}

    \vskip .7cm

%% SINGLE AUTHOR FORMAT
%% --------------------
	\textbf{Flip Tanedo} \\
	\texttt{\footnotesize \email{flip.tanedo@ucr.edu}}

	\vspace{-1em}
    \begin{institutions}[1.5cm]
    \footnotesize
    {\it 
	    Department of Physics \& Astronomy, 
	    University of  California, Riverside, 
	    {CA} 92521	    
	    }    
    \end{institutions}


\end{center}




%%%%%%%%%%%%%%%%%%%%%
%%%  ABSTRACT    %%%%
%%%%%%%%%%%%%%%%%%%%%

\begin{abstract}
\noindent 
This is a crash course on mathematical methods necessary to succeed in the first-year physics graduate curriculum at \acro{UC} Riverside. 
%
The focus is how to solve differential equations using Green's functions.
\end{abstract}



\small
\setcounter{tocdepth}{2}
\tableofcontents
\normalsize
%\clearpage


%%%%%%%%%%%%%%%%%%%%%
%%%  THE CONTENT  %%%
%%%%%%%%%%%%%%%%%%%%%

%!TEX root = P231_notes.tex

\section{Introduction: Why mathematical methods?}
\lecdate{lec~01}

Physics 231:~Methods of Theoretical Physics is a course for first-year physics and astronomy graduate students. It is a `crash course’ in mathematical methods necessary for graduate courses in electrodynamics, quantum mechanics, and statistical mechanics. It is a \emph{boot camp} rather than a rigorous theorem--proof mathematics class. Where possible, the emphasis is on physical intuition rather than mathematical precision. 

\subsection{Green’s functions}

Our primary goal is to solve linear differential equations:
\begin{align}
  \mathcal O f(x) = s(x) \ .
\end{align}
In this equation, $\mathcal O$ is a \emph{differential operator} that encodes some kind of physical dynamics, say $\mathcal O = (d/dx)^2 + 3x\,(d/dx)$.  $s(x)$ is the \emph{source} of those dynamics. Finally, $f(x)$ is the system's physical \emph{response} that we would like to determine. The solution to this equation is:
\begin{align}
  f(x) &= \mathcal O^{-1} s(x) \ .
\end{align}
Simply writing that is deeply unsatisfying! %It is as if we were asked to solve $f'(x) = 3x$ and simply wrote $f(x) = \int 3x\, dx$. The real 
In this course, we think carefully about what $\mathcal O^{-1}$ actually \emph{means} and how we can calculate it. As you may have guessed, $\mathcal O^{-1}$ is the \textbf{Green's function} for the differential operator $\mathcal O$. 

We will approach this problem by analogy to linear algebra, where a linear transformation $A$ acting on a vector space can give equations like:
\begin{align}
  A \vec{v} = \vec{w} \ ,
\end{align}
whose solution is
\begin{align}
  \vec{v} = A^{-1} \vec{w} \ .
\end{align}
We will connect the notion of a linear differential operator to a matrix in infinite dimensional space to give a working definition of $\mathcal O^{-1}$. We will then pull out a bag of tricks from complex analysis to formally solve $\mathcal O^{-1}s(x)$ given $\mathcal O$ and $s(x)$. 


%!TEX root = P231_notes.tex

\subsection{Physics versus Mathematics}
\lecdate{lec~02}


Let’s make one point clear:
\begin{align}
  \text{Physics} \neq \text{Mathematics} \ .
\end{align}
This is a truth in many different respects\footnote{The astronomer Fritz Zwicky would perhaps call this a \emph{spherical truth}; no matter how you look at it, the statement is still true.}:
\begin{itemize}
	\item Physicists are rooted in experimental results\footnote{Even theorists? \emph{especially} theorists.}. 
	
	\item Physicists Taylor expand to their hearts’ content---sometimes even when the expansion is not formally justified\footnote{\url{https://johncarlosbaez.wordpress.com/2016/09/21/struggles-with-the-continuum-part-6/}}.

	\item Physicists use explicit coordinates, mathematicians abhor this.  Even worse, we pick a basis and decorate every tensor with indices\footnote{Those who are not trained may be intimidated by physics because of all the indices we use. Ironically, physicists are often intimidated by mathematics because of the conspicuous absence of any indices.}.

	\item Physicists seek to uncover a truth about \emph{this} universe.
\end{itemize}

% We use mathematical formalism to describe the universe. Sometimes the formalism that we need even spurs on developments in formal mathematics. However, let us be clear that our descriptions of the universe are \emph{mathematical models}. Our models have limits of validity: where they are tested, where they break down. 

\subsection{The most important binary relation}


When we write equations, the symbol that separates the left-hand side from the right-hand side is a binary relation. We use binary relations like $=$ or $\neq$. Sometimes to make a point we’ll write $\cong$ or $\equiv$ or $\dot =$ to mean something like `definition’ or `tautologically equivalent to’ or some other variant of \emph{even more equal than equal}. 

As physicists the most important binary relation is none of those things\footnote{I thank Yuval Grossman for emphasizing this. Unrelated: \url{https://xkcd.com/2343/}}. Usually what we really care about is in $\sim$.\footnote{I use this the same way as $\propto$, which is completely different from `approximately,’ $\approx$.} This tells how how something \emph{scales}. If I double a quantity on the right-hand side, how does the quantity on the left-hand side scale? Does it depend linearly? Quadratically? Non-linearly? The answer encodes something important about the underlying physics of the system. It's the reason why \emph{imagine the cow is a sphere} is a popular punchline in a joke about physicists. 


By the way, implicit in this is the idea that in this class, we will not care about stray factors of 2. As my adviser used to say, if you’re worried about a factor of 2, then your additional homework is to figure out that factor of 2.\footnote{That being said, you're reading these notes and find an error, do let me know about it.} 

\subsection{Units}

There is another way in which physics is different from mathematics. It is far more prosaic. \emph{Quantities in physics have units}. We don’t just deal with numbers, we deal with kilograms, electron volts, meters. It turns out that dimensional analysis is a big part of what we do as physicists. 

%!TEX root = P231_notes.tex

\section{Dimensional Analysis}
\lecdate{lec~03}

You may be be surprised how far you can go in physics by thinking deeply about dimensional analysis. Here we’ll only get you started. To go one step further, you may read more about the Buckingham Pi theorem\footnote{\url{https://aapt.scitation.org/doi/10.1119/1.1987069}} or dive into neat applications\footnote{\url{https://aapt.scitation.org/doi/full/10.1119/1.3535586}, \url{http://inspirehep.net/record/153032?ln=en}}.
%

\subsection{Converting Units}

Imagine that you have three apples. This is a number (three) an a unit (apple). The meaning of the unit depends on what you’re using it to measure. For example, if apples are \$1 each, then you could use an apple as a unit of currency. The way to do this is to simply \emph{multiply by one}:
\begin{align}
  (3\text{ apples}) \times \left(\frac{\text{\$ 1}}{\text{apple}}\right)
  &= \$ 3 \ .
\end{align}
We have used the fact that the exchange rate is simply the statement that
\begin{align}
  1\text{ apple} &= \$1
  & \Rightarrow &&
  1 &= \frac{\$ 1}{1\text{ apple}} \ .
\end{align}
You can do a similar thing for [kilo-]calories or any other conversion rate. 


All that matters is that the conversion is constant. Indeed, the constants of nature make very good `exchange rates.' For example, in high-energy physics we like to use \textbf{natural units}. This is the curious statement that
\begin{align}
  \hbar = c = 1 \ .
\end{align}
At face value, this doesn’t make sense. $\hbar$ has units of action, $c$ is a speed, and 1 is dimensionless. However, because nature gives us a \emph{fundamental} unit of action and a \emph{fundamental} unit of speed, we may use them as conversion factors (exchange rates),
\begin{align}
  c = 3 \times 10^{10}~\text{cm}/\text{s} \ .
\end{align}
If $c=1$, then this means
\begin{align}
  1 \text{ s} &=  3 \times 10^{10}~\text{cm} \ .
\end{align}
This, in turn, connects a unit of time to a unit of distance. By measuring time, the constant $c$ automatically gives us an associated distance. The physical relevance of the distance is tied to the nature of the fundamental constant: one second (or `light-second') is the distance that a photon travels in one second. Observe that this only works because $c$ is a constant. 

\subsection{Quantifying units}

We use the notation that a physical quantity $Q$ has \textbf{dimension} $[Q]$ that can be expressed in terms of units of length, mass, and time:
\begin{align}
  [Q] = L^a M^b T^c \ .
\end{align}
The {dimension} is the statement of the powers $a$, $b$, and $c$. You may want to also include units of, say, electric charge. Sticklers may pontificate about whether electric charge formally carries a new unit or not. 

\begin{example}
What are the units of force? We remember that $\vec{F} = m\vec{a}$, so 
\begin{align}
  [\vec F] &= [m][\vec{a}] = M\times L T^{-2} = L^1 M^1 T^{-2} \ .
  \label{eq:02:force:units}
\end{align}
\end{example}

Life is even easier in \textbf{natural units}, where $c=1$ means that units of length and time are `the same’ and $\hbar = 1$ means that units of time and energy (mass) are inversely related. In natural units, one typically write $[Q]$ to mean the mass-dimension of a quantity. To revert back to conventional units, one simply multiplies by appropriate factors of $1=c$ and $1=\hbar$. 

\begin{example}
What are the units of force in natural units? From \eqref{eq:02:force:units} we multiply by one to convert length and time into mass dimensions:
\begin{align}
  [\vec F] &= [c^{-3} \hbar \vec{F}] = M^2 \ .
\end{align}
In natural units we say $[\vec F] = 2$. Recall that energy and mass have the same dimension, which you may recall from the Einstein relation $E^2 = m^2c^4 + p^2c^2$.
\end{example}


\subsection{Usage: Sanity Check}

The simplest use of dimensional analysis is to check your work. The following expression is obviously wrong:
\begin{align}
  1 + (3~\text{cm}) \ .
\end{align}
This does not make sense. You cannot sum terms with different dimensions. Similarly, $\sin(3\text{ cm})$ does not make sense. What about $e^{5~\text{cm}}$? This doesn't make sense because
\begin{align}
  e^x = 1 + x + \frac{1}{2!} x^2 +  \cdots
\end{align}
Since each term comes with a different power of $x$, the argument of the exponential must be dimensionless. 

\begin{exercise}
Consider the energy spectrum of light emitted from some constant source---a distant star, the ongoing annihilation of dark matter in the galactic center, a laser in the Hemmerling lab. The spectrum encodes how many photons are emitted per unit time. We can plot this spectrum as a curve on a graph. We can even normalize the curve so that it integrates to one photon. This means we only care about the distribution of energy, not the absolute amount. The horizontal axis of such a plot is the photon energy. What are the units of the vertical axis?
\end{exercise}


\subsection{Usage: Solving problems}

Here’s a common problem in introductory physics. Assume you have a pendulum with some [sufficiently small] initial displacement $\theta_0$. What’s the period, $\tau$ of the pendulum? We draw a picture like this:

\begin{center}
\includegraphics[width=.4\textwidth]{figures/lec01_pendulum.pdf}
\end{center}
%
From dimensional analysis, we know that the period has dimensions of time, $[\tau] = T$. The problem gives us a length $[\ell]=L$ and the gravitational acceleration, $[g]=LT^{-2}$. Note that $[\theta_0] = 1$ is dimensionless. This means that the only way to form a quantity with dimensions of time is to use $g^{-1/2}$. This leaves us with a leftover $L^{-1/2}$, which we can fix by inserting a square root of $\ell$:
\begin{align}
  \tau \sim g^{-1/2} \ell^{1/2} \ .
\end{align}
If we wanted to be fancy, we can make this an equal sign by writing a function of the other dimensionless quantities in the problem:
\begin{align}
  \tau = f(\theta_0) \sqrt{\frac{\ell}{g}} \ .
\end{align}

\flip{To do: include problems from Robinett AJP article on dimensional analysis, doi: 10.1119/1.4902882.}


\subsection{Scaling}

A large part of physics has to do with scaling relations. Here’s a somewhat contrived example of how this works\footnote{This is adapted from section 11 of V.\ I.\ Arnold's \emph{Mathematical Methods of Classical Mechanics}, one of my favorite differential geometry textbooks because it's disguised as a book on mechanics.}. Suppose you have some static, central potential $U(\vec r)$. Maybe it’s some planet orbiting a star. 

\begin{center}
\includegraphics[width=.7\textwidth]{figures/lec01_orbit.pdf}
\end{center}

The force law gives:
\begin{align}
  m 
  \ddot{\vec{r}} = - \frac{\partial U}{\partial\vec{r}} \ .
  \label{eq:scaling:eg}
\end{align}
Suppose we are given a solution, $\vec r_0(t)$. Perhaps this is a trajectory that is experimentally verified. Dimensional analysis gives a way to scale this solution into other solutions. For example, let us scale time by defining a new variable $t'$:
\begin{align}
  t \equiv \alpha t' \ .
\end{align}
If the potential is static, then only the left-hand side of the force law changes. Even though the right-hand side formally has dimensions of time $\sim T^{-2}$, it does not transform because those units are carried in a constant, perhaps $G_N$, not a $(d/dt)^2$ like the left-hand side. The left-hand side of the force law gives:
\begin{align}
  m\left(\frac{d}{dt}\right)^2 \vec r_0(t) 
  &=
  m\alpha^{-2} \left(\frac{d}{dt'}\right)^2 \vec r_0(\alpha t') \ .
\end{align}
This begs us to define a new mass $m' = m\alpha^{-2}$. We thus have
\begin{align}
   m' \left(\frac{d}{dt'}\right)^2 {\vec{r}_0}(\alpha t')
  = - \frac{\partial U}{\partial\vec{r}_0} \ .
\end{align}
What this tells us is that $\vec r_1(t') \equiv \vec{r}_0(\alpha t')$ is a solution in the same potential that traces the same trajectory but at $\alpha$ times the speed and with mass $m'$. Changing labels $t'\to t$ for a direct comparison:
\begin{align}
   m' \left(\frac{d}{dt}\right)^2 {\vec{r}_1}(t)
  = - \frac{\partial U}{\partial\vec{r}_1} \ ,
\end{align}
which is indeed\footnote{We were able to swap $\vec r_0$ with $\vec r_1$ simply because $U$ only depends on the position.} \eqref{eq:scaling:eg} with a new mass $m'$ and a trajectory $\vec r_1(t') \equiv \vec{r}_0(\alpha t')$. For example, if $\alpha = 2$, then $\vec r_1(t)$ traces the same trajectory at double the velocity with one fourth of the mass.

\begin{exercise} 
I missed something in the example above. In order for a planet of mass $m'$ to have trajectory $\vec r_1(t')$, what is the mass of the star compared to the original mass $M_\star$?\footnote{Thanks to Eric Zhang (2021) for pointing this out.} 
\end{exercise}

\subsection{Error Estimates}

This section is based on a lovely \emph{American Journal of Physics} article by Craig Bohren.\footnote{\url{https://doi.org/10.1119/1.1574042}} Let’s go back to another high school physics problem. 

\begin{center}
\includegraphics[width=.4\textwidth]{figures/lec01_drop.pdf}
\end{center}

Suppose you drop a mass $m$ from height $h$ that is initially at rest. How long before this hits the ground? You can integrate the force equation to get
\begin{align}
  t_0 = \sqrt{\frac{2h}{g}} \ .
\end{align}
This is the \emph{exact} answer \emph{within our model} of the system. The model made several assumptions. The mass is a point mass, the gravitational acceleration is constant at all positions, there is no air resistance, etc. In fact, we \emph{know} that if we do an experiment, our result will almost certainly \emph{not} be $t_0$. All we know is that $t_0$ is probably a good approximation of what the actual answer is.

\emph{How good of an approximation is it?}

One way to do this is to do the next-to-leading order (\acro{NLO}) calculation, taking into account a more realistic (and hence more complicated) model and then compare to $t_0$. But this is stupid. Why do we need to do a \emph{hard} calculation to justify doing an \emph{easy} one? If we’re going to do the hard calculation anyway, what’s the point of ever doing the easy one?

What we really want is an error estimate. The error is
\begin{align}
  \epsilon &= \frac{t_1 - t_0}{t_0} \ .
\end{align}
This is a dimensionless quantity that determines how far off $t_0$ is from a more realistic calculation, $t_1$. Ideally we don’t actually have to do work to get $t_1$. 

Let’s assume that we’re not completely nuts and that we’re in a regime where the error is small\footnote{Note the error has to be dimensionless in order for us to be able to call it `small,` otherwise it begs the question of `small with respect to what?'}. Then the error is a function of some dimensionless parameters, $\xi$, in the system. We define these $\xi$ so that as $\xi \to 0$, $\epsilon(\xi) \to 0$. In other words, the approximation gets better as the $\xi$ are made smaller. By Taylor expansion:
\begin{align}
  \epsilon(\xi) = \epsilon(0) + \epsilon'(0) \xi + \mathcal O(\xi^2) \ .
\end{align}
By assumption  $\epsilon(0) = 0$ and $\mathcal O(\xi^2)$ is  small. We can then make a reasonable \emph{assumption} that the dimensionless value $\epsilon'(0)$  is $\mathcal O(1)$. This tells us that the error goes like $\epsilon(\xi) \sim \xi$.

By the way $\mathcal O(1)$ is read ``order one'' and is fancy notation for the order of magnitude. Numbers like 0.6, 2, and $\pi$ are all $\mathcal O(1)$. A number like $4\pi$, on the other hand, is $\mathcal O(10)$.  The assumption that a dimensionless number is $\mathcal O(1)$ is reasonable. When nature gives you a dimensionless parameter that is both (a) important and (b) very different from $\mathcal O(1)$, then there's a good chance that it's trying to tell you something about your model. Good examples of this are the cosmological constant, the strong \acro{CP} phase, and the electroweak hierarchy problem\footnote{There are also `bad' examples. The ratio of the angular size of the moon to the angular size of the sun is unity to very good approximation. This is quite certainly a coincidence. Our universe appears to be in an epoch where the density of matter, radiation, and dark energy all happen to be in the same ballpark. Our cosmological models imply that this is purely a coincidence. It would be very curious if this were not the case. As an exercise, you can explore (and critique) the appearance of the anthropic principle in physics.}. 

Here’s how it works in practice. One effect that we miss in our toy calculation of $t_0$ is that the earth is round with radius $R$. This means that assuming a constant $g$ is an approximation. We have two choices for a dimensionless parameter $\xi$:
\begin{align}
  \xi &= \frac{h}{R}
  &\text{or}&&
  \xi &= \frac{R}{h} \ .
\end{align}
There is an obvious choice: $\xi = h/R$, because we know that as $h$ is made smaller (drop the ball closer to the ground) or $R$ becomes bigger (larger radius of Earth) then the constant $g$ approximation gets better. We thus expect that the corrections from the position-dependence of $g$ go like $\mathcal O(h/R)$.
 
% Exercise: check by explicit calculation, 2017 lec 1


\subsection{Bonus: Allometry}

There’s a fun topic called \textbf{allometry}. This is basically dimensional analysis applied to biology. A typical example is to consider two people who have roughly the same shape but different characteristic lengths, $\ell$ and $L$:

\begin{center}
\includegraphics[width=.4\textwidth]{figures/lec01_allometry.pdf}
\end{center}

\begin{exercise}
If both people exercised at the same rate, which one loses more absolute weight? By how much? Let’s assume that weight loss is primarily from the conversion of organic molecules into carbon dioxide. 
\end{exercise}


\begin{exercise}
David Hu won his first IgNobel prize for determining that mammals take about 21 seconds to urinate, largely independently of their size\footnote{I learned about this in his excellent popular science book, \emph{How To Walk on Water and Climb Up Walls}.}. Can you use dimensional analysis to argue why this would be the case? It may be helpful to refer to the paper\footnote{\url{https://doi.org/10.1073/pnas.1402289111}}; as you read this, figure out which terms are negligible (and in what limits), identify the assumptions of the mathematical model (scaling of the bladder and urethra), and prove the approximate scaling relation. Make a note to yourself of which steps were non-trivial and where one may have naively mis-modeled the system. By the way, David Hu won a second IgNobel prize for understanding how wombats poop.
\end{exercise}

The above exercise on mammalian urination is a good example of \emph{modeling}. As physicists, we must identify and make a mathematical model for the most salient features of a problem. We must also be able to quantify the error from neglecting sub-leading contributions. As a rough model for scaling purposes, we can ignore viscosity and surface tension effects on human-sized mammals. For much smaller mammals, these effects become larger---the authors of the study note that mice tend to urinate droplets---in which case one can ignore the `inertial' $\frac{1}{2} \rho v^2$ term in Bernoulli's equation. For human-sized mammals, we may assume that steady state urination is given by Bernoulli's equation:
\begin{align}
  P + \rho g h = \frac{1}{2}\rho v^2 \ ,
\end{align}
where $P$ is the pressure from the bladder, $h$ is the column height of the urethra, $\rho$ is the mass density of urine, and $v$ is the velocity of the urine at the end of the urethra. Let us simplify to the condition where urination is purely driven by gravity---that is, the bladder does not exert any additional pressure, $P=0$. You can now show that the total urination time scales like the mass of the mammal to the one-sixth power, $\tau \sim M^{1/6}$. That is, the urination time has a very weak scaling dependence on how massive the mammal is.

\begin{exercise}
In August 2021, Ezra Klein interviewed Dr.~C\'eline Goudner about the \acro{COVID-19} variant.\footnote{\url{https://www.nytimes.com/2021/08/06/opinion/ezra-klein-podcast-celine-gounder.html}} In the interview, Klein cited the statement that the Delta variant has $\mathcal O(1000)$ times the viral load than prior \acro{COVID} strains. Goudner then interprets this in the following way: if the \acro{CDC} defined `close contact' for prior strains as 15 minutes of being indoors with an infected invdividual without a mask, then the equivalent `close contact' time for the Delta variant is around \emph{one second}. What scaling assumptions go into that estimate? Some of these assumptions are not obvious to me: for example, parts of the respiratory have a fractal-like structure that would lead me to suspect fractal scaling dimensions for surface area. \acro{Remark}: Just because you know dimensional analysis, that does not make you a medical, healthcare, or public policy expert.\footnote{During the early days of the \acro{COVID-19} pandemic, many physicists suddenly became armchair mathematical modelers of epidemics. Some of this was driven by a hubris that our mathematical modeling intuition is much better than anyone in medicine. It seems many the physicists lost interest when their models aligned poorly with what actually happened.} 
\end{exercise}
 

 \flip{To do: add examples and ``there ubiquitous properties of living species'' from Meyer-Vernet and Rospars in AJP doi: 10.1119/1.4917310.}
 % TO DO: add reference to
 % https://aapt.scitation.org/doi/full/10.1119/1.4917310
 % some empirical facts

 % maybe also: https://aapt.scitation.org/doi/10.1119/1.4902882
%!TEX root = P231_notes.tex

\section{Linear Algebra Review}

As physicists, linear algebra is part of our \acro{DNA}, from the vector calculus in our first electrodynamics course to quantum mechanics. So why should we patronize ourselves with yet another review of linear algebra?
%
We want to understand Green’s functions the inverse of a matrix. The `matrix' in question is the differential operator $\mathcal O$ in \eqref{eq:greens:function:equation}.
%
This is important:
\begin{align}
	\text{differential operator}
	&=
	\infty\text{-dimensional matrix} \ .
\end{align}
If differential operators are matrices, what vector space do these matrices act on? These matrices act on a space of functions, which turns out to be a vector space:
\begin{align}
  \text{function space} &= \infty\text{-dimensional vector space} \ .
\end{align}
Don't be intimidated by terminology like \emph{function space}; this is just an abstract place where functions live. Just recall back to your intuition from \acronum{3D} Euclidean vector space, $\mathbb{R}^3$: any 3-vector $\vec{v}$ lives in the vector space $\mathbb{R}^3$. If we transform $\vec{v}$ by a linear transformation ${A}$, you get a new vector  $\vec{w} = {A}\vec{v} \in \mathbb{R}^3$ that is also in the vector space.

%
Weird things can happen when we extend our intuition from finite things to infinite things\footnote{For example, the Hilbert Hotel puzzle.}, but for this course we'll try to draw as much intuition as we can from finite dimensional linear algebra to apply it to infinite dimensional function spaces.

\subsection{The basics}

A \textbf{linear transformation} $A$ acts on a vector $\vec{v}$ as $A\vec{v}$. 
This transformation satisfies
\begin{align}
  A(\alpha \vec{v}+ \beta \vec{w}) = \alpha A\vec{v} + \beta A\vec{w} \ .
\end{align}
Here $\alpha$ and $\beta$ are numbers.
%
This is conventionally matrix multiplication. The result is also a vector. One way that we like to think about vectors is as columns of elements:
\begin{align}
  \begin{pmatrix}
    v^{1} \\ v^{2} \\ \vdots \\ v^{N}
  \end{pmatrix} \ ,
  \label{eq:vector:def:as:column}
\end{align}
where $N$ is the \textbf{dimension} of the vector space. Our notation is that $v^i$ refers to the $i^\text{th}$ component of $\vec{v}$. Sometimes---as physicists---we refer to $v^i$ as the vector itself, which is a slight abuse of notation that occasionally causes confusion.

In this course we always assume a nice orthonormal basis. In this case, $(\vec{v} + \vec{w})^i = v^i + w^i$.
\begin{exercise}
Convince yourself that adding vectors becomes more complicated in polar coordinates. Namely, $(\vec{v} + \vec{w})^i \neq v^i + w^i$.
\end{exercise}

Because the linear transformation of a vector is another vector, we know that the sequential application of linear transformations is itself a linear transformation. This is a bombastic way of saying that you can multiply matrices to produce a matrix.  Here’s how it works in two dimensions. A transformation that takes vectors into vectors takes the following form:
\begin{align}
  A &= 
  \begin{pmatrix}
   A^{1}_{\phantom{1}1} & A^{1}_{\phantom{1}2}
   \\
   A^{2}_{\phantom{1}1} & A^{2}_{\phantom{1}2}
  \end{pmatrix} \ .
\end{align}
We have introduced upper and lower indices; for now treat this as a definition. This sometimes causes confusion. So here are some guidelines:
\begin{itemize}
	\item Treat the upper and lower indices as a definition. The components of the linear transformation $A$ are \emph{defined} by $A^i_{\phantom{i}j}$ where $i$ is the row number and $j$ is the column number. 
	\item We have not yet explained the significance of the heights, but for now we mandate that the first index is always upper and the lower index is always lower. The following objects do not (yet) make sense: $A^{1}_{\phantom{1}2}$ and \emph{not} $A_{12}$, $A^{12}$, or $A_{1}^{\phantom{1}2}$.
	\item We will soon define \emph{additional machinery} to raise and lower indices shortly. This is takes us from a vector space to a metric space.
	\item The heights of the indices are a convenient shorthand notation that we will elucidate shortly; it is related to the choice of upper indices in \eqref{eq:vector:def:as:column}.
	\item All of this may be familiar from special relativity. Extra credit if you realize that this should also be familiar from quantum mechanics.
\end{itemize}
If you’re squeamish about the indices, don’t worry: the elements of $A$ have two indices, the first one is written a little higher than the second one. This notation is neither mathematics nor physics, it’s a convention that we use for future convenience.

The action of a linear transformation $A$ on a vector $\vec{v}$ is:
\begin{align}
  A\vec{v}
  =
  \begin{pmatrix}
    A^{1}_{\phantom{1}1} & A^{1}_{\phantom{1}2}
   \\
   A^{2}_{\phantom{2}1} & A^{2}_{\phantom{2}2}   
  \end{pmatrix} 
  \begin{pmatrix}
    v^1\\
    v^2
  \end{pmatrix}
  =
  \begin{pmatrix}
    A^1_{\phantom{1}1} v^1 + A^1_{\phantom{1}2}v^2\\
    A^2_{\phantom{2}1} v^2 + A^2_{\phantom{2}2}v^2
  \end{pmatrix} \ .
\end{align}
Look at this carefully. The components of the new vector $(A \vec{v})^i$ are sums. In each term, the second/lower index of an $A$ element multiplies the component of $\vec{v}$ with the same index. The first/upper index of $A$ tells you whether that term should is in $(A \vec{v})^1$ or $(A \vec{v})^2$. 

A generic component of $(A\vec{v})$ is
\begin{align}
  (A\vec{v})^i = \sum_j A^i_{\phantom{i}j} v^j
  = A^i_{\phantom{i}j} v^j \quad \text{(Einstein convention)}
   \ .
\end{align}
On the right-hand side we use Einstein notation: \emph{we implicitly sum over repeated upper/lower indices}. We will use this notation from now on.
%
If you are at all in doubt about this, please work out the $2\times 2$ case carefully and compare to the succinct notation above. 

\begin{exercise}
Consider three-dimensional Euclidean space, $\mathbb{R}^3$. A linear transformation $A$ on this space is a $3\times 3$ matrix with elements of the form $A^i_{\phantom{i}j}$. Explicitly write out the second component of the vector $A\vec{v}$. This is a sum of three terms.
\end{exercise}

If $A$ and $B$ are linear transformations, then $A+B$ is a linear transformation. The components of $A+B$ are simply the piecewise sum of the corresponding components of $A$ and $B$:
\begin{align}
  (A+B)^i_{\phantom i j} = A^i_{\phantom i j} + B ^i_{\phantom i j} \ .
\end{align}


\subsection{Linear Transformations and Vector Spaces}

Let’s be a little more pedantic. We need to move past the idea that a vector $\vec{v}$ is some \emph{column of numbers}. A vector space is abstract and we need to to start thinking of vector spaces more generally. The layer of abstraction is encoded in the basis vectors, which we write as $\vec{e}_{(i)}$. For a space of dimension $N$, there are $N$ such vectors indexed by the subscript. Let us more formally write the vector $\vec{v}$ as
\begin{align}
  \vec{v} = 
  v^1 \vec{e}_{(1)}
  +
  v^2 \vec{e}_{(2)} + \cdots
  = v^i \vec{e}_{(i)} \ .
  \label{eq:v:v1:v2:v3}
\end{align}
These basis vectors may be unit vectors in space. In the `column of numbers' representation, they can be unit column vectors, e.g.
\begin{align}
  \vec{e}_{(1)}
  &= 
  \begin{pmatrix}
  1 \\ 0 \\ 0
  \end{pmatrix}
  &
    \vec{e}_{(2)}
  &= 
  \begin{pmatrix}
  0 \\ 1 \\ 0
  \end{pmatrix}
  &
  \cdots \ .
\end{align}
With this basis, \eqref{eq:v:v1:v2:v3} gives \eqref{eq:vector:def:as:column}
But these may be more general objects. For example, you can specify a color of light by specifying the red/green/blue content. We could have $\vec{e}_{(1)}$ be a unit amount of red light, $\vec{e}_{(2)}$ be a unit amount of green light, and $\vec{e}_{(3)}$ be a unit amount of blue light. Then a 3-vector $\vec{v}$ would correspond to light of a particular color. This color space is a vector space.




\subsection{A funny vector space: histogram space}

Here’s a funny vector space that we’re going to use as a pedagogical crutch. Imagine histogram-space. The basis vectors are:

\begin{center}
\includegraphics[width=.45\textwidth]{figures/lec02_e1.pdf}
\includegraphics[width=.45\textwidth]{figures/lec02_e2.pdf}\\
\includegraphics[width=.45\textwidth]{figures/lec02_e3.pdf}
\includegraphics[width=.45\textwidth]{figures/lec02_e4.pdf}
\end{center}

\noindent This is a basis for a histogram over unit bins from $x=0$ to $x=4$. A vector in this space is, for example:

\begin{center}
\includegraphics[width=.8\textwidth]{figures/lec02_hist.pdf}
\end{center}

\noindent We can perform a linear transformation $A$ on $\vec{v}$ which outputs another vector. Let’s say it’s this:


\begin{center}
\includegraphics[width=.8\textwidth]{figures/lec02_hist2.pdf}
\end{center}

\begin{exercise}
From the image above, can you derive what $A$ is? 
\end{exercise}

\noindent The answer to the above exercise is \emph{no}. Please make sure you convince yourself why: there are many different transformations that convert to old histogram into the new histogram. If you're not convinced: the matrix $A$ is $4\times 4$ and thus has 16 entries that we need to define. The matrix equation $A\vec{v} = \vec{w}$ for known vectors $\vec{v}$ and $\vec{w}$ encodes only four equations.

The power of this admittedly strange formalism is that we can think of these histograms as approximations of continuous functions:

\begin{center}
\includegraphics[width=.4\textwidth]{figures/lec02_histfun.pdf}
\end{center}

Thus a vector in this approximate (discretized) \emph{function} space is 
\begin{align}
  \vec{f} = 
  \begin{pmatrix}
    f^1 \\
    f^2 \\
    \vdots\\
    f^N
  \end{pmatrix} \ .
\end{align}

\subsection{Derivative Operators}

Our discretized function space allows us to define a [forward] derivative:
\begin{align}
  \vec{f'} =
  \frac{1}{\Delta x}
  \begin{pmatrix}
    f^2 - f^1 \\
    f^3 - f^2 \\
    \vdots
    \\
    f^{i+1}-f^i
    \\
    \vdots
  \end{pmatrix} \ .
\end{align}
This is familiar if you’ve ever had to manually program a derivative into a computer program. Note that the right-hand side looks like a linear transformation of $\vec{f}$. In other words, we expect to be able to write a matrix $D$ so that
\begin{align}
  \vec{f'} = D\vec{f} \ .
\end{align}
One problem is apparent: what happens at the `bottom’ of the vector? What is the last component of the derivative, $\vec{f'}^N$? Formally, this is
\begin{align}
  {(f')}^N = \frac{1}{\Delta x}(f^{N+1} - f^N) \,
\end{align}
but now we have no idea what $f^{N+1}$ is. That was never a component in our vector space. There is no $\vec{e}_{(N+1)}$ basis vector. 

This demonstrates and important lesson that we’ll need when we move more formally to function spaces: \emph{boundary conditions are part of the definition of the function space}. 

In the present case, let’s assume Dirichlet boundary conditions. A convenient way to impose this is to define what happens to all functions outside the domain of the function space:
\begin{align}
  f^{i > N} = f^{i < 1} = 0 \ .
\end{align}
This solves the problem of the derivative on the last component:
\begin{align}
  {(f')}^N = \frac{1}{\Delta x}(f^{N+1} - f^N) 
  = 
  \frac{- f^N}{\Delta x}  \ .
\end{align}

Alternatively, we could have also imposed periodic boundary conditions:
\begin{align}
  f^{i} &= f^{i+ kN}
  & k\in \mathbb{Z} \ .
\end{align}
This would then give
\begin{align}
  {(f')}^N = \frac{1}{\Delta x}(f^{N+1} - f^N) 
  = 
  \frac{1}{\Delta x}(f^{1} - f^N) 
  \ .
\end{align}
One could have also defined a backward derivative where $(f')^i \sim f^{i}-f^{i-1}$ \ . The second derivative may be defined symmetrically:
\begin{align}
  (f'')^i = \frac{(f^{i+1} - f^i) - (f^i - f^{i-1})}{\Delta x^2} \ .
\end{align}
You may pontificate about the reason why the first derivative does have a symmetric discretization while the second derivative does. 


\subsection{Derivatives in other function space bases}

There are other ways to write a discrete basis of functions. Here’s a natural one for functions that are up to second-order polynomials:
\begin{align}
  \vec{e}_{(0)} &= 1
  &
  \vec{e}_{(1)} &= x
  &
  \vec{e}_{(2)} &= x^2 \ .
\end{align}
Let’s sidestep questions about orthonormality for the moment. Clearly linear combinations of these basis functions can produce any quadratic function:
\begin{align}
  f(x) &= a x^2 + bx + c
  & \Rightarrow&&
  \vec{f} &=
  \begin{pmatrix}
     c \\ b \\ a
   \end{pmatrix} \ . 
\end{align}
The derivative operator has an easy representation in this space:
\begin{align}
  D = 
  \begin{pmatrix}
    0 & 1 & 0   \\
    0 & 0 & 2   \\
    0 & 0 & 0   
  \end{pmatrix} \ .
\end{align}
We can see that
\begin{align}
  D \vec{f}  &= 
  \begin{pmatrix}
     b \\
     2 a \\
     0
  \end{pmatrix} 
  &
  D^2 \vec{f}  &= 
  \begin{pmatrix}
     2a \\
     0 \\
     0
  \end{pmatrix} 
  &
  D^3 \vec{f}  &= 
  0 \ .
\end{align}
The last line is, of course, the realization that the third-derivative of a quadratic function vanishes. Feel free to attach mathy words to this like \emph{kernel}.

There are other bases that we may use for function space. A particularly nice one that we will use over and over is the Fourier basis, which we prosaically refer to as \emph{momentum space}. The basis vectors are things like sines, cosines, or oscillating exponentials. These do not vanish for any power of $D$.


\subsection{Locality}

Notice that in the histogram basis, the derivative matrix $D$ is sparse. It is zero everywhere away from the diagonal. The only non-zero elements on row $i$ are around the $i\pm 1$ column.  Higher powers of $D$ will sample further away, but the non-zero elements are always clustered near the diagonal.

This is simply a notion of locality. Remember the Taylor expansion:
\begin{align}
  f(x) = f(0) + f'(0) x + \frac{1}{2} f''(0)x^2 + \cdots \ .
\end{align}
If we think about the histogram as a discretization of a continuous function, then it is clear what the higher derivatives are doing. Given a function $f(x) = \vec{f}$, one might like to know about the function around some point $x_0$ corresponding to some index $i$. That is: $f^i = f(x_0)$. If you’d like to learn more about the function around that point, one can express the derivative at $x_0$. Thus $D\vec{f}$ says something about the slow, $D^2\vec{f}$ says something about the curvature, and so on. Because each successive power of $D$ samples terms further away from $f^i$, you can tell that these higher order terms are learning about the function further and further away from $x_0$. 

Now think about the types of differential equations that you’ve encountered in physics. They often include one or two derivatives. You hardly ever see three, four, or more derivatives. There’s a reason for this: nature appears to be local, so our mathematical models of nature have locality built in. Physics at one spacetime point should not depend on spacetime points that are far away. 

The reason for this is perhaps most elegant in special relativity. One of the key points of special relativity is that causality may be observer-dependent if two events do not occur at the same spacetime point. If we want to build causal theories of nature, then the dynamics at $x_0$ should not rely on what is happening at $x_1$, a finite distance away.\footnote{This is different from saying that information could propagate from $x_0$ to $x_1$; such propagation could come from some causal excitation of the electromagnetic field traveling every infinitesimal distance between the two positions.}


 

\subsection{Row Vectors and all that}

In high school we didn’t distinguish between row vectors and column vectors. They both seemed to convey the same information. The row vectors were just `tipped over.’ Perhaps you noticed that they follow the rules of `matrix multiplication’ act on a vector:
\begin{align}
  \begin{pmatrix}
    w_1 & w_2 & \cdots
  \end{pmatrix}
  \begin{pmatrix}
    v^1 \\
    v^2 \\
    \vdots
  \end{pmatrix}
  &= 
  w_1 v^1 + w_2 v^2 + \cdots \ .
\end{align}
In fact, this is like $\vec{w}^T$ is a function that acts linearly on $\vec{v}$: 
\begin{align}
  \vec{w}^T(\vec v) &= w_1 v^1 + w_2 v^2 + \cdots \ .
\end{align}

Indeed, let is be a bit more formal about this. This layer of formalism is uncharacteristic of our approach in this course, but this underpins so much of the mathematical structure of our physical theories that it is worth getting right from the beginning. 

Let $V$ be a vector space. It contains vectors, $\vec{v}$. Sometimes these are called contravariant vectors or kets. They have basis vectors $\vec{e}_{(i)}$. 

Now introduce a related but \emph{completely distinct} vector space called $V^*$. This is the space of \textbf{dual vectors} to $V$. A \textbf{dual vector} is what you may know as a \textbf{row vector}, a \textbf{ket}, a \textbf{covariant vector}, or a [differential] \textbf{one-form}. These are all words for the \emph{same idea}. An element of $V^*$, say $(\vec{w}^T)$ is a linear function that takes vectors and spits out numbers:
\begin{align}
  \vec{w}^T \in V^* \Rightarrow \vec{w}^T: V \to \mathbb{R} \ .
 \end{align}
Don’t think about $\vec{w}^T$ as some kind of operation on a vector $\vec{w}\in V$; at least not yet. For now the `$^T$' is just part of the name of $\vec{w}^T$. The two spaces $V$ and $V^*$ are totally different. We haven’t said anything about how to turn elements of $V$ into elements of $V^*$ or vice versa.
%
It should be clear that there is a sense of `duality’ here: the vectors $V$ are also linear functions that take a dual vector and spit out a number. 

Let us call the basis of dual vectors $\tilde{\vec{e}}^{(i)}$. This notation is cumbersome, so we’ll change to something different soon. The upper index is deliberate. The defining property of $\tilde{\vec{e}}^{(i)}$ is:
\begin{align}
  \tilde{\vec{e}}^{(i)}\left(\vec{e}_{(j)}\right) \equiv \delta^i_j \ .
\end{align}
One may check that this gives
\begin{align}
  \left(w_i\tilde{\vec{e}}^{(i)}\right)\left(v^j\vec{e}_{(j)}\right)
  = w_i v^j \delta^i_j = w_i v^i = w_1 v^1 + w_2 v^2 + \cdots \ .
\end{align}



\subsection{Orthonormal Bases}

At this point we should take a deep breath and state explicitly that we’ve been assuming an orthonormal basis. In this course we will continue to use an orthonormal basis. You may object to this and say that you used to believe in orthonormal bases until you had to write down the gradient (or worse, the Laplacian) in spherical coordinates. 

There are many things to be said about this, none of them are particularly edifying without a full discussion. With no apologies, I’ll make the following [perhaps perplexing] remarks:
\begin{enumerate}
\item There is no such thing as a `position vector.' Positions refer to some base space, whereas vectors (like differential operators) act on the tangent space at a point of that base space. 
\item A given tangent space is `nice’ and has a nice orthonormal basis. 
\item That basis may not be the same for neighboring tangent spaces (perhaps due to coordinates, perhaps due to intrinsic curvature). 
\end{enumerate}
In this course these nuances will not come up. In the rest of your life you’ll still have to deal with curvilinear coordinates. But suffice it to say that our study of function space will be nice an orthonormal. Of course, we haven’t yet given a definition of `normality.’

\subsection{Bra-Ket Notation}

There is neither any physics nor mathematics contained in a choice of notation. However, a convenient notation does simplify our lives. Let us introduce bra-ket notation. In this notation, we denote vectors by kets:
\begin{align}
  |v\rangle = v^i|i\rangle \ ,
\end{align}
where $|i\rangle = \vec{e}_{(i)}$ is the basis of vectors that span the vector space $V$. There is nothing new or different about this object,  $\vec{v} = |v \rangle$.

We denote dual vectors (row-vectors, one-forms) as bras:
\begin{align}
  \langle w | &= w_i \langle i| \ ,
\end{align}
where $\langle i | = \tilde{\vec{e}}^{(i)}$. The orthonormality of this basis is encoded in 
\begin{align}
  \langle i | j \rangle = \delta^i_j \ .
\end{align}

In bra-ket notation a linear transformation $A$ has a basis
\begin{align}
  A = A^i_{\phantom{i}j} |i\rangle \langle j| \ .
\end{align}
The notation $|i\rangle \langle j|$ is shorthand for $|i\rangle \otimes \langle j|$. If the $\otimes$ doesn’t mean anything to you, that’s fine. It doesn’t mean much to me either. Matrix multiplication proceeds as before:
\begin{align}
  A\vec{v} = A|v\rangle = 
  A^i_{\phantom{i}j} |i\rangle \langle j| v^k |k \rangle
  = 
  A^i_{\phantom{i}j}  v^k  |i\rangle \langle j|k \rangle
  = 
  A^i_{\phantom{i}j}  v^k  |i\rangle \delta^j_k
  = 
  A^i_{\phantom{i}j}  v^j  |i\rangle \ .
\end{align}
Observe that the power of the notation is clear: the object with the index $v^i$ is just a number. It commutes with everything. All of the vector-ness is carried in the basis objects: the bras, kets, and ket-bras. Those do not commute. But they have a well defined way in which kets act on bras (or vice versa).\footnote{This is where the $\oplus$ notation is handy. It keeps track of which kets/bras might hit which other bras/kets. This falls under the name of multi-linear algebra.}


\subsection{Eigenvectors are nice}

Give a sufficiently \emph{nice} linear transformation, $A$, there is a particularly convenient basis: the eigenvectors of $A$. These are kets $|\lambda\rangle$ such that
\begin{align}
  A |\lambda\rangle = \lambda |\lambda\rangle \ .
\end{align}
In other words, $A$ acts on the eigenvector by rescaling. The rescaling coefficient is the eigenvalues. For \emph{nice} transformations, there is a complete set of such vectors to span the vector space.

If you write a general vector $|v\rangle$ in terms of this eigenbasis,
\begin{align}
  |v\rangle = v^i |\lambda_{(i)} \rangle \ ,
\end{align}
Then the action of $A$ on this vector is easy:
\begin{align}
  A |v\rangle = \sum_i \lambda_{(i)} v^i |\lambda_{(i)} \rangle \ .
\end{align}
In fact, assuming that all of the eigenvalues are non-zero, even the matrix inverse is easy:
\begin{align}
  A^{-1}|v\rangle = \sum_i \lambda_{(i)}^{-1} v^i |\lambda_{(i)} \rangle \ .
\end{align}


\subsection{The Green’s Function Problem}

Going back to the big picture: recall that we want to solve differential equations of the form $\mathcal O f(x) = s(x)$. If we had a sense of the \emph{eigenfunctions} of $\mathcal O$, then we could expand $s(x)$ in a basis of those eigenfunctions and then apply $\mathcal O^{-1}$ to both sides. 

The analog is this:

\begin{center}
\includegraphics[width=.7\textwidth]{figures/lec02_green01.pdf}
\end{center}

 The operator $A$ encodes the \emph{physics} of the system, the underlying dynamics. This is presumably local: it is a near-diagonal matrix coming from one or two powers of derivatives.  The ket $|s\rangle$ is the source. This is the thing that \emph{causes} the dynamics. The ket $|\psi\rangle$ is some state that we would like to determine. 

\subsection{Metrics}

Thus far we have introduced vector spaces. The dual vector space is a set of linear functions that act on elements of a vector space; these are bras/row-vectors/one-forms. Let us now introduce a new piece of machinery: a \textbf{metric}. This is also known as an \textbf{inner product} or a \textbf{dot product}. A space with a metric is called a metric space. We only state this fact to emphasize that we are \emph{adding this structure by hand}. Vector spaces don‘t come with metrics---someone makes up a metric and slaps it onto the vector space.

The \textbf{metric} is a function that takes two vectors and spits out a number. It is linear in each argument. In other words, a metric $g$ is:
\begin{align}
  g:\; V\times V\to \mathbb{R} \ .
\end{align}
Occasionally one may want a metric defined such that the output is a complex number. We thus have:
\begin{align}
  g(\alpha \vec{v} + \beta\vec{w}, \delta \vec{x} + \gamma \vec{y})
  &= 
  \alpha \delta g(\vec{v},\vec{x}) + \alpha \gamma g(\vec{v},\vec{y}) + \beta\delta g(\vec{w},\vec{x}) + \beta\gamma g(\vec{w},\vec{y}) \ .
\end{align}
One more special assumption about the metric is that it is \textbf{symmetric}:
\begin{align}
  g(\vec{v},\vec{w}) = g(\vec{w}, \vec{v}) \ .
\end{align}
In indices one may write
\begin{align}
  g &= g_{ij} \langle i | \otimes \langle j |
\end{align}
so that
\begin{align}
  g(\vec{v}, \vec{w}) = g_{ij} v^{i}w^j \ .
 \end{align}
 Here we see the usefulness of the $\otimes$ notation. It tells us that the bras and kets resolve as follows:
 \begin{align}
   g_{ij}\langle i | \otimes \langle j | \left(v^k|k\rangle\right)\left(w^\ell|\ell\rangle\right) 
   = 
   g_ij v^k w^\ell 
   \langle i | k\rangle \langle j |\ell\rangle
   = 
   g_ij v^k w^\ell  \delta^i_k \delta^j_\ell 
   = 
   g_{ij} v^{i}w^j \ .
 \end{align}
 For ordinary Euclidean space in flat coordinates, the metric is simply the unit matrix: $g_{ij} = \text{diag}(1,\cdots, 1)$. In Minkowksi space there’s a relative minus sign between space and time. In curvilinear coordinates things get ugly. 
 
Here’s the neat thing about metrics. We can take a metric and pre-load it with a vector. Given a metric $g$ and a vector $\vec{v}$, we may define a function
\begin{align}
  g(\vec v,\qquad ) \ .
\end{align}
This is simply means that the function $f(\vec{w}) = g(\vec v,\vec w)$. Observe that $f(\vec{w})$ is a linear function that takes elements of $V$ and returns a number. In other words, this is a \emph{dual vector} (row-vector, one-form). Observe what having a machine like the metric has done for us: it has allowed us to convert vectors into dual vectors:
\begin{align}
  g(\vec v,\qquad )  = g_{ij} v^i \langle j| \ .
\end{align}

Similarly, one may define an inverse metric $g^{-1}$ such that $g^{-1}g = \mathbbm{1}$. In a slight abuse of notation, the inverse metric is written with two upper indices: $g^{ij}$. Note that we do not write the `$^{-1}$.' The inverse metric will \emph{raise} the index on a lower-index object, while the metric \emph{lowers} the index of an upper-index object.\footnote{Of course: what’s really happening is that the metric has a basis $\langle i|\otimes \langle j|$ while the inverse metric has a basis $|i\rangle \otimes |j\rangle$.}


\subsection{Hermitian Conjugate}

Now that we can go between column and row vectors, thanks to the metric and its inverse, it is worth thinking a bit about what we meant by the `transpose’ operator. The transpose precisely turned a column vector $\vec{v}$ into an associated column vector $\vec{v}^T$. The generalization of this idea is the Hermitian conjugate, $^\dag$. 






%!TEX root = P231_notes.tex

\section{Function Space}
\lecdate{lec~04}

\textbf{Function spaces} are vector spaces where the vectors are functions. We introduced `histogram space' in Section~\ref{sec:histogramspace} as a crude model of a finite dimensional function spaces. An alternative finite dimensional model is the space of polynomials up to some degree, which we introduced in Section~\ref{sec:derivatives}. Our models of nature are typically continuous\footnote{This does not mean that nature is fundamentally continuous. There is a deep sense in which our models are valid whether or not nature is continuous so long as any granularity is smaller than our experimental probes.}. This requires \emph{infinite} dimensional function spaces, or Hilbert spaces.

\subsection{The Green's Function Problem in Function Space}

The Green's function can be defined by analogy to the finite-dimensional inverse transformation. The finite-dimensional linear system $A\vec v = \vec w$ can be solved by applying the inverse transformation $A^{-1}$ in the same way that the continuum (infinite-dimensional) system $\mathcal O \psi(x) = s(x)$ can be solved with the Green's function $G(x,y)$ of the operator $\mathcal O$:
\begin{align}
	v^i &= \sum_i \mat{\left(A^{-1}\right)}{i}{j} w^j
	&\Rrightarrow
	&
	&
	\psi(x) &= \int  dy\, G(x,y) s(y) \ .
	\label{eq:def:of:function:space:Greens:function}
\end{align}
We've explicitly written out the sum over the dummy index $i$ to emphasize the analogy to the integration over the dummy variable $y$. The arguments of the functions play the role of `continuum indices.'

\subsection{Differential Operators}

Linear transformations on function space are differential operators. In principle you can imagine linear transformations that are not differential operators, for example a finite translation. However, because our models of nature are typically \emph{local} and \emph{causal}, the linear transformations that we obtain from physical models are differential operators\footnote{This is not to say that finite transformations are somehow not permitted. The dynamics that govern our models of nature, however, only dictate how information is transmitted infinitesimally in space and time. Propagation forward in time by some finite interval is described by the exponentiation of infinitesimal forward time translations. This is, of course, why the time-translation operator in quantum mechanics is $e^{i\hat H t}$, where the Hamiltonian $H$ is described as a local function with perhaps one or two derivative operators.}. 

Let's write a general differential operator as:
\begin{align}
	\mathcal O = 
	p_0(x) 
	+ p_1(x) \frac{d}{dx}
	+ p_2(x) \left(\frac{d}{dx}\right)^2
	+ \cdots
	\label{eq:differential:operator}
\end{align}
where the $p_i(x)$ are polynomials. Sometimes we will write this as $\mathcal O_x$ to make it clear that the argument of the polynomials is $x$ and the variable with which we are differentiating is $x$.  
\begin{exercise}
Explain why \eqref{eq:differential:operator} is a linear operator acting on function spaces.
\end{exercise}
\begin{exercise}
A confused colleague argues to you that \eqref{eq:differential:operator} cannot possibly be `linear.' Just look at it, your colleague says: the functions $p_i(x)$ are polynomials---those aren't \emph{linear}! There are also powers of derivatives---how is that possibly linear? Explain to your colleague why the $p_i(x)$ does not have to be linear nor is one restricted to finite powers of derivatives for the operator $\mathcal O$ to be a linear operator acting on function space.
\end{exercise}
Technically \eqref{eq:differential:operator} is called a \textbf{formal operator} because we haven't specified the boundary conditions of the function space. Recall in our discretized `histogram space' in Section~\ref{sec:histogramspace} that we had to be careful about how to define the derivative acting on the boundaries of the space. A differential operator along with boundary conditions is called a \textbf{concrete operator}.

\subsection{Inner Product}

There's a convenient inner product that you may be familiar with from quantum mechanics. For two functions $f(x)$ and $g(x)$ in your function space, define the inner product to be
\begin{align}
	\langle f,g\rangle 
	=
	\int dx\, f^*(x)g(x) \ .
	\label{eq:L2:inner:product}
\end{align}
\begin{example}
Wave functions in 1D quantum mechanics obey this norm. For an infinite domain, we typically restrict to square-integrable functions meaning that $|f|^2$ goes to zero fast enough at $\pm \infty$ so that the integral $\langle f, f\rangle$ is finite. 
\end{example}
Sometimes the inner product is defined with respect to a \textbf{weight} function $w(x)$:
\begin{align}
	\langle f,g\rangle_w 
	=
	\int dx\, w(x)\, f^*(x)g(x) \ .
	\label{eq:weighted:inner:product}
\end{align}
There's nothing mysterious about inner products with weights. They typically boil down to the fact that one is not using Cartesian coordinates. 
\begin{example}
Have you met the Bessel functions? If not, you're in for a treat in your electrodynamics course. The Bessel functions satisfy a funny orthogonality relation with weight $w(x)\sim x$ because they show up as the radial part of a solution when using polar coordinates. When you separate variables, $d^2x = rdr\,d\theta$, we see that the measure over the radial coordinate $r$ carries a \emph{weight} $r$.
\end{example}
We will assume unit weight until we go to higher spatial dimensions\footnote{My dissertation focused on theories of extra dimensions. I also noticed that my weight increased in my final year of graduate school as I spent most of my time writing about extra dimensions and eating cafe pastries.}.

\subsection{Dual Vectors}

What are the `dual functions' (dual vectors, bras) in function space? These are linear functions on act on functions and spit out numbers. Taking inspiration from \eqref{sec:ket:as:pre:loaded:metric}, these are integrals that are pre-loaded with some factors. Assuming unit weight:
\begin{align}
	\langle f | = \langle f, \qquad \rangle
	= 
	\int dx \, f^*(x) \left[\text{ insert ket here }\right] \ .
\end{align}

\subsection{Adjoint operators}

What is the adjoint of a differential operator? The definition of the adjoint \eqref{eq:adjoint:definition} and the function space inner product \eqref{eq:L2:inner:product} give us a hint. We define $\mathcal O^\dag$ by the property
\begin{align}
	\int dx \, \left[\mathcal O f(x)\right]^* g(x)
	= 
	\int dx \, f^*(x) \left[\mathcal O^\dag g(x)\right] \ .
\end{align}
The strategy is: given an inner product (integral) over $f^*$ and $g$ where there is some stuff ($\mathcal O$) acting on $g$, can we re-write this as an integral with no stuff acting on $g$ and some \emph{other} stuff acting on $f^*$? If so, then the `other stuff' is the adjoint $\mathcal O^\dag$.
\begin{example}
What is the adjoint of the derivative operator, $\mathcal O = d/dx$? Assume an interval $x\in[a,b]$ and Dirichlet boundary conditions, $f(a)=f(b)=0$. There's a simple way to do this: integrate by parts.
\begin{align}
	\int dx \, \left[\frac{d}{dx} f(x)\right]^* g(x)
	=
	- \int dx \, f^*(x) \left[\frac{d}{dx}g(x)\right]
	+
	\left[f^*(x)g(x)\right]^b_a \ 
	=
	- \int dx \, f^*(x) \left[\frac{d}{dx}g(x)\right] \ . 
\end{align}
From this we deduce that
\begin{align}
	\left(\frac{d}{dx}\right)^\dag = -\frac{d}{dx} \ .
\end{align}
\end{example}

We will be especially interested in \textbf{self-adjoint} (Hermitian) operators for which
\begin{align}
	\mathcal O^\dag = \mathcal O \ .
\end{align}
This is, as we mentioned for the finite-dimensional case, because self-adjoint operators are \emph{nice}: they have real eigenvalues and orthogonal eigenvectors. Since most physical values are real eigenvalues of some operator, one may expect that the differential operators that show up in physics are typically self-adjoint.
\begin{exercise}
We saw above that the derivative operator is not self-adjoint. What is an appropriate self-adjoint version of the derivative operator? \emph{Hint: what is the momentum operator in quantum mechanics?}\footnote{\url{https://aapt.scitation.org/doi/abs/10.1119/1.9932}} 
\end{exercise}
\begin{example}
Consider $\mathcal O = -\partial_x^2$ defined on the domain $x\in [0,1]$ with the boundary conditions $f(0)=f(1)=0$. Is this operator self-adjoint? We want to check of $\langle f,\mathcal O g\rangle = \langle O f, g \rangle$. We have one trick: integration by parts. Let's see how this works.
\begin{align}
	\langle f, \mathcal O g\rangle &= - \int dx\, f^*(x)\partial^2 g(x) \ .
\end{align}
This is compared to
\begin{align}
	\langle \mathcal O f, g\rangle 
	&= -\int^1_0 dx\, \left[\partial^2 f(x)\right]*g(x) 
	\\
	&= 
	-\left.\left(\partial f(x)\right)^*g(x)\right|^1_0
	+ \int^1_0 dx \, \left[\partial f(x)\right]^* \partial g(x) 
	\\
	&= \left.f^*(x)\partial^2 g(x)\right|^1_0
	- \int^1_0 dx \, f^*(x) \partial^2 g(x)  
	\\
	&=
	- \int^1_0 dx \, f^*(x) \partial^2 g(x)  
	\ .
\end{align}
And so we see that indeed $(-\partial^2)^\dag = -\partial^2$.
\end{example}
\begin{exercise}
In the previous example, what is the significance of the overall sign of the operator? \emph{Hint: the sign doesn't matter, it's because we typically think of $-\partial^2$ and its higher-dimensional derivatives as the square of the momentum operator.}
\end{exercise}
\begin{example}\label{ex:eigenfunction:fourier}
The \textbf{eigenfunctions} $f_n$ of $-\partial^2$ defined on $x\in [0,1]$ with Dirichlet boundary conditions are simply 
\begin{align}
	f_n(x) &= A_n \sin(n\pi x) 
	&
	\lambda_n = - n^2\pi^2 \ ,
	\label{eq:fourier:basis:unit:interval}
\end{align}
where $\lambda_n$ is the associated eigenvalue and $A_n$ is some normalization that. These eigenfunctions are orthonormal in the following sense:
\begin{align}
	\langle f_n, f_m\rangle = \int_0^1 dx\, \sin(n\pi x)\sin(m\pi x) = \frac{A_nA_m}{2} \delta_{nm} \ ,
\end{align}
from which we deduce that the normalization is $A_n = \sqrt{2}$. That's basically all there is to know about Fourier series.
\end{example}
\begin{exercise}\label{exe:eigenfunction:fourier}
A function $g(x)$ defined on an interval $x\in [0,1]$ with Dirichlet boundary conditions can be written with respect to the Fourier basis \eqref{eq:fourier:basis:unit:interval}. In ket notation, the $n^\text{th}$ component of $g$ with respect to this basis is
\begin{align}
	g^n = \langle f_n| g\rangle \ .
\end{align}
Confirm that this is precisely what you know from Fourier series. In other words, we can decompose $g(x)$ as
\begin{align}
	g(x) &= \sum_n \langle f_n| g\rangle f_n(x)  \ .
\end{align}
\end{exercise}


\lecdate{lec~10}
\subsection{Completeness in Function Space}

We rarely have much to say about the unit matrix in linear algebra. However, much like when we discussed units, we can squeeze a lot out of inserting the identity in our mathematical machinations. In order to help with translate this to function space, let's review how it works in finite dimensional vector spaces. The unit matrix is $\mathbbm{1}$ and may be written:
\begin{align}
	\mathbbm{1} = \sum_i |i\rangle\langle i| \ ,
	\label{eq:unit:matrix}
\end{align}
where $|i\rangle$ and $\langle j|$ are basis (dual-)vectors. 
\begin{exercise}
Take a moment and convince yourself that \eqref{eq:unit:matrix} is true and obvious. It may be helpful to explicitly write out $|i\rangle \langle j|$ as a matrix. 
\end{exercise}
\begin{exercise}\label{ex:completeness:for:non:cartesian:basis}
Suppose you have a two-dimensional Euclidean vector space. Show that \eqref{eq:unit:matrix} is true for the basis
\begin{align}
	|1 \rangle &= 
	\frac{1}{\sqrt{2}}
	\begin{pmatrix}
	1 \\ 1
	\end{pmatrix}
	&
	|2 \rangle &= 
	\frac{1}{\sqrt{2}}
	\begin{pmatrix}
	1 \\ -1
	\end{pmatrix}
	\\
	\langle 1 | &= 
	\frac{1}{\sqrt{2}}
	\begin{pmatrix}
	1 & 1
	\end{pmatrix}
	&
	\langle 2 | &= 
	\frac{1}{\sqrt{2}}
	\begin{pmatrix}
	1 & -1
	\end{pmatrix} \ .
\end{align}
\end{exercise}
In fact, \eqref{eq:unit:matrix} defines what it means that a set of basis vectors is \textbf{complete}. You can write any vector $|v\rangle$ with respect to the basis $|i\rangle$---the components are simply
\begin{align}
	v^i = \langle i | v \rangle
\end{align}
so that 
\begin{align}
	|v\rangle = \sum_i |i\rangle \langle i | v \rangle \ ,
	\label{eq:completeness:by:inserting:1}
\end{align}
which we recognize as nothing more than `multiplying by the identity.' 
%

What does completeness look like in function space?
\begin{framed}\noindent
Let $e_{(n)}(x)$ be a set of basis functions. The basis is \textbf{complete} if
\begin{align}
	\sum_n \left[e_{(n)}(x)\right]^* e_{(n)}(y) = \delta(x-y) \ .
	\label{eq:function:space:completeness}
\end{align}
\end{framed}
Compare this \emph{very carefully} with the completeness relation \eqref{eq:unit:matrix}. The sum over $i$ in the finite-dimensional case has been relabeled into a sum over $n$ in the function space---this is just my preference\footnote{I think this is because we will deal with complex functions and I want to avoid using $i$ as an index. But if we're being honest, it's just become a habit.}. The $\mathbbm{1}$ has been replaced by a Dirac $\delta$-function, $\delta(x-y)$. Let's confirm that this makes sense. The \emph{multiply by one} completeness relation \eqref{eq:completeness:by:inserting:1} in function space is
\begin{align}
	|g\rangle 
	&= 
	\sum_n |e_{(n)}\rangle\langle e_{(n)}| g\rangle
	&
	\langle e_{(n)}| g\rangle &=
	\int dy \, [e_{(n)}(y)]^* g(y) \ .
\end{align}
We have deliberately changed the name of the integration variable to $y$ to avoid confusion; since this variable is integrated over it's simply a \emph{dummy variable} and it doesn't matter what we name it---the quantity $\langle e_{(n)}|g\rangle$ is independent of $y$ because $y$ is integrated over\footnote{By the way, this should ring a bell from our summation convention. When an upper and lower tensor index are contracted, the resulting object behaves as if it didn't have those indices: $\mat{A}{i}{j}v^j$ behaves as a vector with one upper index.}. Writing this out explicitly as functions:
\begin{align}
	g(x) &= \sum_n\left[\int dy\, e_{(n)}^*(y)g(y)\right] e_{(n)}(x) \ .
	\label{eq:complenesss:function:space:in:action }
\end{align}
The factor in the square brackets is simply $\langle e_{(n)}| g\rangle$, which is just a \emph{number}---it has no functional dependence on $x$.
If this seems unusual, please refer back to Example~\ref{ex:eigenfunction:fourier} and Exercise~\ref{exe:eigenfunction:fourier}. 

By the way, you'll often hear people (perhaps even me) say that the Dirac $\delta$ function is not strictly an \emph{function} but rather a \textbf{distribution}---this means that it only makes sense when it is integrated over. As physicists we'll sometimes be sloppy and talk about physical quantities that could be Dirac $\delta$-functions. There is \emph{never} an appropriate, measurable physical quantity that is described by a $\delta(x)$. Anything with a $\delta(x)$ is an object that was meant to be integrated over. When you imagine that a point charge density is a $\delta$-function, this is only because you will eventually integrate over it to determine the total charge. This is precisely what we saw in the charged cat in Example~\ref{eq:charged:cat}. If you ever calculate a \emph{measurable} quantity to be $\delta(x)$ check your work. If you ever find $\delta(x)^2$, then go home, it's past your bed time.

\begin{example}
One can vaguely motivate the $\delta$-function as the unit matrix by appealing to the `histogram basis' of discretized function space. In an ordinary finite-dimensional vector space, unit matrix can be written as
\begin{align}
	\mathbbm{1} = |1\rangle\langle 1 | + |2\rangle\langle 2 | + \cdots
	= \sum_{i,j}\delta_{i}^j|i\rangle\langle j| \ .
\end{align}
The Dirac $\delta$-function in histogram space is analogous to
\begin{align}
	\delta(x-x') &\to \delta_{x}^{x'}|x\rangle\langle x'| \ ,
\end{align}
where $x$ and $x'$ are discrete bins on the right-hand side. Thus for a discretized function $f = f(x_1)|x_1\rangle + f(x_2)|x_2\rangle + \cdots$, one has
\begin{align}
	\int dy\, \delta(x-y) f(y) = f(x) \longrightarrow \sum_j \delta_{x_j}^{x_i}|x_i\rangle\langle x_j| f\rangle  =  f(x_i) |x_i\rangle\ .
\end{align}



\end{example}

\subsection{Orthonormality in Function Space}

One should contrast the notion of completeness of a a basis this with that of \textbf{orthonormality} of the basis. Orthonormality is the statement that
\begin{align}
	\langle i | j \rangle = \delta^j_i \ .
\end{align}
Completeness has to do with the `outer product' $|i\rangle \langle i|$ while orthonormality has to do with the `inner product' $\langle i | i\rangle = \langle i, i\rangle$. The function space generalization of orthonormality is\footnote{If you're a purist, you'll note that $\delta_{nm}$ should really be written as $\delta^n_m$ because the dual basis vector has an upper index. While this may be true, I'm making the present notational choice because the object that we would call $\tilde{\vec{e}}^{(n)}$ really does contain $e_{(n)}^*(x)$, the complex conjugate of $e_{(n)}(x)$.}
\begin{align}
	\langle e_{(n)} | e_{(m)} \rangle = \int dx \, e_{(n)}^*(x) e_{(m)}(x) = \delta_{nm} \ .
	\label{eq:function:orthonormality}
\end{align}
\begin{exercise}
Why does \eqref{eq:function:orthonormality} have a Kronecker $\delta$ with discrete indices when \eqref{eq:function:space:completeness} has a Dirac $\delta$? Please make sure you can answer this; it establishes the conceptual foundation of the analogy between finite- and infinite-dimensional vector spaces.
\end{exercise}
For the completeness relation, we sum over the same eigenfunction label $n$ for a function and its conjugate evaluated at different continuous positions. For the orthonormality relation, we integrate over the positions of two different eigenfunction indices, $n$ and $m$. 

Do not confuse the eigenfunction label with the index of a vector. If this is confusing, please refer back to Exercise~\ref{ex:completeness:for:non:cartesian:basis}. You may be stuck thinking about basis vectors in the Cartesian basis---this is the analog of thinking about basis functions in the `histogram basis' of Section~\ref{sec:histogramspace}. What we want to do is generalize to more convenient bases, like the eigenfunctions of differential operators (e.g.~the Fourier basis for $-\partial^2$).

\subsection{Completeness and Green's Functions}

The utility of the completeness relation should be clear. If you happen to have a nice (self-adjoint) linear differential operator $\mathcal O$ with a nice (complete, orthogonal) eigenfunctions $e_{(n)}$ and eigenvalues $\lambda_n$, then we can expand any function $\psi(x)$ with respect to these eigenfunctions. Then it is easy to invert the differential equation $\mathcal O \psi(x) = s(x)$ to determine the response $\psi(x)$ to a source $s(x)$:
\begin{align}
	\psi(x) 
	&= \mathcal O^{-1}
	\sum_n \langle e_{(n)}|s\rangle e_{(n)}(x)
	= \sum_n \frac{\langle e_{(n)}|s\rangle}{\lambda_n} e_{(n)}(x) \ ,
\end{align}
where we've simply used \eqref{eq:linear:aglebra:inverse:eigenvectors}. The inner product $\langle e_{(n)}|s\rangle$ is an overlap integral between known functions:
\begin{align}
	\psi(x) &= 
	 \int dy\, \sum_n \frac{e_{(n)}^*(y) e_{(n)}(x)}{\lambda_n} s(y) \ ,
	 \label{eq:Greens:function:by:completeness}
\end{align}
where we have rearranged terms rather suggestively. This is now in the same form as our prototype Green's function example \eqref{eq:electrostatics:greens:func}. 

Referring back to \eqref{eq:def:of:function:space:Greens:function}, 
we see that our completeness relation---that is, our trick of inserting unity---in \eqref{eq:Greens:function:by:completeness} tells us an explicit form for the Green's function of a differential operator $\mathcal O$ if you know the eigenfunctions and eigenvalues of that operator:
\begin{align}
	G(x,y) &= \sum_n \frac{e_{(n)}^*(y) e_{(n)}(x)}{\lambda_n} \ .
	\label{eq:G:from:completeness}
\end{align}
This is formally an infinite sum and so is only practically useful if each term is successively smaller. 


\subsection{Green's Function by Completeness: why is this helpful?}
\label{sec:Greens:fuctions:by:completeness}

If you look at \eqref{eq:G:from:completeness} and think about our goals for the class, you may say \emph{hooray! We're done.} After all, given the Green's function $G(x,x')$ for a given differential operator\footnote{We've explicitly written the $x$ in $\mathcal O_x$ to indicate that derivatives are with respect to that variable, \emph{not} $x'$.} $\mathcal O_x$, then we know how to \emph{invert} $\mathcal O_x$. So for any source $s(x)$ and differential equation $\mathcal O_x \psi(x) = s(x)$, we can find $\psi(x)$ by
\begin{align}
	\psi(x) &= \int dx' \, G(x,x') s(x') \ .
\end{align}
The integral is over the domain on which we've defined the function space and subject to functions satisfying the boundary conditions. We interpreted the integral over $x'$ as an integral over the source configuration. Armed with an expression for $G(x,x')$, we can simply perform the overlap integral with $s(x')$---numerically if needed---and that gives us $\psi(x)$. Easy! What are we missing?

First, all of this \emph{assumed} that you know the eigenfunctions of $\mathcal O$. This is actually a fairly safe assumption. There are only so many differential operators that matter in physics, especially since the physically motivated operators are typically self-adjoint and respect many symmetries. In fact, there are so few of these that their eigenfunctions are all famous---so when you're slogging through electrodynamics dealing with spherical harmonics, Bessel functions, and Legendre polynomials---you know that these special functions are `special' because they're eigenfunctions of variations of the Laplacian that show up in physics over and over again. They are so important that ancient graduate students had to use them \emph{before} one could just plug them into \emph{Mathematica}\footnote{Have you ever heard of Gradshteyn and Ryzhik? When I was a student there was a story that most of it was written while the authors were bored in Siberia. In Cornell the theoretical physics journal club used to be called the Gradsteyn seminar because it would ``integrate the knowledge of the graduate student participants.'' (Source: Michael Peskin, private communication.) Anyway, if you've made it this far in the footnote: you should consider running a journal club with your lab/classmates. It may be the best preparation you can give yourself for being a young academic.}. All that is to say for any differential equation that you will probably ever care about in physics, the eigenfunctions are probably known and their properties are well documented\footnote{By the way, if you were expecting this class to be about the properties of Bessel functions and all that, then forget it! I find nothing fun about that. We're going to stick to good old sines and cosines because \emph{all} of the essential intuition is already there. If you deeply understand the orthogonality, completeness, projections onto trigonometric functions, then you can `read' the special functions as generalizations of the trigonometric functions for their respective differential operators. By the way, beware of any young person who seems to know the Bessel function properties \emph{a little too well}... that person has probably been through some shit.}. 

Okay, so if the identification of eigenfunctions is not a problem, why isn't \eqref{eq:G:from:completeness} the end of this course? One reason is that it is an \emph{infinite} series. The differential operator is a `matrix' in  infinite-dimensional space, so there are an infinite number of eigenfunctions that space the space. If you're like me, you really only want to deal with one or two terms---very rarely is it worth it to have to go to many more terms\footnote{One notable local exception is Prof.~Hai-Bo Yu's work on self-interacting dark matter calculations. In the resonant regime, some of these numerical results require sums over hundreds of partial waves.}. This means that the infinite sum is only practical is each successive term is a small correction to the previous terms. While this is not always the case, this may start to sound familiar to you. Let's see it in action with an example.

\begin{example}
The eigenfunctions for the angular part of the Laplacian, $\nabla^2$, in spherical coordinates are the \textbf{spherical harmonics}, $Y_{\ell m}(\theta, \varphi)$. When you tack on the radial piece, the Green's function for the Laplacian in spherical coordinates is
\begin{align}
	G(\vec{r},\vec{r}')
	&=
	\sum_{\ell=0}^\infty
	\sum_{m=-\ell}^\ell
	\frac{1}{2\ell+1}
	Y_{\ell m}(\theta, \varphi)
	Y_{\ell m}^*(\theta', \varphi')
	\frac{r_<^\ell}{r_>^{\ell+1}} \ ,
	\label{eq:greens:function:spherical:harmonics}
\end{align}
where $r_> = \text{max}(\vec{r},\vec{r}')$ and $r_< = \text{min}(\vec{r},\vec{r}')$. To be concrete you can assume that $r > r'$ so that $r_> = r$ and $r_< = r'$. This corresponds to an observer further away from the origin than the source. Remind yourself of where expressions \emph{just like this} show up in electrodynamics---for example, a charged cat curled up into a small lump near the origin of your coordinate system.

Note that \eqref{eq:greens:function:spherical:harmonics} has \emph{two} sums over eigenfunction `labels' $m$ and $\ell$. That's okay---this simply generalizes the case of a single sum. Clearly this expression has the form of a completeness relation with the radial piece tacked on.

The upshot of having his expression is that you can take \emph{any} source $\rho(\vec{r})$, such as that lump of charged cat, and write a closed form expression for the state (e.g.\ the electrostatic potential):
\begin{align}
	\Phi(\vec{r})
	&=
	\int d^3 \vec{r}'
	\frac{r_<^\ell}{r_>^{\ell+1}}
	\left[
		\sum_{\ell, m}
		\frac{1}{2\ell+1}
		Y_{\ell m}(\theta, \varphi)
		Y_{\ell m}^*(\theta', \varphi')
	\right]
	\rho(\vec{r}') \ ,
\end{align}
where the expression in the bracket has a special name:
\begin{align}
	P_\ell(\hat{\vec{r}}\cdot\hat{\vec{r}}')
	&=
	\sum_{\ell, m}
		\frac{1}{2\ell+1}
		Y_{\ell m}(\theta, \varphi)
		Y_{\ell m}^*(\theta', \varphi') \ .
\end{align}
The $P_\ell$'s are called Legendre polynomials\footnote{Once when I was teaching a class of undergraduates in electromagnetism I asked them if they knew what these special functions, $P_\ell$ were called. One of them enthusiastically shouted, \emph{ooh! Is that a Pessel function}? That's when I learned to appreciate the joy of serendipity in teaching.} In the limit where $r\gg r'$, the expression takes the following form:
\begin{align}
	\Phi(\vec{r})
	&=
	\sum_{\ell, m}
	\frac{1}{2\ell+1}
	\frac{Y_{\ell m}(\theta, \varphi)}{r^{\ell+1}}
	\left[\int d^3 \vec{r}'
	 	 	\, r'^{\ell}
	 	 		Y_{\ell m}^*(\theta', \varphi')
	 	 		\rho(\vec{r}')\right] \ ,
\end{align}
where now the term in the brackets is purely a property of the source. Do you recognize what it is? This is simply the \textbf{multipole expansion} of the charged, lumpy cat. Observe that each successive term in the sum is suppressed by an additional power of $r'/r$. As long as $r\gg r'$---that is, as long as we are far away from the charged, lumpy cat---we can approximate its electrostatic potential as the sum of a monopole term, dipole term, etc. 
\end{example}
What we see from the above example is that in the limit where there is a small parameter, the Green's function series expression \eqref{eq:G:from:completeness} coming from the completeness of eigenfunctions can be seen as a Taylor expansion.

\subsection{Patching a Green's function together}
\label{sec:patching}

There is another clever\footnote{`Clever' is not always a positive word. A mathematical technique that is \emph{clever} may have an aesthetic quality that we can appreciate, but it's not practically useful if you have to be \emph{clever} to know to use it. We would rather prefer something that is general and systematic. By the way, this is the reason that high-energy experimentalists all know how to use version control software for their thousand-person publications while theorists have a hard time working simultaneously on a draft between three people.} way of solving for Green's functions. We'll leave most of this work to your homework, but let's sketch the procedure. 

Recall that Green's functions are the analogs to inverse matrices in a finite dimensional vector space. In other words, 
\begin{align}
	A(A^{-1}) = \mat{A}{i}{j}\mat{(A^{-1})}{j}{k} = \mat{\mathbbm{1}}{i}{k} = \delta^i_k \ .
\end{align}
The infinite dimensional version of this is
\begin{align}
	\mathcal O_x G(x,x') = \delta(x-x') \ .
	\label{eq:Greens:func:as:inverse}
\end{align}
\begin{example}
Let's do a quick `sanity' check for why $\delta(x-x')$ could plausibly play the role of an identity element, $\delta^i_k$. When $\delta^i_k$ acts on a vector, it acts on each component as (writing the sum explicitly):
\begin{align}
	\sum_k \delta^i_k v^k = v^i \ .
\end{align}
Recalling that finite-dimensional indices are arguments in function space, the analog for the $\delta(x-x')$ acting on a function $f$ is
\begin{align}
	\int dx' \, \delta(x-x') f(x') = f(x) \ .
\end{align}
\end{example}
If we compare \eqref{eq:Greens:func:as:inverse} to the class of equations that we wanted to solve, $\mathcal O \psi(x) = s(x)$, we realize that the Green's function $G(x,x')$ is simply the \emph{state} $\psi$  at position $x$ coming from an idealized $\delta$-function source at position $x'$. Of course, there's no such thing as Dirac $\delta$-function sources in nature, so we emphasize that this interpretation should not be taken literally\footnote{In undergraduate electrodynamics we say that the Coulomb potential is the result of a $\delta$-function point source... but you don't actually believe that electrons are $\delta$ functions in charge do you? If you do, take some time to think about this. By the way, this is related to Exercise~\ref{ex:hydrogen:problem}.}.  Heuristically, the Green's function equation looks like:
\begin{align}
\mathcal O_x G(x,x')
&=
	\vcenter{
		\hbox{\includegraphics[width=.3\textwidth]{figures/lec10_.png}
		}}
	\ . 
\end{align}
Note that the source is \emph{zero} for everywhere. This means that everywhere to the left of $x=x'$ is described by a homogeneous equation,
\begin{align}
	\mathcal O_x G_<(x,x') = 0 \ .
\end{align}
Further, everything to the right of $x=x'$ is described by \emph{another} homogeneous equation,
\begin{align}
	\mathcal O_x G_>(x,x') = 0 \ .
\end{align}
These are different equations for \emph{different functions}: $G_<$ and $G_>$ are two different functions that obey homogeneous equations \emph{in their respective domains}. $G_<(x,x')$ is \emph{not} defined for $x>x'$. Usually solving homogeneous equations is easier\footnote{I wouldn't really know, but it seems to take less time on \emph{Mathematica} so there you go.}. 

The strategy then  is to solve for $G_<(x,x')$ and $G_>(x,x')$ as functions of $x$ and \emph{patch them together}: 
\begin{align}
	G(x,x') = 
	\begin{cases}
	G_<(x,x') & \text{ if } x<x'\\
	G_>(x,x') & \text{ if } x>x'
	\end{cases}\ .
\end{align}

Here $x'$ is just a spectator variable---we're keeping it fixed. For simplicity, you may even want to shift your coordinates so that $x'=0$. When we do this, we usually have a second order differential equation---some variant of the Laplacian because that's 99\% of what we do---so we need to have enough boundary conditions to fix our cofficients. Since we have two functions in a second order differential equation, we need \emph{four} boundary conditions. When we defined the Green's function problem, presumably we are considering functions over some interval $x,x'\in [a,b]$. This gives boundary conditions at $a$ and $b$, which may even be at $a=-\infty$ and $b=\infty$. The two additional boundaries are obtained at $x=x'$. These come from requiring the continuity of the solutions
\begin{align}
	G_<(x',x') = G_>(x',x')
\end{align}
and a `jump condition' between the first derivatives of the soltuiosn:
\begin{align}
	\lim_{\epsilon\to 0}\int_{x'-\epsilon}^{x'+\epsilon}dx \mathcal O_x G_(x,x') = 1 \ ,
\end{align}
where this comes from simply integrating the defining equation $\mathcal O_xG(x,x') = \delta(x-x')$ over a sliver around $x=x'$. Since $\mathcal O_x$ is assumed to be second order, the jump condition reduces to saying that the first derivatives of $G_<$ and $G_>$ are discontinuous at $x=x'$ by a certain amount. Applying these boundary conditions then gives a piece-wise solution for the Green's function.



\subsection{Where we're going}
\subsection{sec:ways:to:solve:G}

Our primary goal in this course is to find the Green's function $G(x,x')$ given a differential operator $\mathcal O$. There are three primary ways to do this:
\begin{enumerate}
\item \textbf{Eigenfunctions and completeness}, Section~\ref{sec:Greens:fuctions:by:completeness}. Assuming one knows the eigenfunctions of the differential operator, this gives a series solution for the Green's function. It is practically useful only when the series is convergent.

\item \textbf{Patching}, Section~\ref{sec:patching}. This method assume that one can solve the \emph{homogeneous} differential equation $\mathcal O_x G(x,x')=0$ and then produces a piece-wise solution to the \emph{inhomogeneous} differential equation that defines the Green's function, $\mathcal O_x G(x,x')=\delta(x-x')$. This is practically useful in one dimension where the boundary conditions where the pieces are connected are easy to define.

\item \textbf{Fourier transform and its cousins}. This will be the topic of the rest of our course. We convert the differential equation into an \emph{algebraic equation} in momentum space. Aspects of the causal structure of the system that are manifested in complex momentum space. Furthermore, one can use contour integrals to do the `hard work.'
\end{enumerate}

Recently I filled a hole in my undergraduate education and used a fourth method called the \emph{method of variations} to solve inhomogeneous differential equations. The method is sketched out in Appendix~\ref{app:method:of:variations}. We won't have anything further to say about that here, except that it turns out you can get a faculty job in theoretical particle physics without knowing how to use it.

\begin{example}
This problem is from Matthews \& Walker Section 9-4. Consider a unit string with frequency $k= \omega/c$ and Dirichlet boundary conditions at $x=0,1$; where we note that we are using units of `length of the string.' The differential operator describing standing waves is
\begin{align}
 	\mathcal O_x = \frac{d^2}{dx^2} + k^2 \ .
 \end{align}
Let's solve this using eigenfunctions. We know the normalized eigenfunctions from Example~\ref{ex:eigenfunction:fourier}:
\begin{align}
	f_n(x) &= \sqrt{2} \sin (n\pi x) 
	&
	\mathcal O_x f_n(x) 
	& = \left(-n^2\pi^2 + k^2\right)f_n(x) \equiv \lambda_n f_n(x) \ .
\end{align}
By the way, it should  have been \emph{obvious} that these are the eigenfunctions and eigenvalues, even though $\mathcal O_x$ is \emph{not} the same as $d^2/dx^2$. Using our completeness relation, the Green's function is
\begin{align}
	G(x,x') &= \sum_n \frac{f^*(x)f(x')}{\lambda_n}
	=
	2\sum_n\frac{\sin(n\pi x) \sin (n\pi x')}{k^2 - n^2\pi^2} \ .
\end{align}
Thus the solution to the system with some inhomogeneous source $s(x)$
\begin{align}
	\left[\frac{d^2}{dx^2} + k^2\right] f(x) &= s(x)
\end{align}
is simply
\begin{align}
	f(x) &= \int_0^1 dx' \, G(x,x') s(x') \ .
\end{align}
\end{example}

\begin{example}
Let's do the same example with the patching method. In this case we start with the equation (the analog of $A(A^{-1}) =\mathbbm{1}$):
\begin{align}
	\left[\frac{d^2}{dx^2} + k^2\right]G(x,x') &= \delta(x-x')
\end{align}
We now separate the domain into $x<x'$ and $x>x'$, with \emph{a priori} independent solutions $G_<(x,x')$ and $G_>(x,x')$:
\begin{center}
\includegraphics[width=.5\textwidth]{figures/lec11_GgGl.png}
\end{center}
Applying the Dirichlet boundary conditions at $x=0,1$ gives
\begin{align}
	G(x,x') &=
	\begin{cases}
	G_<(x,x') = a\sin(kx) & \text{ for } x < x'\\
	G_>(x,x') = b\sin\left(k(x-1)\right) & \text{ for } x > x'
	\end{cases} \ .
\end{align}
Make sure you understand why $G_>(x,x')$ has a factor of $(x-1)$ and not $x$; this is simply the boundary condition at $x=1$ without setting $b=0$.
The two coefficients $a$ and $b$ must be fixed by the matching at $x=x'$. 
To do this, integrate the second order differential equation over a sliver around $x'$:
\begin{align}
	\int_{x'-\varepsilon}^{x'+\varepsilon}
	dx \ , 
	\left[
	 \frac{d^2}{dx^2} + k^2
	\right]
	G(x,x')
	&=
	\int_{x'-\varepsilon}^{x'+\varepsilon} dx\, \delta (x-x') \ .
\end{align}
Note that the $k^2 G$ term in the integrand vanishes since it scales like $\varepsilon$. The integral of the second derivative is simple since it is simply the integral of a derivative, $\int dx\, d/dx(G') = \int d(G') = G'$ so that
\begin{align}
	\left.\frac{d}{dx}G\right|_{x'-\varepsilon}^{x'+\varepsilon}
	&=
	G'_>(x',x') - G'_<(x',x')
	=
	 1 \ .
\end{align}
This is the jump condition of the first derivative. If we integrate the jump condition once more over a sliver between $x\pm\varepsilon$ gives the continuity condition:
\begin{align}
	G_<(x',x') &= G_>(x',x') \ .
\end{align}
The jump and continuity conditions give
\begin{align}
	ka \cos(kx') + 1 &= kb \cos \left(k(x'-1)\right)
	\\
	a\sin(kx') &= b\sin \left(k(x'-1)\right) .
\end{align}
One can solve for the coefficients:
\begin{align}
	a&= \frac{\sin \left(k(x'-1)\right)}{k\sin k}
	&
	b&= \frac{\sin kx'}{k\sin k} \ .
\end{align}
Plugging this all in gives 
\begin{align}
	G(x,x') &=
	\frac{1}{k\sin k}
	\begin{cases}
	\sin kx \, \sin \left(k(x'-1)\right) &\text{ if }x < x'
	\\
	\sin kx' \, \sin \left(k(x-1)\right) &\text{ if }x > x' \ .
	\end{cases}
\end{align}
\end{example}
Observe that you find two \emph{different} expressions for $G$ by these methods. Please confirm---perhaps numerically---that these indeed represent the same function $G(x,x')$. 

\begin{exercise}
This example is from Butkov, chapter 12.1. Consider a taut string of length $L$ under the load of a weirdly-shaped rock:
\begin{center}
\includegraphics[width=.5\textwidth]{figures/lec12_eg.png}
\end{center}
The force equation for the vertical displacement of the string, $u(x)$, is
\begin{align}
	T u''(x) = F(x),
\end{align}
where $T$ is the tension and $F(x)$ is the force-per-unit-length. One may write this more simply as $u''(x) = f(x) = F(x)/T$. Show that the Green's function may be written as a piecewise function
\begin{align}
	G(x,x') &=
	\begin{cases}
	\left(\frac{x'-L}{L}\right)x  & \text{ if } x<x'
	\\
	\left(\frac{x-L}{L}\right)x' & \text{ if } x>x' 
	\end{cases}\ .
\end{align}
Suppose you replace the weird rock by a square paper weight of size $x < L$ in the middle of the string. Sketch what the paper weight on the string looks like.
\end{exercise}
%!TEX root = P231_notes.tex

\section{Complex Analysis Review}
\lecdate{lec~12}

\subsection{Teaser: why are we doing this?}

We now switch gears to complex analysis. One of the results that I hope you'll come to appreciate in your study of physics is the rich role of complex numbers (and their generalizations) in describing nature. Our study of complex analysis will focus on motivating the residue theorem to calculate integrals in complex space. This, in turn, is useful when we represent functions in \emph{momentum space} by Fourier transforming. For example, a Green's function $G(x,x')$ can be written with respect to the Fourier momentum variable $k$ by
\begin{align}
	G(x,x') &= \int \dbar k \, e^{-ikx} \tilde G(k,x') \ ,
	\label{eq:complex:intro:G:fourier}
\end{align}
where I've used a convenient notation where $\dbar = d/2\pi$. This, in turn is useful because our defining relation for the Green's function is \eqref{eq:Greens:func:as:inverse}:
\begin{align}
	\mathcal O_xG(x,x') &= \delta(x-x') \ ,
\end{align}
which we may then write (Fourier transforming both sides) as
\begin{align}
	\mathcal O_x \int \dbar k \, e^{-ikx} \tilde G(k,x') 
	=
	\int \dbar k \, e^{-ikx} e^{ikx'} \ .
	\label{eq:complex:intro:defining:G:}
\end{align}
On the right-hand side we've written the $\delta(x-x')$ in its Fourier representation. We have assumed that $\mathcal O_x$ is a polynomial in derivatives with respect to $x$:
\begin{align}
	\mathcal O_x = \sum_{n=0}^{\infty}
	p_n(x) \left(\frac{d}{dx}\right)^n
	\equiv P\left(x,\frac{d}{dx}\right) \ .
	\label{eq:complex:intro:defining:Ox}
\end{align}
What's powerful about this is that that when we apply a differential operator in $x$ onto the Fourier representation of a function, say \eqref{eq:complex:intro:G:fourier}, then the derivatives only `see' the exponential factor $e^{-ikx}$. This means that derivatives are particularly easy:
\begin{align}
	\left(\frac{d}{dx}\right)^n e^{-ikx} &= (-ik)^n \ .
\end{align}
This means that we may write the differential operator $\mathcal O_x$ in \eqref{eq:complex:intro:defining:Ox} when it acts on the Fourier basis function $e_{(k)}(x) = e^{-ikx}$ as
\begin{align}
	\mathcal O_x e^{-ikx} = \sum_{n=0}^{\infty}
	p_n \left(-ik\right)^n e^{-ikx}
	\equiv P(-ik) e^{-ikx}  \ .
\end{align}
For simplicity we have assumed that the polynomial coefficients in $x$ are simply constants, $p_n(x) = p_n$, and written $P(y) \equiv P(0,y)$.\footnote{This simplification does not affect anything in this course. For the more general case, one would want to also Fourier transform the $p_n(x)$ and use the convolution theorem to write everything as a single Fourier expansion; see e.g.~\url{https://mathoverflow.net/a/37873}.}
The defining relation for $G$,
\eqref{eq:complex:intro:defining:G:}, may in turn be simplified:
\begin{align}
	\int \dbar k \, e^{-ikx} P(-ik) \tilde G(k,x') &= \int \dbar k \, e^{-ik(x-x')} \ ,
\end{align}
which suggests a simple expression for the Fourier coefficients:
\begin{align}
	\tilde G(k,x') &= e^{ikx'} \left[P(-ik)\right]^{-1} \ ,
	\label{eq:Greens:function:Fourier:transform:heuristic}
\end{align}
where one may find $G(x,x')$ from simply taking the inverse Fourier transform. There will turn out to be a few neat features of this approach, but before getting there, let's start from the very beginning.

\begin{exercise}
This procedure is completely analogous to solving $A \vec{g} = \mathbbm{1}$ for $\vec{g}$ by expanding in eigenfunctions of $A$, $\vec g = g^k \vec{e}_{(k)}$. Then we have $A\vec{g} = \sum_k \lambda_k g^k e_{(k)}$. Follow through this analogy to argue that \eqref{eq:Greens:function:Fourier:transform:heuristic} corresponds to $g^k = c_k/\lambda_k$ where $c_k$ is identified with $e^{ikx'}$.
\end{exercise}

\begin{exercise}
For $\mathcal O_x = (d/dx)^2$, what is the value of $P(-ik)$? What is $\tilde G(k,x')$?
\end{exercise}

\lecdate{lec~13} % or Lec 11 of 2017

\subsection{Complex Numbers}

A complex number $z$ may be decomposed into real and imaginary parts
\begin{align}
	z = x + i y \ ,
\end{align}
where $x$ and $y$ are both real numbers. We may also define the complex conjugate as
\begin{align}
	\bar z = z^* = x - i y \ .
\end{align}
One may also write these variables in a polar notation,
\begin{align}
	z &= re^{i\theta}
	&
	\bar z &= re^{-i\theta} \ ,
\end{align}
where $r \in [0,\infty]$ and $\theta \in [0, 2\pi]$:
\begin{center}
\includegraphics[width=.5\textwidth]{lec13_complexvar}
\end{center}
Complex numbers live in the space of complex numbers $\mathbbm{C}$, a \emph{two-dimensional} real space with coordinates $(x,y)$ in $\mathbbm{R}^2$ augmented by an additional rule for multiplication (take two complex numbers and return a complex number) that is called the \emph{complex structure}. This is basically the definition $i^2 = -1$, though one may generalize to higher complex dimensions. 


\subsection{Complex functions}

Complex functions take complex numbers and return complex numbers. Inspired by the two-dimensional-ness of $\mathbbm{C}$, let us write a complex function as $f(z)=f(x,y)$ where $z = x+i y$. Because $f(x,y)$ is, itself, a complex number for a given $z=x+i y$, we can decompose $f$ into two real-valued functions
\begin{align}
	f(z) = f(x,y) &= u(x,y) + i v(x,y) \ .
\end{align}
This is clearly a map from $\mathbbm{C} \to \mathbbm{C}$ and so we can simply ``do multivariable calculus'' on this space. 

\begin{exercise}
The goal of this problem is to review line integrals in a 2D space. Perform the integral
\begin{align}
	\int_i^{1+i} dz \; z \,
\end{align}
with respect to the path that connects $z=i$ and $z=i+1$ by a straight line segment. This is \emph{not} a closed loop, there's no magic formula for this. Just parameterize the path by writing $z$ as a function of a real parameter. Then write out $dz$ in terms of the parameter. Do the integral. Answer: $\frac{1}{2}+i$.

 Integrals on the complex plane can seem mysterious because we end up using tricks that we develop below. This problem is to remind you that what we are doing is really a version of the Riemann sum with complex-valued chunks. 
\end{exercise}

If we forget the complex structure, then this \emph{seems} to simply be the calculus of maps between $\mathbbm{R}^2 \to \mathbbm{R}^2$ where
\begin{align}
	f(x,y) &= u(x,y) \hat{\vec{x}} + v(x,y)\hat{\vec{y}} \ .
\end{align}
In that case, we need to collect the partial derivatives. On the one hand, we may write the differential of $f$ as
\begin{align}
	df = \frac{\partial f}{\partial x} dx + \frac{\partial f}{\partial y} dy \ .
\end{align}
Please note that the partial derivatives of $f$ contain both real and imaginary parts. That is to say: $\partial f/\partial x$ generally contains both r real ($\hat x$) and imaginary ($\hat y$) pieces. 

% From Appel, Ch. 4.1.c
From the definitions of $z=x+iy$ and $\bar z = x-iy$, we may write $dz = dx+idy$ and $d\bar z = dx-idy$. This lets us rewrite
\begin{align}
	dx &= \frac{1}{2}(dz+d\bar z)
	&
	dy & \frac{1}{2i}(dz-d\bar z) \ .
\end{align}
Plugging this into our expression for the differential of $f$ gives
\begin{align}
	df &= \frac{1}{2}
	\left(
		\frac{\partial f}{\partial x}
		-i\frac{\partial f}{\partial y}
	\right)dz
	+ 
	\frac{1}{2}
	\left(
		\frac{\partial f}{\partial x}
		+i\frac{\partial f}{\partial y}
	\right)d\bar z 
	\\
	&\equiv \frac{\partial f}{\partial z}dz
	+ \frac{\partial f}{\partial \bar z}d\bar z
	\label{eq:complex:2D:deviation}
	\ .
\end{align}
From this we may identify the partial derivatives in the $z$ and $\bar z$ coordinates:



% We know that we can write the variation of $f$ as
% \begin{align}
% \delta f(x,y) &= \delta u(x,y) \hat{\vec{x}} + \delta v(x,y)\hat{\vec{y}} 
% =
% \left(
% 	\frac{\partial u}{\partial x}\delta x +
% 	\frac{\partial u}{\partial y}\delta y
% \right)\hat{\vec{x}}+
% \left(
% 	\frac{\partial v}{\partial x}\delta x +
% 	\frac{\partial v}{\partial y}\delta y
% \right)\hat{\vec{y}} \ .
% \end{align}
% Remembering our complex structure, we can write this succinctly as
% \begin{align}
% 	\delta f &= 
% 	\frac{\partial f}{\partial z}\delta z +
% 	\frac{\partial f}{\partial \bar z}\delta \bar z \ ,
% 	\label{eq:complex:2D:deviation}
% \end{align}
% where we're using
% \begin{align}
% 	\frac{\partial}{\partial z} &= 
% 	\frac{1}{2}
% 	\left(
% 	\frac{\partial}{\partial x}
% 	+
% 	\frac{\partial}{\partial (iy)}
% 	\right)
% 	&
% 	\frac{\partial}{\partial \bar z} &= 
% 	\frac{1}{2}
% 	\left(
% 	\frac{\partial}{\partial x}
% 	+
% 	\frac{\partial}{\partial (-iy)}
% 	\right) 
% 	\\
% 	&=
% 	\frac{1}{2}
% 	\left(
% 	\frac{\partial}{\partial x}
% 	-i
% 	\frac{\partial}{\partial y}
% 	\right)
% 	&
% 	\frac{\partial}{\partial \bar z} &= 
% 	\frac{1}{2}
% 	\left(
% 	\frac{\partial}{\partial x}
% 	+i
% 	\frac{\partial}{\partial y}
% 	\right) 
% 	\ .
% 	\label{eq:ddz:ddzst}
% \end{align}
\begin{align}
	\frac{\partial}{\partial z} &= 
	\frac{1}{2}
	\left(
	\frac{\partial}{\partial x}
	-i
	\frac{\partial}{\partial y}
	\right)
	&
	\frac{\partial}{\partial \bar z} &= 
 	\frac{1}{2}
	\left(
	\frac{\partial}{\partial x}
	+i
	\frac{\partial}{\partial y}
	\right) 
	\ .
	\label{eq:ddz:ddzst}
\end{align}
The fact that partial derivatives combine like this may be familiar\footnote{This comes from the underlying mathematical structure. The partial derivatives are basis vectors for the \emph{tangent space} at some point $(x,y)$. These partial derivatives ask: how much does the function I'm acting on change if I move by one unit in this direction?}. 

So far we have emphasized that $f(x,y)$ behaves like a function in two real dimensions. Critically, in some ways we want to think about functions $f(z)$ that are functions of `one' [complex] dimension rather than two real dimensions. In two real dimensions, calculus came with a notion of directional derivative. When we wanted the rate of change for a function, we had to ask \emph{in what direction} are we checking the rate of change. At a given function may change by a positive amount in the $x$-direction, but a negative amount in the $y$-direction, for example. This is in contrast to one-dimensional calculus where each point had an unambiguous derivative that represented how much the function changed in the positive $x$ direction with respect to a point $x_0$:
\begin{align}
	df(x) = \frac{d f(x_0)}{d x} (x-x_0) + \cdots \ .
\end{align}
This is in contrast to the change in a general complex function \eqref{eq:complex:2D:deviation}. In fact, if we explicitly wrote some expansion with respect to a point $z_0$, it looks a little funny:
\begin{align}
	df(z) &= 
	\frac{\partial f}{\partial z}(z-z_0) +
	\frac{\partial f}{\partial \bar z}(\bar z-\bar z_0) + \cdots \ .
\end{align}
What is going on here? Why should $f(z)$ be a function of one complex variable $z$ but the expansion seems to depend not only on $(z-z_0)$ but also $(z-z_0)^*$? 

\subsection{Analytic complex functions are nice}

This leads us to a definition of ``nice'' complex functions. The following terms are all more-or-less equivalent for this sense of nice-ness: \textbf{[complex] differentiable}, \textbf{analytic}, \textbf{holomorphic}, \textbf{regular}. This is simply the restriction that the $\partial f/\partial \bar z$ term should vanish so that the variation $f(z)$ only depends on $\delta z$ and not $\delta\bar z$. In other words, an analytic function admits usual single-dimensional definition of the derivative,
\begin{align}
	\frac{df(z_0)}{dz} &= \lim_{z\to z_0} \frac{f(z)-f(z_0)}{z-z_0} \ .
	\label{eq:df:dz:analytic:def}
\end{align}
The key aspect of this is that there is only \emph{one} value of $df/dz$ at $z_0$ no matter how you approach $z_0$. In other words, it doesn't matter if $\delta z = z-z_0$ is coming from the positive/negative real/imaginary direction. The value of $df/dz$ is the same. 

\begin{example}
We cannot overemphasize the interpretation of analyticity here. The left-hand side of \eqref{eq:df:dz:analytic:def} is supposed to be a \emph{unique number}, analogous to the slope of $f$ at $z_0$. However, on the right-hand side you can see that the denominator is very different if we take $z = z_0+ i \epsilon$ or $z= z_0 + \epsilon$ for real $\epsilon$. In the first case the denominator gives an overall factor of $-i = i^{-1}$ to $df(z_)/dz$, in the second case it does not. Thus the idea of having a unique `slope' at $z_0$ hinges on requiring $f(z)$ to be sufficiently well-behaved that it does not matter how $z$ approaches $z_0$.
\end{example}

For most of these lectures we will use the term \textbf{analytic} to refer this property. Let us see what happens when we impose $df/d\bar{z} = 0$ onto the component functions $u(x,y)$ and $v(x,y)$:
\begin{align}
	\frac{1}{2}\left(\frac{\partial}{\partial x} + i \frac{\partial}{\partial y}\right)
	\left[u(x,y)+iv(v,y)\right]
	&= 
	\frac{1}{2}
	\left( \frac{\partial u}{\partial x} - \frac{\partial v}{\partial y} \right)
	+
	\frac{i}{2}
	\left( \frac{\partial u}{\partial y} + \frac{\partial v}{\partial x} \right)
	= 0 \ .
\end{align}
This implies the \textbf{Cauchy--Riemann equations}, which are equivalent to the function $f$ being analytic:
\begin{align}
	\frac{\partial u}{\partial x} & = +\frac{\partial v}{\partial y}
	&
	\frac{\partial u}{\partial y} & = -\frac{\partial v}{\partial x} \ .
\end{align}

Thus far we've talked about functions being analytic or not. It turns out that analytic functions are \emph{so} nice that there really aren't that many physically relevant functions that are analytic \emph{everywhere}\footnote{The analog here is people who are nice. Most people are nice most of the time. But very few people are nice \emph{all} of the time. I suspect Fred Rogers may be one of the few who could plausibly be \sout{analytic} \emph{nice} everywhere.}. Instead, we will often talk about functions that are analytic in most places but are not analytic (differentiable) in other places. Indeed, the \emph{non}-analyticity of Green's functions is a key result in this class.

\subsection{Analyticity is Differentiability}

We have motivated the Cauchy--Riemann equations from the notion of being independent of $\bar z$. This is a useful crutch and fits our theme of motivating rather than rigorously proving results. Let us be clear, however, that the underlying definition of analyticity is that:
\begin{quote}
a function that is analytic at a point $z_0$ is differentiable at $z_0$.
\end{quote}
A function is differentiable at a point $z_0$ if its derivative, \eqref{eq:df:dz:analytic:def}, has a well-defined \emph{finite} value. A function that is itself infinite at a point cannot have a well-defined limit \eqref{eq:df:dz:analytic:def}, and thus are \emph{not} analytic at that point.
We distinguish this from $df/d\bar z =0$ because of the importance of \emph{singular} functions in physics. 
\begin{example}
The following function is non-analytic at $z=2i$ and $z=1$:
\begin{align}
	f(z) = \frac{3z}{(z-2i)(z-1)} \ .
\end{align}
It may seem like $df/dz^* = 0$ at these points because $f(z)$ is independent of $\bar z$. However, one cannot say that $df/\bar dz=0$ because the limit definition analogous to \eqref{eq:df:dz:analytic:def} is not defined. 
\end{example}
\begin{exercise}
The notion of `independent of $\bar z$' sometimes causes more confusion than its worth. After all, $z$ and $\bar z$ are clearly \emph{not} independent variables. Given $z$, you know exactly what $\bar z$ is. The notion in which we \emph{pretend} that $z$ and $\bar z$ are independent is that
\begin{align}
	\frac{\partial z}{\partial \bar z} = 
	\frac{\partial \bar z}{\partial z} = 0 \ ,
\end{align}
which we understand to come from an interpretation of $\mathbbm C \sim \mathbbm R^2$. Use \eqref{eq:ddz:ddzst} to prove the above partial derivative relations.
\end{exercise}



\subsection{A geometric point of view (optional)}
\label{sec:analytic:geometric}

For culture, let us comment on a geometric perspective of analyticity/complex-differentiability. The differential of a function, $df$, is a \textbf{differential one-form}. If you recall some of our nomenclature for dual vectors, a one-form is a kind of `row vector': it is a linear function that takes in a vector and spits out a number.\footnote{The number that is spit out is unique for each spacetime point $z_0$, but is independent of which direction on the complex plane one is probing the `slope' of the function $f$.} If $df$ is such a dual vector, what is the vector space upon which it acts?

Consider the differential of $f$ at a specific point $z_0$ in the complex plane: $df(z_0)$. This is not a number---it is a dual vector. In order to get a number from this, you have to feed it a vector. What are the vectors? A vector is an element of the tangent plane\footnote{The tangent plane is exactly what it sounds like: if you have a curved surface like a globe, the tangent plane at $z_0$ and glued a rigid piece of cardboard to one point, $z_0$. Now imagine that the piece of paper has graph paper on it.} of the complex plane at $z_0$: $T_{z_0}\mathbbm{C}$. We can sketch this if you allow me to artificially `curve' the complex plane to help make it easy to distinguish:
\begin{center}
\includegraphics[width=.5\textwidth]{figures/lec13_mani.png}
\end{center}
Using the jargon of differential geometry, we call the complex plane the \emph{base manifold}. The set of all tangent spaces is called a \emph{tangent bundle}\footnote{More general types of vector spaces can be attached to each point of the base manifold. These are called fiber bundles. They are the underlying mathematical structure of mechanics and reflect why Hamilton's equations are so damn special. They are also the mathematical structure that defines gauge theories and so address the question ``what is a force?''}. Each tangent space is a vector space attached to a single point of the base space. 

The object $df(z_0)$ is a dual vector on the vector space\footnote{The standard notation is to write a dual vector of a space $V$ as a vector in the dual space $V^*$, so one could write $df(z_0) \in T^*_{z_0}(\mathbbm{C})$.} $T_{z_0}(\mathbbm{C})$.  The vectors of this space are `velocities' of trajectories that pass through $z_0$. For now let's think of them as quantities
\begin{align}
	\vec v = v^1 + i v^2 = \Delta x + i \Delta y \ .
\end{align}
where we identify $v^1 = \Delta x$ and $v^2 = \Delta y$. Alternatively, $\vec{v}$ can be thought of as a `velocity' along some trajectory $\gamma(t)$ on the space $\mathbbm{C}$. The velocity in the real direction is $v^1 = d\gamma^1(t)/dt$ and the velocity in the imaginary direction is $v^2 = d\gamma^2(t)/dt$, where $\gamma = \gamma^1 + i \gamma^2$.\footnote{This is a vector space with $\vec{e}_{1} = 1$ and  $\vec{e}_{2} = 2$. The only difference from $\mathbbm{R}^2$ is that the vector space $\mathbbm{C}$ is equipped with a complex structure: the rule $i^2 = -1$.}

\begin{center}
\includegraphics[width=.5\textwidth]{figures/lec13_tanvec.png}
\end{center}
Then we may write the linear action of $df(z_0)$ on $\vec v \in T_{z_0}(\mathbbm{C})$ as
\begin{align}
	df(z_0)\left[\vec{v}\right] &=
	f'(z_0)\left(v^1 + i v^2\right)
	% \Delta x + f'(z_0)i\Delta y 
	\ ,
	\label{eq:diff:geo:CR:1}
\end{align}
where we assume that $f$ is analytic so that $f'(z_0)$ is unambiguously defined (independent of the direction of $\delta z$) as in \eqref{eq:df:dz:analytic:def}. 

However, if we think about this complex space as a two-dimensional real space, the following is also true:
\begin{align}
	df(z_0)\left[\vec{v}\right] &=
	\left.\frac{\partial f}{\partial x}\right|_{z_0} \Delta x
	+
	\left.\frac{\partial f}{\partial y}\right|_{z_0} \Delta y
	=
	\left.\frac{\partial f}{\partial x}\right|_{z_0} v^1
	+
	\left.\frac{\partial f}{\partial y}\right|_{z_0} v^2
	\label{eq:diff:geo:CR:2}
\end{align}
Comparing \eqref{eq:diff:geo:CR:1} and \eqref{eq:diff:geo:CR:2} gives 
\begin{align}
	\frac{\partial f}{\partial x} = f'(z_0) = - i\frac{\partial f}{\partial y} \ .
\end{align}
Recalling that $f= u+iv$ then gives
\begin{align}
	\frac{\partial u}{\partial x}
	+ i
	\frac{\partial v}{\partial x}
	=
	-i
	\frac{\partial u}{\partial y}
	+
	\frac{\partial v}{\partial y} \ .
\end{align}
Setting the real and imaginary parts equal to one another give, as expected, the Cauchy--Riemann equations. 

\subsection{Analytic and harmonic (optional)}

One more comment is in order. If a function $f(z) = u(x,y)+ i v(x,y)$ is analytic, then its real and imaginary parts are independently 2D harmonic with respect to $\mathbbm{R}^2$:
\begin{align}
	\partial_x^2 u(x,y) + \partial_y^2 u(x,y) &= 0
	&
	\partial_x^2 v(x,y) + \partial_y^2 v(x,y) &= 0 \ .
\end{align}
Harmonic functions show up all over the place---though admittedly you probably care mostly about three-dimensional harmonic functions. The above realization may be helpful, however, for 3D systems for which the third dimension is trivial. For example, you may want to consider fluid flow across an airplane wing. One can take the limit where the wing is infinitely long so that the fluid dynamics reduces to the two-dimensional motion around a cross section of the wing. Alternatively, maybe you care about an electrostatic system which are similarly `effectively' 2D. In this case you can `promote' your real harmonic variable (a fluid dynamics potential or electrostatic potential) to an analytic function and then bear the full power of complex analysis to attack the problem. One of the most fascinating---if outdated---methods is called \textbf{conformal mapping} by which one can solve for a harmonic function with weird boundary conditions by mapping to much simpler boundary conditions. This topic is beautiful and gives a first, intuitive example of \emph{conformal transformations} which are a pillar of theoretical physics (in any discipline). The technique is how engineers in the pre-personal computer era designed airplane wings. For experimentalists, it is pretty neat that \acro{UCR}'s own Nathan Gabor has created an analogous micromagnetic `solver' of this very same problem\footnote{\url{https://arxiv.org/abs/2002.07902}}.

\subsection{Complex functions as maps}

A complex function $f(z)$ takes in a complex number and returns a complex number. These are maps of the complex plane to itself. It is useful to build an intuition for this rather trivial-sounding statement: complex functions `deform' the complex plane. 

\begin{example}
Consider the function that multiplies a complex number by a constant phase,
\begin{align}
	f(z) = e^{i\theta} z \ .
\end{align}
This takes each point and rotates it in the counterclockwise direction by $\theta$. 
\begin{center}
% \includegraphics[width=.5\textwidth]{figures/lec13_map1.png}
\includegraphics[width=.4\textwidth]{figures/Complex_01_rot.pdf}
\end{center}
\end{example}
We can see this more viscerally by imagining the action on the image of a cat:
\begin{center}
\includegraphics[width=.5\textwidth]{figures/Complex_03_rot.pdf}
\end{center}

\begin{exercise}
What happens to points under $f(z) = z+ z_0$?
\end{exercise}


\begin{example}
Now consider the simplest non-trivial polynomial, the function that squares its argument:
\begin{align}
	f(z) =  z^2 \ .
\end{align}
For a point $z = \rho e^{i\theta}$, $f(z)  = \rho^2 e^{2i\theta}$. This squares the modulus (length) of each point and rotates according to how much the point is already rotated relative to the positive real axis.
\begin{center}
% \includegraphics[width=.7\textwidth]{figures/lec13_map2.png}
\includegraphics[width=.7\textwidth]{figures/Complex_03_sq.pdf}
\end{center}
Points that are inside the unit circle get pulled closer to the origin while points outside the unit circle are pushed away from the origin.  Points in the first quadrant (positive real and imaginary parts) are sent to points in the upper half plane (positive imaginary part).
\end{example}

In the above example of $f(z)= z^2$, note that some points in the domain are mapped to the same points in the image: $f(-1) = f(1) = 1$. That's perfectly fine: different points can be mapped to the same point under a function---there's only a problem when the same point is apparently mapped to different points in the image. But that never happens... \emph{right?}

\begin{exercise}
Show that under the map $f(z)=z^2$, a vertical line gets mapped onto a parabola:
\begin{center}
\includegraphics[width=.7\textwidth]{figures/lec13z2.jpg}
\end{center}
Figure from Matthews and Walker, points $a$, $b$, etc.\ are mapped onto $a'$, $b'$ etc. The $W$-plane corresponds to the image of the complex plan under $z^2$.
\end{exercise}

Now let's consider a curious case. What about the square root function?
\begin{align}
	f(z) = \sqrt{z} \ .
\end{align}
Something very weird happens here. Consider a closed path $\gamma(t) = e^{it}$ with path parameter $t \in [0,2\pi]$. This just means consider a bunch of points corresponding to $\gamma(t)$ with a bunch of values of $t$ in the specified range. Clearly this corresponds to a circle. However, when we put those points through the square root function, these only map onto \emph{half} of the circle. 
\begin{center}
% \includegraphics[width=\textwidth]{figures/lec13_map3.png}
\includegraphics[width=.7\textwidth]{figures/Complex_04_log.pdf}
\end{center}
The technical fact that this happens is not surprising at all, you can see that $f(\gamma(t)) = e^{it/2}$ which only spans the upper half circle. But a deeper question bubbles to the surface: if we're thinking about complex functions as maps of the complex plane to itself, what does it mean that $f(z) = \sqrt{z}$ appears to have `lost' the entire lower half of $\mathbbm{C}$? Or more strangely, how is it that a \emph{closed path} has been mapped to an \emph{open path} with endpoints?

It's clear that if we wanted to reach the \emph{lower} half plane in the iamage, we'd need $\theta \in [0, 4\pi]$. Algebraically that means that $f(\gamma(t)) = e^{it/2}$ covers the entire complex plane \emph{only} if we allow angles from 0 to $4\pi$
\footnote{You may be familiar with another case where `rotating by $2\pi$' only takes you halfway---spinors in relativistic quantum mechanics. You should know that the `double cover' nature of the spinor is (to the best of my view) independent of the present discussion of Riemann sheets. That double cover has to do with the universal cover of the Poincar\'e group. I refer to the literature on the Wigner theorem, see e.g.\ volume 1 of Weinberg's \emph{Quantum Theory of Fields}. For an unrelated cute spinor paper, see \url{https://www.jstor.org/stable/2318771}.}. In some sense, this is totally ridiculous, since $\rho e^{i\theta} \equiv \rho e^{i(\theta+2\pi)}$. I used the $\equiv$ sign because the two sides really are \emph{that} equal.

If, say, $z_1=e^{i\pi/3}$ and $ z_2 e^{7i\pi/3}$ are supposed to be the \emph{same} point, $z_1\equiv z_2$, how is it that $f(z_1) = e^{i\pi/6}$ and $f(z_2) = e^{7i\pi/6}$ so that $f(z_1)\neq f(z_2)$?
\begin{center}
\includegraphics[width=.7\textwidth]{figures/Complex_01_log.pdf}
\end{center}
Stated differently, functions are supposed to be single valued. It appears that $f(z)=\sqrt{z}$ is \emph{multivalued} since the same point $z_1=z_2$ is mapped onto two different points. One way to do this is to \emph{define} additional structure and extend the domain (pre-image) of $f$. In order to do this, we take \emph{two} copies of the complex plane and stitch them together:
\begin{center}
% \includegraphics[width=.8\textwidth]{figures/lec13_map4.png}
\includegraphics[width=.7\textwidth]{figures/Complex_07_2sheet.pdf}
\end{center}
Each copy of the complex plane is called a \textbf{Riemann sheet}. We `glue' these sheets together by defining that $\rho e^{i(\theta+2\pi)}$ of one sheet maps onto $\rho e^{i\theta}$ of the other sheet. Implicitly this chooses the positive real axis as a place where we \emph{cut} each sheet and glue them together. We call this a \textbf{branch cut}. One could, of course, have chosen the domain of each sheet to be any interval, say $\theta \in [-\pi,\pi]$ and picked a different branch cut ($\theta_\text{branch} = \pi$). 
\begin{exercise}
Is $f(z)=\sqrt{z}$ analytic on the extended complex plane (two Riemann sheets glued together at, say, $\theta=2\pi$)?
\end{exercise}
Take a moment to ponder the above exercise. As you may suspect, the relevant question isn't whether a function $f(z)$ `is analytic,' but rather \emph{where} is that function analytic? Is it analytic (differentiable) anywhere? Sure---when stay within one Riemann sheet, say $\theta\in (0,2\pi)$ and staying away from the boundary, $f(z)=\sqrt{z}$ is perfectly differentiable. Further, there's nothing special about the branch cut---we could have placed it anywhere. Thus it doesn't seem like there's \emph{any} place at which $f(z)$ is \emph{not} analytic. 

This should make sense: analyticity is about whether a function is differentiable at a given point. This is inherently a \emph{local} notion. The misbehavior of $f(z)=\sqrt{z}$---the necessity of Riemann sheets and a branch cut---are \emph{global} issues when we try to explore the \emph{whole} space. The interplay between local properties (like derivatives and Taylor expansions) and global properties (like integrals over a closed surface or topology) is a critical theme in mathematical physics. The function $f(z) = \sqrt{z}$ is analytic everywhere, but that doesn't mean we don't have to be careful. One of the lessons from this section is that the existence of a branch cut can be critically important if you happen to take trajectories that are \emph{loops} in the complex plane. For those who know where this is going: whenever there are branch cuts, you have to be careful with your contour integrals. This typically happens whenever you have a function that is not some series of integer powers of $z$.

As a final example, let's consider the complex logarithm. We'll use the notation of Byron and Fuller and write $\log$ to denote the complex natural logarithm and $\ln$ to denote the `usual' natural logarithm acting on real numbers. For $z=r e^{i\theta}$, the complex logarithm satisfies
\begin{align}
	\log z &= \ln r + i \theta
	\\
	\log(z_1z_2)
	&=\ln(r_1r_2) + i (\theta_1+\theta_2)
	= \log z_1 + \log z_2 \ .
\end{align}
\begin{example}
What does the domain of the complex logarithm look like? As a function, $f(z) = u(z) + i v(z)$, we see that $v(z) = \theta$ and $v(z_1z_2)=\theta_1+\theta_2$. This means that if we take a trajectory that keeps going around the origin, $v(z)$ just keeps increasing\footnote{This is where you can start humming `stairway to heaven.'}. Each time you go around you must be on a new Riemann sheet. 
\end{example}
Rather than having an infinite number of Riemann sheets, an alternative way of thinking about this is to imagine an infinite number of equally valid `complex logarithm' functions labeled by $n$:
\begin{align}
	f_n(z) &= \ln r + i \theta + 2\pi n i \ .
\end{align}
\begin{center}
\includegraphics[width=\textwidth]{figures/lec13_map5.png}
\end{center}
The image of the $f_n(z)$ corresponds to a horizontal strip with $v\in \left(2\pi n, 2\pi(n_1)\right)$.
\begin{example}
Is the complex logarithm analytic? Everywhere but at $z=0$.
\end{example}
Let us end by repeating the main point of introducing branch cuts:
\begin{framed}
\begin{center}
Do not integrate across a branch cut!
\end{center}
\end{framed}

\subsection{Integration on the Complex Plane}
% StoGo 17.2.1

Let's now focus on integrating analytic functions. Oops, I just misspoke. What I meant to write was that we want to \emph{integrate functions in a region of $\mathbbm{C}$ where those functions are analytic.} Rarely are functions simply `analytic' or `not-analytic.' The functions we care about will be analytic in most places, but non-analytic in others. 

When we integrate on the complex plane, $\mathbbm{C}$, we have to define a \textbf{contour}---this is just the curve, $C$, that we are integrating over. A contour has an \emph{orientation}: the direction along the curve over which one is integrating. Recall that for single variable real calculus,
\begin{align}
	\int_a^b dx\; f(x) = - \int_b^a dx\; f(x) \ ,
\end{align}
you get a relative minus sign if you integrate in the opposite direction.

Integrating along a complex contour is just like a ``line integral'' in $\mathbbm{R}^2$: you are doing a one-dimensional integral in a two[ish]-dimensional space. An integral over the contour $C$ is:
\begin{align}
	\int_C dz\, f(z) &= 
	\int_C (dx + i dy) \ , \left[u(x,y)+ i v(x,y)\right]
	\\
	&= 
	\int_C \left[u(x,y)dx - v(x,y)dy\right] 
	+ i \int_C \left[u(x,y)dy + v(x,y)dx\right] \ .
\end{align}
Written this way, we are simply doing calculus on $\mathbbm{R}^2$ and keeping track of funny factors of $i$. 
%
It may be helpful to remind ourselves what this means.
\begin{example}\textbf{Integration on the real line.}
As a reminder, let's review integration along the real line, pictorially. Usually we think of the Riemann sum as follows:
\begin{center}
\includegraphics[width=.5\textwidth]{figures/Complex_06_RRiem.pdf}
\end{center}
A more appropriate picture is to draw $f(x)$ on a separate real line:
\begin{center}
\includegraphics[width=.7\textwidth]{figures/Complex_06_RRiem2.pdf}
\end{center}
The integral is a sum over the function evaluated at some point $f(x_i)$ multiplied by the difference $(x_i-x_{i-1})$. We usually call the difference $\Delta x$ and assume that it is constant and positive. Note, however, that $(x_i-x_{i-1})$ has an \emph{orientation}: there's a sense of direction along the real line. If you go in the opposite direction, you get an overall minus sign.
\end{example}



The integral along a curve can be thought of as two-dimensional version of a Riemann sum. By comparison to the above example, the appropriate picture is this:
\begin{center}
% \includegraphics[width=\textwidth]{figures/Lec_2017_12_integral.png}
\includegraphics[width=.7\textwidth]{figures/Complex_06_CRiem.pdf}
\end{center}
Note that the complex plane on the right-hand side has coordinate $u(x,y)$ and $v(x,y)$ since $f(z) = u(x,y) + i v(x,y)$. We have written down the generalization of the Riemann sum to $\mathbbm{C}$. 

The one-dimensional real-valued Riemann sum approximated the integral as a sum of terms $f(x_i)\Delta x$. In more than one dimension, the quantity $\Delta x$ is promoted to a `vector'\footnote{As I write this I feel sick to my stomach. In fact, $\Delta x$ is \emph{not} promoted to a `vector,' though it is promoted to something that points in a direction. More properly, it is a [differential] one-form which is more like an \emph{dual vector} in the sense referenced in Section~\ref{sec:analytic:geometric}.}. In other words, we may write
\begin{align}
	\int_C dz\, f(z) = \sum_i \Delta z_i \, f(z_i) \ ,
	\label{eq:complex:integral:Riemann}
\end{align}
where $\Delta z$ has a real and imaginary part: it \emph{points} to a direction in the complex plane. In the picture above, the left-hand side shows the curve $C$ on which we are integrating. Consider some point $z_i$ on the curve. That point is mapped by the function $f$ to a point $f(z_i) = u(z_i)+iv(z_i)$. At that point, the curve also has a differential element, $\Delta z_i$ which is tangent to the curve $C$ at point $z_0$ and has some fixed infinitesimal length for all $i$. The \emph{direction} of $\Delta z$ matters. It contributes to the overall phase of the term $\Delta z_i \, f(z_i)$.
\begin{example}
Consider the function $f(z) = z^2+i$.  Consider a specific point, $z_0 = 1+i$. If we consider different curves $C_i$ that pass through $z_0$, the integral $\int_{C_i}dz\,f(z)$ will have different contributions at $z_0$ depending on the \emph{direction of the curve} as it passes through $z_0$. To see this, consider the specific term in the sum \eqref{eq:complex:integral:Riemann} corresponding to $z_i = z_0$. If the curve $C$ happens to be moving in the positive vertical (pure imaginary) direction, then the contribution to the sum from this point is
\begin{align}
	\int_C dz\, f(z) = \cdots + i\Delta y\left[(1+i)^2 + i\right] + \cdots \ .
\end{align}
However, if the curve $C$ happens to be moving in the horizontal (pure real) direction, the contribution to the sum from this point is
\begin{align}
	\int_C dz\, f(z) = \cdots + \Delta x\left[(1+i)^2 + i\right] + \cdots \ .
\end{align}
\end{example}
\begin{exercise}\label{eq:fundamental:theorem:calculus} \textbf{(Important!)}
Suppose $f$ is itself a derivative of an analytic function, $f(z)=dF(z)/dz$.
Convince yourself that if $C$ is a smooth connected path between two points $z_0$ and $z_N$, then the integral of $f(z)$ over $C$ is
\begin{align}
	\int_C dz\, f(z) &= F(z_N) - F(z_0) \ ,
\end{align}
as you would expect from real-number calculus. This fact probably has some pompous name, like the fundamental theorem of calculus. \textbf{Hint}: write out the Riemann sum.
% \footnote{By the way, this all a generalization of Stokes' theorem: the integral of the derivative of some function $f$ over some domain $D$ is simply the function evaluated on the boundary of the domain, $\partial D$. The $\partial D$ is notation for `boundary of $D$.' The fancy way to write this is
% \begin{align}
% 	\int_D df &= \int_{\partial D} f \ .
% \end{align}
% This statement holds for $f$ as a differential $n$-form (the generalization of a one-form and related to an $n$-index tensor) and $D$ is an $n$-dimensional domain. Even if you are not familiar with this nomenclature, please note the qualitative similarities with the integral of a derivative in 1D $\int dx\, (df/dx)$---which is simply the difference of $f$ at the endpoints of the domain, the integral of a curl in 2D---which is related to the `circulation' around the boundary of the 2D domain, and the integral of a divergence in 3D---which is simply related to the flux across the enclosing surface. Vector calculus isn't hard---it's just weird when we treat the 2D and 3D cases as different from the general $n$-dimensional case. 
% }.  
\end{exercise}


 

\subsection{Cauchy Integral Theorem}
% Cahill p. 160 for a sketch

As we may have referred to earlier---function that are analytic everywhere are too nice. Have you ever read a novel where everyone just got along nicely? Not very interesting. Complex functions are the same---it turns out that integrals of functions around closed curves in domains where they are analytic end up being zero. 

\subsubsection{Little tiny circles}

Let's show this starting from the simplest possible case. Consider a function $f$ that is analytic in some region $R\in \mathbbm C$. The boundary of this region is a curve $C = \partial R$. First consider the integral of $f$ around a small circle of radius $\varepsilon$ around some point $z_0$:
\begin{center}
\includegraphics[width=.7\textwidth]{figures/Lec_2017_12_circle.png}
\end{center}
In other words, $C$ can be parameterized by the angle $\theta$. We can write points $z$ on the curve as
\begin{align}
	z(\theta) &= z_0 + \varepsilon e^{i\theta} 
	&
	dz &= i \varepsilon e^{i\theta} d\theta \ .
\end{align}
Then the integral around the little circle around $z_0$ is
\begin{align}
	\oint_C dz\, f(z) &= \int_0^{2\pi} i \varepsilon e^{i\theta} d\theta \, f\left( z_0 + \varepsilon e^{i\theta} \right) \ .
\end{align}
Now we observe that since $f$ is analytic, we can differentiate it---which means we can write it as a Taylor expansion:
\begin{align}
	f(z)
	 = f(z_0) + f'(z_0)(z-z_0) + \mathcal O(\varepsilon^2) \ ,
\end{align}
where we recognize that $z-z_0 = \varepsilon e^{i\theta}$. Plugging this in gives
\begin{align}
	\oint_C dz\, f(z) &= 
	\int_0^{2\pi} i \varepsilon e^{i\theta} d\theta \, f(z_0) 
	+
	\int_0^{2\pi} i \varepsilon^2 e^{2i\theta} d\theta \, f'(z_0) 
	+ \cdots
	\ .
\end{align}
Observe that the only $\theta$-dependence in the integrand shows up in factors of $e^{n i\theta}$ for positive integers $n$. However, we note that
\begin{align}
	\int_0^{2\pi} d\theta e^{in\theta} 
	= 
	\frac{1}{in} \left(e^{2\pi i n} - e^{0}\right)
	= 0 \ .
	\label{eq:complex:theta:integral:trivial}
\end{align}
What we find is that 
\begin{align}
	\oint_C dz\, f(z) &= 0
\end{align}
for the contour $C$ being a small circle around some arbitrary point $z_0$ inside the region $R$ in which $f$ is analytic. 
\begin{exercise}
Convince yourself that it didn't matter that $C$ is a small circle. It could have been any small shape. 
\end{exercise}
Notice that we did not rely on $\varepsilon\to 0$ in order to motivate this argument. The circle didn't actually have to be small. We used $\varepsilon\to 0$ to motivate the idea of truncating the Taylor series. But armed with \eqref{eq:complex:theta:integral:trivial}, we realize that \emph{every} term in the Taylor series will vanish, no matter how high the power since that simply corresponds to some larger positive integer $n$. At this point you should start to wonder whether there may be any loopholes in this argument.

\subsubsection{Finite regions}

Let's now prove a more general version of this known as the \textbf{Cauchy Integral Theorem}. If $f$ is analytic in a connected region $R \in \mathbbm{C}$ with some boundary $C = \partial R$ , then 
\begin{align}
	\oint_{C=\partial R} dz\, f(z) &= 0 \ ,
\end{align}
even if $R$ is some \emph{finite} region, not just some infinitesimally small circle. From the discussion of the case where $C$ is a little circle, you may have already guessed this. Let's show this more carefully. For any such region $R$, let's break it up into boxes:
\begin{center}
\includegraphics[width=.5\textwidth]{figures/Lec_2017_12_plaquette.png}
\end{center}
The boundary of a region $\partial R$ has some orientation. By convention a positive orientation corresponds to counterclockwise. Now that we've carved up the region $R$ into a grid like Manhattan\footnote{Search for `Manhattanhenge.' It has nothing to do with this course, but this is a photogenic consequence of the Manhattan grid.}, we integrate around the boundary of one of these \emph{plaquettes}:
\begin{center}
\includegraphics[width=.4\textwidth]{figures/Lec_2017_plaq_int.png}
\end{center}
We've assumed that each plaquette has characteristic size $\varepsilon$. Note that for each side of the plaquette the orientation of $\Delta z$ is different. Let $z_0$ correspond to the center of this plaquette. Since the function is analytic, we can write the function as a Taylor function around the plaquette:
\begin{align}
	f(z) = f(z_0) + f'(z_0) (z-z_0)^2 + \cdots
\end{align}
This means that we can approximate the integral around the square as
\begin{align}
	\oint_\text{plaquette} dz\,  f(z)
	&= 
	f(z_0)
	\left(
		\Delta z_1 + \Delta z_2 + \Delta z_3 + \Delta z_4
	\right)
	+ \cdots
\end{align}
where the $\Delta z_i$ are given in the figure above. It should be clear that the sum of the $\Delta z_i$ is zero since opposite sides give equal and opposite contributions. We have shown that to leading order the integral of an analytic function $f$ around a little box around it is zero---not too surprising given our previous result for integrating around a little circle. 
\begin{exercise}
What about the next-to-leading order contribution\footnote{Life advice: ``What about the next-to-leading order contribution'' is a good question to ask whenever you have shown that the leading order contribution is zero.}? Show that the next-to-leading order contribution, which goes like $\sum_i f'(z_0)\delta z_i \Delta z_i$ vanishes. Here the $\delta z_i$ is the separation from $z_0$ to the $i^\text{th}$ side, as shown in the figure above. 
% \begin{align}
% 	\left[
% 	\left(\frac{\varepsilon}{2}\right)
% 	(i\varepsilon)
% 	+
% 	\left(\frac{i\varepsilon}{2}\right)
% 	(-\varepsilon)
% 	+
% 	\left(-\frac{\varepsilon}{2}\right)
% 	(-i\varepsilon)
% 	+
% 	\left(-\frac{i\varepsilon}{2}\right)
% 	(\varepsilon)
% 	\right] = 0 \ .
% \end{align}
\end{exercise}
The above statement is true about each plaquette. However, note that when we piece the plaquettes together, the sum of the integrals around each little boundary corresponds to the integral around the boundary of the combined region:
\begin{center}
\includegraphics[width=.4\textwidth]{figures/Lec_2017_plaqses.png}
\end{center}
One can see that neighboring boundaries have opposite orientations so that the integrals along those regions cancel. 
%
This means that the integral around a the boundary $C$ of finite region $R$ is equivalent to the sum of the integrals around the little plaquettes that tile that region:
\begin{align}
	\oint_C dz\, f(z) &= \sum_i \oint_{\text{plaquette}_i} dz\, f(z) = 0 \ .
\end{align}
Since the integral around the boundary of each plaquette is zero, the integral along $C$ is zero. This proves the Cauchy integral theorem, which may now be stated as follows:
\begin{quote}
Analytic functions are \emph{so} nice that they're boring.
\end{quote}
In other words, if a function $f$ is analytic in a connected region $R$ and you try to integrate over a closed path $C$ that is inside $R$, then the result is zero. 

\subsection{An alternative argument (optional)}
\label{sec:stokes:theorem:aside}

% \footnote{By the way, this all a generalization of Stokes' theorem: the integral of the derivative of some function $f$ over some domain $D$ is simply the function evaluated on the boundary of the domain, $\partial D$. The $\partial D$ is notation for `boundary of $D$.' The fancy way to write this is
% \begin{align}
% 	\int_D df &= \int_{\partial D} f \ .
% \end{align}
% This statement holds for $f$ as a differential $n$-form (the generalization of a one-form and related to an $n$-index tensor) and $D$ is an $n$-dimensional domain. Even if you are not familiar with this nomenclature, please note the qualitative similarities with the integral of a derivative in 1D $\int dx\, (df/dx)$---which is simply the difference of $f$ at the endpoints of the domain, the integral of a curl in 2D---which is related to the `circulation' around the boundary of the 2D domain, and the integral of a divergence in 3D---which is simply related to the flux across the enclosing surface. Vector calculus isn't hard---it's just weird when we treat the 2D and 3D cases as different from the general $n$-dimensional case. 
% }.  

For those who are mathematically inclined, here's a geometrically-motivated argument for Cauchy's integral theorem. Consider two points $z_1$ and $z_2$ in a connected region $R$ where a function $f$ is analytic. Now consider any path $C_1$ that connects $z_1$ to $z_2$ and any other path $C_2$ that connects $z_2$ to $z_1$. The orientation matters.
\begin{center}
\includegraphics[width=.5\textwidth]{figures/Lec_2017_paths.png}
\end{center}
In the region of analyticity, a function $f(z)$ has an anti-derivative $F(z)$ such that $f(z) = dF(z)/dz$, recall Exercise~\ref{eq:fundamental:theorem:calculus}. Then we know that
\begin{align}
	\int_{C_1} dz\, f(z) &= F(z_2) - F(z_1)
	&
	\int_{C_2} dz\, f(z) &= F(z_1) - F(z_2) \ .
\end{align}
We deduce that the integral of the combined path $C=C_1+C_2$ is zero. The above argument is true without having to specify what $F(z)$ is and for \emph{any} paths $C_1$ and $C_2$ that share common endpoints. We thus prove Cauchy's Integral Theorem.

\paragraph{Commentary}
This argument is highlights a general idea in differential geometry, which is the generalized Stokes' theorem. In words, one can state this as:
\begin{quote}
The integral of the derivative of some function $f$ over some domain $D$ is simply the function evaluated on the boundary of the domain, $\partial D$. The $\partial D$ is notation for `boundary of $D$.'
\end{quote}
Technically it is relevant for the domain $D$ to be sufficiently \emph{nice}; this includes the domain being connected and having a reasonable boundary. Note that the boundary $\partial D$ is always oriented---when we integrate over an interval $[a,b]\in \mathbbm{R}$, the overall sign depends on which boundary is `on top' of the integral and which is `on the bottom.' In other words, it matters if we integrate over $[a,b]$ or $[b,a]$. 

We write formally as follows:
\begin{align}
	\int_D df &= \int_{\partial D} f \ .
\end{align}
Here the domain $D$ is an $n$-dimensional space and $df$ is called a \emph{differential} $n$-\emph{form}\footnote{See \texttt{\href{https://arxiv.org/abs/2009.10356}{2009.10356}} for a recent introduction to the utility of differential forms to elucidate physics at an undergraduate level.}. From the name you should guess that it is related to the idea of a \emph{one-form} as a dual vector; see the optional discussion in Section~\ref{sec:analytic:geometric}.  This is a type of tensor with $n$ indices that is contracted with the `volume form' of the $n$-dimensional space. Formally it looks like this:
\begin{align}
	df &= (\partial_{\mu_1} f_{\mu_2\cdots \mu_n}) \; dx^{\mu_1}\wedge dx^{\mu_2}\wedge\cdots\wedge dx^{\mu_n} \ .
	\label{eq:Stokes}
\end{align}
The quantity $dx^{\mu_1}\wedge\cdots\wedge dx^{\mu_n}$ is the volume form and is an \emph{oriented} $n$-dimensional differential volume element. The funny wedge symbols ($\wedge$) represent an antisymmetric tensor product. This should not be so surprising: recall that the vector triple product $\vec{v}\cdot\left(\vec{w}\times\vec{u}\right)$ is a totally antisymmetric combination of vectors that produces the volume of the parallelogram with edges corresponding to the three vectors $\vec{v}$, $\vec{w}$, $\vec{u}$. This is precisely $\vec{v}\wedge\vec{w}\wedge\vec{u}$. The $n$-dimensional wedge product generalizes this notion to $n$-dimensional volumes. 

On the right-hand side of \eqref{eq:Stokes} is an integral over the \emph{boundary} of $D$. This is an oriented $(n-1)$-dimensional space. The integrand is the differential $(n-1)$-form. We thus see that the integral of $df$ over $D$ is related to a lower-dimensional integral of its primitive, $f$, on the boundary of $D$. 

Stokes' theorem is the underpinning of everything you've seen in vector calculus. In one dimension:
\begin{align}
	\int_a^b dx\, \frac{df(x)}{dx} = \int_{f(a)}^{f(b)} df = f(b)-f(a) \ .
\end{align}
In two dimensions, when you have a vector field $\vec{f}(x)$, the appropriate derivative (and the one that pops out of differential form notation) is the curl, $\nabla\times \vec{f}(x)$. This becomes an integral with respect to the differential surface element $d\hat{S} = \hat{\vec{n}} dS$, which may be chosen to be oriented in the $z$-direction perpendicular to the plane\footnote{Alternatively, $\hat{\vec{n}}$ is the unit normal of the 2D curved surface in a 3D space.}. Then we have the usual  Green's theorem:
\begin{align}
	\int_S dS\, \hat{\vec{n}}\cdot \left(\nabla\times \vec{f}(x)\right) 
	=
	\oint_{C=\partial S} \vec{f}(x)\cdot d\vec{x} 
	\ ,
\end{align}
where $d\vec{x}$ is a differential line element along the oriented curve $C$ that is the boundary of the integration region $S$. Finally, the divergence theorem in 3D relates the divergence of a vector field to its value on the surface:
\begin{align}
	\int_V dV \, \nabla\cdot \vec{f}(x)
	=
	\int_{S=\partial V} dS \, \hat{\vec{n}} \cdot \vec{f}(x) \ ,
\end{align}
where $\hat{\vec{n}}$ is the unit normal vector at each point on the surface $S$ bounding the volume $V$. These three versions of the `fundamental theorem of calculus' all look tantalizingly similar---and here we see that in fact, they're all manifestations of the general Stokes' theorem. 

Observe that the `boundary of a boundary' is nothing. If you have a disc, the boundary is a circle. The circle itself has no endpoints---in contrast to an interval. We can start to see some of the neat features of differential geometry when we look at Stokes' theorem, \eqref{eq:Stokes}, from this perspective. Indeed, one can read Stokes' theorem as a relation between the differential operator $d$ acting on an integrand and the `boundary operator' $\partial$  acting on the space. The `boundary of a boundary = 0' mantra can be loosely translated into `derivative of a derivative' is zero; where we are not being technically rigorous at all. You've already seen variants of this in vector calculus:
\begin{align}
	\nabla\times \nabla f(x) &= 0
	&
	\nabla \cdot \nabla\times \vec{f}(x) &=0 \ ,
\end{align}
and so forth. I used to think vector calculus was very challenging because there seemed to be so many different rules for how to differentiate and integrate. It turns out that differential geometry unifies this nicely into a general mathematical structure that, when applied to specific dimensions of space, produce all of the funny things you learn in undergraduate electrodynamics. It should not surprise you that this mathematical structure has a lot to say about the structure of theories like electrodynamics and gravity\footnote{One of my favorite introductions: \url{https://arxiv.org/abs/hep-th/0611201v1}.}.
\begin{exercise}
To see how this structure shows up in a simple physical system, look up Montgomery's treatment of the Falling Cat Problem. Similarly, Shapere and Wilczek's treatment of swimming animals at low Reynolds number. Wilczek famously won the Nobel prize for his contributions to the gauge theory of the strong nuclear force; a theory based on precisely this type of geometric structure we've discussed here. A final reference is to look up the Parallel Parking Problem, which was a topic of discussion in the first version of this P231 class that I taught in 2016. In that problem, one asks how a car can move transversely given that its only degrees of freedom are to move forward/backward and to turn the steering wheel. For the next two years students raised concerns that this class was hopelessly mathematical---so now here we are at the tail end of an exercise with no specific directions in an optional section of the lecture notes.
\end{exercise}

\subsection{Cauchy's Integral Formula}

Cauchy's Theorem tells us that integrating functions over domains where they are analytic is boring. Cauchy's Integral Formula\footnote{At this point you wonder: how is it that Cauchy got his name on everything in this business?} is the first step to something that is decidedly \emph{not} boring. The formula applies to a function $f$ that is analytic in some connected domain and a path $C$ entirely contained in that domain:
\begin{align}
	f(z_0) &= \frac{1}{2\pi i} \oint_C dz \frac{f(z)}{(z-z_0)}  \ .
	\label{eq:cauchy:integral}
\end{align}
This is the statement that the value of some function at some point $z_0$ is related to an integral of the function around the point. In fact, maybe this shouldn't be so surprising: the right-hand side looks like some kind of average of the function in the neighborhood of the function. Given the relation between analytic functions and harmonic functions, this sounds plausible. On the right-hand side there's a factor of $[2\pi (z-z_0)]^{-1}$ which indeed looks like one is averaging over the circumference around the point $z_0$. The factor of $i$ is curious. Those who are familiar with all of this will notice the `famous' combination $(2\pi i)$.

Let is highlight that the integrand on the right-hand side is 
\begin{align}
	g(z) &= \frac{f(z)}{z-z_0} \ .
	\label{eq:g:z:cauchy:integral:theorem}
\end{align}
Unlike $f(z)$, $g(z)$ is absolutely \emph{not} analytic `everywhere' in the region that we're looking at. It is not analytic at $z=z_0$. What's the derivative of $g(z)$ at the singularity? Heck if I know\footnote{Is it infinity? I suppose, but the notion of infinity in complex space can be a little tricky. At any rate, usually `infinity' is not the kind of answer that inspires much confidence.}. This is important because it is our first example of a function that is analytic in a region up to a single point\footnote{%
%
One observant student said that it is not obvious that $g(z)$ is non-analytic at the singularity $z=z_0$. If we define analytic to mean that a function is independent of $z^*$, then it is indeed not clear why $g(z)$ is non-analytic at $z_0$. This ends up being a failure of that definition: the more constructive definition of analyticity is having a unique and well-defined derivative as defined by the limiting procedure, $$\lim_{\Delta z\to 0}[g(z+\Delta z)-g(z)]/\Delta z \ .$$ 
%
}; we say that the function $g(z)$ has a \textbf{pole} at $z=z_0$.

Since the integrand $g(z)$ has a pole, we probably shouldn't integrate over it. No problem, our integration contour $C$ away from $z_0$. However, $z_0$ still punctures our domain over which the integrand is analytic\footnote{We've given our domain a topology. This, by the way, perhaps the one of the lamest things you can give someone. Once when I was a child I was sad when my parents got me a pair of socks for my birthday. I'd have been even more disappointed with topology. At least the socks kept my feet warm.}. To see what this does, let's consider the integral on the right-hand side.
\begin{center}
\includegraphics[width=.8\textwidth]{figures/Lec_2017_holes.png}
\end{center}
Let us call $C_\text{out}=C$, the original contour over which we're integrating. It is oriented counter clockwise by assumption. Because we're curious about the singularity at $z=z_0$, let's deform the contour by building a little bridge $B_1$ that heads towards $z_0$, then a little circle $-C_\text{in}$ that goes around the pole\footnote{Note the minus sign! We define $C_\text{in}$ to have positive/counter-clockwise orientation. In the picture above, we see that we traverse this little circle in the clockwise direction, so we put a minus sign on $C_\text{in}$.}, and then a little bridge $B_2$ that returns to $C_\text{out}$ where we originally left it. Clearly the integral over $B_1$ and $B_2$ cancel because
\begin{align}
	B_1 = -B_2
\end{align}
as paths. Note, however, that the region enclosed by the total curve $C_\text{out} +B_1-C_\text{in}+B_2$ is a region in which $g(x)$ is totally analytic. The $-C_\text{in}$ boundary separates the pole from the region enclosed\footnote{An engineer, a physicist, and a mathematician are tasked to optimize the amount of space surrounded by a finite length of fencing. The engineer builds a square pen since that makes it simple to construct. The physicist mumbles something about variational principles and builds a circle, stating that it optimizes the area enclosed for fixed perimeter. The mathematician takes the fence, throws away most of it, and then makes a tiny enclosure. The mathematician then carefully steps inside and says, ``I declare myself to be on the outside.''}. This means that Cauchy's Theorem holds. Because the bridge integrals over $B_1$ and $B_2$ cancel, the theorem tells us that
\begin{align}
	\oint_{C_\text{out}} dz\, g(z)
	-
	\oint_{C_\text{in}} dz\, g(z)
	= 0 \ ,
\end{align}\
where the minus sign came from $\oint_{-C_\text{in}} = - \oint_{C_\text{in}}$. The first term is precisely the right-hand side of \eqref{eq:cauchy:integral}. Apparently the second term is supposed to be $f(z_0)$. We can evaluate the second term along the contour by parameterizing $C_\text{in}$ as
\begin{align}
	C_\text{in}: \quad z(\theta) = z_0 + \varepsilon e^{i\theta} \ ,
\end{align}
which goes around $C_\text{in}$ for $\theta \in [0,2\pi]$. Now watch carefully. The relevant quantities in our integrand are:
\begin{align}
	dz &= i\varepsilon e^{i\theta} d\theta 
	&
	z-z_0 &= \varepsilon e^{i\theta} \ .
\end{align}
Now watch carefully: 
\begin{align}
	\oint_{C_\text{in}} dz\, g(z)
	=
	\oint_{C_\text{in}} dz\, \frac{f(z)}{z-z_0}
	= 
	i\int_0^{2\pi} d\theta f(z) 
	= 2\pi i f(z_0) \ . 
	\label{eq:cauchy:integral:theorem:step}
\end{align}
Note that once we wrote the integral with respect to $d\theta$, this is just an ordinary `real' integral where the integrand happens to have complex numbers in it. What is critical is that the powers of $e^{i\theta}$ canceled. Compare this to what happened in \eqref{eq:complex:theta:integral:trivial}, which was the analogous critical step for showing that the integral of analytic functions vanishes. In \eqref{eq:cauchy:integral:theorem:step} we ended up with \emph{no} factors of $e^{i\theta}$ so that the $d\theta$ integral ended up being non-zero. 
\begin{exercise}
Derive the last equality of \eqref{eq:cauchy:integral:theorem:step}:
\begin{align}
	i\int_0^{2\pi} d\theta f(z) 
	&= 2\pi i f(z_0) \ .
\end{align}
Why are we able to insert $f(z_0)$ in place of $f(z)$? Does this depend on the smallness of $\varepsilon$? (Answer: no.) {Hint}: $f(z)$ is analytic, which means it admits a Taylor expansion. Show that only the zeroth order term contributes, independently of how small $\varepsilon$ may be.
\end{exercise}

Putting this all together gives the desired result,
\begin{align}
	f(z_0) = \frac{1}{2\pi i}\oint_C dz\, \frac{f(z)}{z-z_0} \ .
	\label{eq:cauchy:integral:theorem}
\end{align}

The star of this discussion is not the analytic function $f(z)$; rather it is the analytic-up-to-a-pole function $g(z)$. Unlike our \emph{boring} scenario of functions that are analytic in a given domain, interesting stuff happens when our functions have singularities. We get to live dangerously and dance around theses singularities. In general, we will refer to singularities that go like $(z-z_0)^{-n}$ to be poles at $z=z_0$. The positive integer $n$ is called the order of the pole. The case $n=1$ is called a \textbf{simple pole}.

\subsection{From Taylor to Laurent}

Functions with poles are clearly not analytic \emph{everywhere}. However, they're pretty close to being analytic. They're analytic except for isolated poles. A function that is analytic up to poles is called \textbf{meromorphic}. If you're like me, you should classify these as \emph{nice, but not too nice} functions. They're just not-nice enough to be interesting. If you're keeping up with the `big picture,' you'll recall that the Fourier transform of a differential operator's Green's function  \eqref{eq:Greens:function:Fourier:transform:heuristic} appears to be in this class.

In a region where $f$ is analytic, differentiability meant that one could write a Taylor expansion: a series of terms that go like $(z-z_0)^n$ for positive integers $n$. When $f$ is merely \emph{meromorphic}, the Taylor expansion is generalized to a \textbf{Laurent expansion}:
\begin{align}
	f(z) = \sum_{n=-N}^\infty  a_n(z_0) (z-z_0)^n \ ,
\end{align}
where $N$ is the order of the pole at $z_0$ (if there is one). 

\subsection{The Residue Theorem: a first look}

Now we arrive at our mail tool. Suppose $f(z)$ is meromorphic in some region of the complex plane. In fact, suppose $f(z)$ has a simple pole at $z_0$; the function $g(z)$ in \eqref{eq:g:z:cauchy:integral:theorem} is a function precisely of this type. Consider the integral of $f(z)$ around a closed contour that goes around the pole once. For example, the pole is in some connected region $R$ and the contour is the boundary of this region, $C=\partial R$. Applying the Laurent expansion about $z_0$ for this meromorphic function $f$ gives
\begin{align}
	\oint_C dz\, f(z) &= \oint_C dz 
	\left[
	\sum_{n<0} a_n (z-z_0)^n + \sum_{n\geq 0} a_n (z-z_0)^n
	\right] \ .
\end{align}
All we have done is separated the positive-power terms of the Laurent expansion from the negative power terms. We know that the positive power terms integrate to zero because those terms are analytic in $R$. Thanks, Cauchy's Theorem.

What about the term with negative powers? Since we assumed that $z_0$ is a simple pole, we know that only the $a_{-1}$ term is non-zero in this Laurent expansion about $z_0$. That means that we can write the integral as
\begin{align}
	\oint_C dz\, f(z) &= 
	\oint_C dz \,
	\frac{a_{-1}}{z-z_0} 
	\ .
	\label{eq:residue:int:step:1}
\end{align}
The coefficient $a_{-1}$ of the Laurent expansion about a pole is called the \textbf{residue} of the function $f$ at the pole $z_0$. We will use the notation $\text{Res}_f(z_0)$. 


Now recall Cauchy's Integral Theorem. For the sake of clarity (I have used $f$ too many times), let us write the integral theorem with respect to an analytic-in-this-neighborhood function $h$:
\begin{align}
	\oint_C \frac{h(z)}{z-z_0} = {2\pi i} \, h(z_0) \ ,
\end{align}
we've gone ahead and re-arranged some terms. Comparing this to \eqref{eq:residue:int:step:1}, we may take $h(z) = a_{-1} = \text{Res}_f(z_0)$. This means that $h(z_0) = \text{Res}_f(z_0)$ so that we ultimately have
\begin{align}
	\oint_C dz\, f(z) &=  2\pi i  \, \text{Res}_f(z_0) \ .
\end{align}
We will generalize this shortly, but the main idea is in this simple example. If you can identify the $a_{-1}$ coefficient of a meromorphic function at its pole, then you can easily integrate the function around the pole. As long as there aren't any other poles in the neighborhood, it doesn't matter what your contour is as long as you are going around counter-clockwise and you go around exactly once.


% Lec 13

\subsection{The Residue Theorem: more carefully}

The residue theorem is our primary tool for calculating integrals on the complex plane. Let's see how it generalizes. Consider a meromorphic function $g(z)$ with two simple poles:
\begin{align}
	g(z) &= \frac{h(z)}{(z-z_1)(z-z_2)} 
	&
	h(z) \text{ analytic over }\mathbbm{C} \ .
\end{align}
The simple poles are located at $z=z_1$ and one at $z=z_2$. Consider a contour $C$ that encloses both poles. The integral of $g(z)$ along $C$ is simply $2\pi i$ times the sum of the residues of the two simple poles. This is easy to see from dividing $C$ into the sum of two paths:
\begin{center}
\includegraphics[width=.5\textwidth]{figures/Lec_2017_13_2poles.png}
\end{center}
Here $C_1$ is a closed contour that encloses $z_1$ but not $z_2$. Similarly, $C_2$ is a closed contour that encloses $z_2$ but not $z_1$. Observe that $C_1$ and $C_2$ overlap along a line that separates the two poles. Because both $C_1$ and $C_2$ are positively oriented, the integrals along this line cancel. From this it is clear that
\begin{align}
	\oint_C dz\, g(z) &=
	\oint_{C_1} dz\, g(z) 
	+
	\oint_{C_2} dz\, g(z)
	= 2\pi i\left(\text{Res}_f(z_1) + \text{Res}_f(z_2)\right) \ .
\end{align}
This gives a more general form of the residue theorem. The integral of a meromorphic function $f(z)$ around a closed contour $C$ that is the boundary of some region of the complex plane\footnote{Note that this means that $C$ is positively oriented and only circles around the region once.} is
\begin{align}
	\oint_C dz\, f(z) &= 2\pi i \sum_{i\in \text{poles}} \text{Res}_f(z_i) \ ,
\end{align}
where the sum is over the poles of $f$ at $z_i$ enclosed inside $C$.
\begin{exercise}
How would the residue theorem change if the contour $C$ were oriented in the opposite direction? What if the contour circled the poles multiple times? What if the contour circled some poles some number of times, and other poles a different number of times?
\end{exercise}


\subsection{Non-simple poles (a case study in being careful)}

So far we've focused only on meromorphic functions that have \emph{simple} poles. What about higher order singularities? Here's a tangible example:
\begin{align}
	\oint_C dz \, f(z) = ?
	&&
	f(z) &= \frac{1}{(z-2i)^2} \ ,
	\label{eq:non-simple:pole:example}
\end{align}
where $C$ is a positively oriented contour that circles the second-order pole $z_0 = 2i$. Because $f(z)$ has no simple pole at $z_0$, it looks like there's no contribution to the integral. To check this, we recall Cauchy's integral formula, \eqref{eq:cauchy:integral:theorem:step}. The reason why simple poles contributed to the contour integral is the observation that for an integer,  $n$, the integral of $e^{in\theta}d\theta$ behaves as follows: 
\begin{align}
	\int_0^{2\pi}d\theta\, e^{in\theta} 
	&=
	\begin{cases}
	0 & \text{\quad if } n\neq 0
	\\
	2\pi  & \text{\quad if } n= 0
	\end{cases} \ .
\end{align}
Following the same in the previous sub-sections, the integral in \eqref{eq:non-simple:pole:example} about a contour $C$ that circles $z_0=2i$ once is equivalent to the integral over smaller contour $C'$ that is a small circle that surrounds $z_0$. Parameterize $C'$ by the angular variable:
\begin{align}
	z(\theta) &= z_0 + \varepsilon e^{i\theta} & dz &= i\varepsilon e^{i\theta} \, d\theta \ .
\end{align}
Then the integral is:
\begin{align}
	\oint_C dz\, f(z) 
	= 
	\oint_{C'} dz\, f(z) 
	= 
	\int_0^{2\pi} i\varepsilon e^{i\theta} d\theta\, 
	\frac{1}{\varepsilon^2 e^{2i\theta}} 
	=
	\frac{i}{\varepsilon}
	\int_0^{2\pi} d\theta\, 
	e^{-i\theta}
	= 0 \ .
	\label{eq:non-simple:pole:eg:zero}
\end{align}
So this is all consistent with the mantra of \emph{find the $a_{-1}$ coefficient of the Laurent expansion, that's the residue}. 

We should be careful, though. If we're too slick we can convince ourselves of wrong things. For example, suppose we wanted to generalize \eqref{eq:non-simple:pole:example} by changing the numerator of $f(x)$:
\begin{align}
	\oint_C dz \, f(z) = ?
	&&
	f(z) &= \frac{h(z)}{(z-2i)^2} \ ,
	\label{eq:non-simple:pole:example:hz}
\end{align}
where $h(z)$ is an analytic function. For the sake of argument, let's assume that $h(z)/(z-2i)^2$ has been simplified so that there are no common factors between the numerator and denominator\footnote{For example, if $h(z)=(z-2i)$ then clearly $f(z)$ has a simple pole at $z_0=2i$ so the integral picks up a non-zero residue.}. One might think that when we shrink the contour from $C$ to $C'$, we can approximate $h(z)= h(z_0)$ so that
\begin{align}
	\oint_{C'}dz\, f(z)
	&=
	\oint_{C'} i\varepsilon e^{i\theta}d\theta\, 
	\frac{h(z_0)}{\left(z-2i\right)^2}
	=
	\frac{i}{\varepsilon}h(z_0)
	\int_0^{2\pi} d\theta\, 
	e^{-i\theta}
	= 0\,?
\end{align}
In general, this argument is \emph{wrong}. It is wrong even though the \eqref{eq:non-simple:pole:eg:zero} happens to be true. 
\begin{exercise}
Can you spot the incorrect assumption?
\end{exercise}
The incorrect assumption is that we could approximate $h(z) = h(z_0)$ because the little circle $C'$ is always close to $z_0$. The intuition is fine, but we weren't careful enough: even though $h(z)$ is analytic everywhere, we should remember write a Taylor expansion:
\begin{align}
	h(z) = h(z_0) + h'(z_0)(z-z_0) + \cdots
\end{align}
Even if our intent is to only keep the first term, it's a good habit to remember that the Taylor expansion contains many terms. Let's see what happens: 
\begin{align}
	\oint_{C'}dz\, f(z)
	&=
	\oint_{C'} i\varepsilon e^{i\theta}d\theta\, 
	\frac{h(z_0)+ h'(z_0)(z-z_0) + \cdots}{\left(z-2i\right)^2}
	\\
	&
	=
	\frac{ih(z_0)}{\varepsilon}
	\int_0^{2\pi} d\theta\, 
	e^{-i\theta}
	+
	{ih'(z_0)}
	\int_0^{2\pi} d\theta
	+\cdots
	\label{eq:non-simple:pole:example:hz:punchline}
\end{align}
While the first term vanishes, the second term is clearly non-zero since it's a contour integral. Indeed, one ends up with
\begin{align}
	\oint_{C'}dz\, \frac{h(z)}{(z-2i)^2}
	= 2\pi i h'(z_0) \ ,
\end{align}
where evidently $\text{Res}_f(2i)=h'(2i)$ is the residue of $f(z)=h(z)/(z-2i)^2$ at $z_0=2i$. 
\begin{exercise}
Why don't higher order terms in \eqref{eq:non-simple:pole:example:hz:punchline} do not contribute to the contour integral.
\end{exercise}
The purpose of this example was to show that it can be a little tricky to identify the residue of a function by simply `looking at the denominator.' You will derive an explicit formula in your homework. Fortunately, we will rarely consider non-simple poles in this course. 

% Lec 14 2017

Let's go through a few examples. These are from the third edition of Boas, chapter 14.6. 

\begin{example}
Let $f(z)=\cot z$. Find the residue of $f$ at $z=0$. 

To solve this, we can write out the numerator and denominator of the cotangent:
\begin{align}
	\cot z &= \frac{\cos z}{\sin z}
	= \frac{1-z^2/2 + \cdots}{z - z^3/3!} \ .
\end{align}
From this we see that $z\to 0$ is indeed a simple pole, and one may write the residue as
\begin{align}
	\text{Res}_f(0) &= \lim_{z\to 0} z \frac{1}{z} = 1 \ .
\end{align}
\end{example}


\begin{example}\label{ex:cot:2:residue}
Let $f(z)=\cot^2 z$. Find the residue of $f$ at $z=0$. 

Writing out the numerator and denominator again:
\begin{align}
	\cot^2 z &= 
	= \frac{1-z^2 + \cdots}{z^2 - 2z^3/3! + \cdots} 
	\sim \frac{1}{z^2} + \mathcal O(1)
	\ .
\end{align}
From this we deduce that $\text{Res}_f(0) = 0$. 
\end{example}

\begin{exercise}
Why were we able to be slick in Example~\eqref{ex:cot:2:residue} after making a big deal about being careful in \eqref{eq:non-simple:pole:example:hz:punchline}?
\end{exercise}

\begin{exercise}
Let $f(z)=z\cot^2 z$. Find the residue of $f$ at $z=0$. 
\end{exercise}


\begin{exercise}
Let $f(z)$ be
\begin{align}
	f(z) &= \left(\sum_{n=0}^\infty c_n z^n\right)\cot^2 z
\end{align}
Show that the residue of $f$ at $z=0$ is $c_1$.
\end{exercise}


\subsection{The Killer App: Real Integrals}

The \emph{killer app}\footnote{This is an old phrase from the first dot-com bubble.} for complex contour integrals is to solve integrals along the \emph{real line}. It is often the case that some of the real functions that we would like to integrate happen to have poles in the complex plane. In this case, the integral along the real line can be completed into a closed contour $C$ in the complex plane. If we can separate the contribution of the real line from the rest of the contour, then we can use the residue theorem to do the integral without any of the hard work of, uh, integrating.

\begin{example}
Consider the real function
\begin{align}
	f(x) = \frac{1}{x^2 +1} \ .
\end{align}
Typically we care about real functions with real arguments. However, let's \emph{analytically continue} $f(x)$ into a meromorphic function $f(z)$
\begin{align}
	f(z) = \frac{1}{z^2+1} \ .
\end{align}
This is the (unique\footnote{The proof is left to you to derive or look up.}) complex function that is analytic and agrees with the original function on the real line. We immediately notice that $z^2+1 = (z+i)(z-i)$ so that $f(z)$ has simple poles at $z=\pm i$. Consider the following contour which includes the real interval $x\in [-R,R]$ as part of it:
\begin{center}
\includegraphics[width=.7\textwidth]{figures/Lec_2017_14_contour.png}
\end{center}
We will care about the limit $R\to \infty$ where the integral includes the entire real line.
Observe that the contour only encloses the pole at $z=i$. Let us integrate the function around $C$. From the residue theorem, we have
\begin{align}
	\oint_Cdz\, \frac{1}{(z+i)(z-i)} 
	= 2\pi i \text{Res}_f(i) 
	= \pi \ ,
\end{align}
where we've used $\text{Res}_f(i)=1/2i$. However, we can also write the integral as a sum of the real integral plus a counter-clockwise arc of radius $R$:
\begin{align}
	\oint_C dz\, f(z) = \int_{-R}^R dx\, f(x) 
	+ \int_0^\pi iR e^{i\theta} d\theta  \, f\left(Re^{i\theta}\right) \ .
\end{align}
where in the second term we've used the parameterization $z(\theta) = Re^{i\theta}$. Note that the first term on the right-hand side is the `ordinary' purely real integral. The second term behaves as follows:
\begin{align}
	\int_0^\pi iR e^{i\theta} d\theta  \, f\left(Re^{i\theta}\right)
	&= 
	\frac{Re^{i\theta}}{R^2 e^{2i\theta}+1} d\theta \ .
\end{align}
As we take the limit $R\to\infty$, the integral goes like $\sim 1/R \to 0$. Thus as long as we actually care about the integral along the entire real line, $R\to \infty$, we have
\begin{align}
	\oint_Cdz\, \frac{1}{(z+i)(z-i)}  
	= \pi 
	= \lim_{R\to\infty} \int_{-R}^R \frac{dx}{x^2+1} \ .
\end{align}
And so we find
\begin{align}
	\int_{-\infty}^\infty dx \, 
	\frac{dx}{x^2+1}
	= \pi \ .
\end{align}
This is an absolutely correct result about a \emph{real} integral that we sneakily derived without actually doing the integral. I assume one may check this by the appropriate trigonometric substitution\footnote{Just kidding, you can just check it on \emph{Mathematica}. You can even tell people that you did the trig substitution if it makes you feel better.} Observe that the critical step was that the curved part of the contour (the one that actually had imaginary parts) went to zero in a well defined limit. 
\end{example}
\begin{exercise}
What result do you get when you solve for the same integral of function $f(x)=(x^2+1)^{-1}$ over the real line, but this time you use the contour $C'$ that encircles the lower half of the complex plane?
\end{exercise}
Let's address the question of \textbf{analytic continuation}\footnote{See Appel section 5.3 for a more careful discussion.}. We started with a purely real function $f(x)$ and `generalized it' to a complex function $f(z)$ that happened to have poles in the complex plane. Because we used the residue theorem, the details of these poles (residue and whether they're enclosed by $C$) are critical for performing the integral. So a good question is: is the \emph{complex function} $f(z)$ even uniquely defined given the \emph{real} function $f(x)$?\footnote{My notation is a little glib and assumes the answer. If you want, you can write $f_\mathbbm{C}$ and $f_\mathbbm{R}$ to differentiate the two functions that are different in principle.}
\begin{center}
\includegraphics[width=.9\textwidth]{figures/Lec_2017_14_analytic_continuation.png}
\end{center}
There is a handy theorem that states that when two analytic functions agree on their domain of overlap, then they agree on their common domain. For \emph{meromorphic} functions, we can just imagine the domain where those singularities are not included. In our example above, $f(x)$ is analytic along the real line and $f(z)$ is a function that trivially (by construction) is analytic and matches $f(x)$ on the real line. Then the theorem says that $f(z)$ and $f(x)$ `agree' in the entire domain of analyticity; so one can \emph{analytically continue} $f(x)$ to the complex plane as long as one avoids any singularities.

The gist of the proof is that if two functions $f$ and  $g$ agree in some overlapping domain of analyticity, then their difference $h(z)\equiv f(z)-g(z)$ is also an analytic function in this domain. Analyticity means one has finite radius of convergence because it has a well defined Taylor expansion. One can then use this to argue that $h(z)$ can be defined beyond the overlapping domain. Continuity requires that $h(z)$ remains zero even outside the overlapping domain, which ensures that $f(z)=g(z)$. In other words, there is a unique analytic continuation of a function to a maximal domain of analyticity.

There's another effect you may wonder about. Are there some `edge' effects from the corner of the contour?
\begin{center}
\includegraphics[width=.9\textwidth]{figures/Lec_2017_corner.png}
\end{center}
A hand-waving answer is to say that we take the $R\to\infty$ limit `first' and so that corner is pushed off into infinity. This should be totally unsatisfying since mathematical physics \emph{rarely} depends on the order in which limits are taken. 

A better answer is that the complex plane \emph{with infinity} is a slightly different object from the complex plane without infinity. One way to see this is to note that the complex plane can be mapped one-to-one onto a sphere. This is called \emph{stereographic projection}. The picture is this:
\begin{center}
\includegraphics[width=.9\textwidth]{figures/Lec_2017_stereographic.png}
\end{center}
The idea is to place a unit sphere at the origin of $\mathbbm{C}$. The south pole of the sphere is touching $z=0$. Then for any point $z\in\mathbbm{C}$, one can draw a line from $z$ to the north pole of the sphere. There is a single unique point on the sphere that intersects that line. This maps the entire complex plane onto the sphere---except for the north pole of the sphere. We identify the north pole with complex infinity. That's right: all infinities are the same: $\infty$, $i\infty$, $\infty e^{2\pi/3}$, $-\infty$, etc. From this point of view, the real line maps onto the prime meridian of the sphere and there there are no `corners' in the contour. In fact, there's no arc component, either! The integral along the real line is a closed contour with respect to the stereographic projection. Note that the orientation of the contour depends on which hemisphere you are figuratively standing in.

\subsection{The Main Example: follow this carefully}
Let's try another instructive example. Here's an real integral that we would like to solve using contour integral techniques:
\begin{align}
\int dx\, f(x) = \int_{-\infty}^\infty  dx\,
\frac{2\cos x}{x^2+1} \ .
\end{align}
We can analytically continue this into a complex function by simply replacing the real variable with a complex variable, $x\to z$. This complex function $f(z)$ is analytic on the real line. The presence of simple poles at $z=\pm i$ mean that it is meromorphic with respect to the whole complex plane. It is convenient---for reasons that will be clear momentarily---to write the cosine as a sum of exponentials:
\begin{align}
	f(z) 
	= \frac{e^{iz}+e^{-iz}}{(z+i)(z-i)} 
	= \frac{e^{iz}}{(z+i)(z-i)} + \frac{e^{-iz}}{(z+i)(z-i)} 
	\equiv f_+(z) + f_-(z)
	\ .
\end{align}
With some foresight, we have separated $f(z)=f_+(z)+f_-(z)$ into two different functions. 
%
There are now two obvious choices for convenient integration contours that include the real line with the appropriate orientation:
\begin{center}
\includegraphics[width=.5\textwidth]{figures/Lec_2017_14_whichcontour.png}
\end{center}
We label the two contours are $C_1$ and $\bar C_2$. The bar over $\bar C_2$ is to remind us of the clockwise orientation; alternatively one could\footnote{This all boils down to notation. You are free to be creative with notation, but the underlying quest is \emph{clarity}. Notation doesn't change the underlying mathematics or physics, but a good notation make your ideas clearer---perhaps even to yourself.} have written $\bar C_2 = -C_2$.  Does it matter which contour we pick?

Our criteria for a convenient contour are that (1) the contour includes the real line with the appropriate orientation and (2) the rest of the contour integrates to zero. That way we can use the residue theorem:
\begin{align}
	\int_C f(z) dz 
	= \int_{-\infty}^\infty dx\, f(x)
	+ \int_{\text{arc}}dz\, f(z) = 
	\int_{-\infty}^\infty dx\, f(x)
	= \sum_{i=\text{poles}}\text{Res}_f(z_i) \ ,
\end{align}
where we recall that $i$ is runs over the poles enclosed by the contour $C$. The condition that the integral over the arc vanishes determines which contour we take. We may conveniently parameterize the arcs as follows:
\begin{align}
	z(\theta) &= R\cos\theta + i R\sin\theta
	\label{eq:z:paramterization:large:arc}
\end{align}
where for $C_1$ the arc is given by $\theta \in [0,\pi]$ while for $\bar C_2$ the arc is $\theta\in[0,-\pi]$. We assume the limit $R\to \infty$. The arc integrals are then:
\begin{align}
	\int_{\text{arc}}
	dz \, 
	%  \left[f_+(z) + f_-(z)\right]
	f_\pm(z)
	=
	\int_\text{arc}
	d\theta \, 
	\frac{e^{\pm iz}}{z^2+1}
	% + 
	% \int_\text{arc}
	% d\theta \, 
	% \frac{e^{iz}}{z^2+1} 
	\ .
\end{align}
When we plug in the parameterization \eqref{eq:z:paramterization:large:arc}, we see that the denominator of $f_\pm$ scales like $R^2$. There's an additional factor of $R$ in the numerator coming from $dz$, and so it may  appear that \emph{either} $C_1$ or $\bar C_2$ could be used. However, recall that $f_{\pm}(z)\sim e^{\pm iz}$ and exponentials beat polynomials. Plugging in $z(\theta)$ into the exponentials gives
\begin{align}
	e^{\pm iz} &= e^{\pm i\left(R\cos\theta + i R\sin\theta\right)} \ ,
\end{align}
where the difference between $C_1$ and $\bar C_2$ is the range (and direction) of $\theta$.
\begin{itemize}
	\item $C_1$ lives in the upper half plane so the imaginary part of $z$ is positive: $R\sin\theta > 0$ \ .
	\item $C_2$ lives in the lower half plane so the imaginary part of $z$ is negative: $R\sin\theta < 0$\ .	
\end{itemize}
This makes it easy to read off the convergence properties of the exponential
\begin{align}
	e^{\pm iz} &= (\text{oscillating})e^{\mp R\sin\theta} \ .
\end{align}
The oscillating part is not relevant: since $|e^{i\varphi}| = 1$ , this factor doesn't actually affect the magnitude of the contribution of the integrand, just the phase. The $\exp(\mp R\sin\theta)$ factor, on the other hand, is everything. 
\begin{itemize}
	\item $e^{- R\sin\theta}$ is exponentially suppressed when $\sin\theta > 0$, this corresponds to the upper half plane and so we use the $C_1$ arc to complete the integration contour of the $f_+(z)\sim \exp(-R\sin\theta)$ integrand.
	\item $e^{+ R\sin\theta}$ is exponentially suppressed when $\sin\theta < 0$, this corresponds to the lower half plane and so we use the $\bar C_2$ arc to complete the integration contour of the $f_-(z)\sim \exp(+R\sin\theta)$ integrand.
\end{itemize}
Because the $e^{-R}$ exponential suppression defeats the polynomial $R^{-1}$ scaling, the vanishing of the integral along the arc is determined only by this exponential factor: $\lim_{R\to\infty}e^{-R}/R\to 0$. We thus have the following approach to the original real integral:
\begin{align}
	\int f(x) &= \int_{-\infty}^\infty  dx\,
\frac{2\cos x}{x^2+1} 
	= 
	\int_{-\infty}^\infty  dx\,
	\frac{e^{+ix}}{x^2+1} 
	+
	\int_{-\infty}^\infty  dx\,
	\frac{e^{-ix}}{x^2+1} 
	\\ 
	&=
	\int_{C_1} dz\, 
	\frac{e^{+iz}}{z^2+1} 
	+
	\int_{\bar C_2} dz\, 
	\frac{e^{-iz}}{z^2+1} 
	\\
	&=
	2\pi i \text{Res}_{f_+}(i)
	-
	2\pi i \text{Res}_{f_-}(-i)
\ .
\end{align}
In the second step we have used the fact that the integral along the real line can be extended by large arcs along $C_1$ or $\bar C_2$ as long as the integral along those arcs vanishes. The residues are straightforward: 
\begin{align}
	\text{Res}_{f_\pm}(\pm i) = \pm \frac{1}{2ie}
\end{align}
so that the final result is
\begin{align}
	\int_{-\infty}^\infty dx\, f(x) &= \frac{2\pi}{e} \ .
\end{align}


 \subsection{A fancy example}

 Here's an example of the residue theorem used to calculate an integral along just the positive real line,
 \begin{align}
 	\int_0^\infty dx\, x^{1/3} F(x) \ .
 \end{align}
 We can trivially analytically continue the integrand by replacing $x\to z$. Let us assume that $F(x)$ is a meromorphic function with poles away from the positive real line. Let us further assume further that $F(z)$ is exponentially suppressed for $|z| = R \gg 1$, say $F(z)\sim e^{-R}$. Note that the $x^{1/3} \to z^{1/3}$ factor is problematic. It tells us that there's an \emph{branch cut} in our integrand: if we start at some point and went a full circle around the origin, say $z(\theta) = e^{i\theta}$ for $\theta\in[0,2\pi]$, then $z^{1/3}$ doesn't return to the origin. This means you probably need to extend the complex plane into Riemann sheets. Practically what this means is that we have to pick a branch cut that prevents us from taking contours that are `full circles' around the origin. We have the freedom to pick the orientation of this branch cut, but in this case it is convenient to pick it along the positive real line\footnote{If you want to be careful to make sure that you're not interfering with the original integral, you can put the branch cut just below the real line by some vanishingly small amount and take that amount to zero at the end of the calculation.}. Let's take a peculiar contour that includes the integration region we want, $x\in [0,\infty)$, then circles the entire complex plane, and returns back to zero along $x\in (\infty, 0]$. 
 \begin{center}
 \includegraphics[width=.8\textwidth]{figures/Lec_2017_14_branch.png}
 \end{center}
 In the past, we'd have said that the integral along $x\in [0, \infty)$ should cancel that from $x\in (\infty, 0]$. However, we now notice that there's a branch cut separating those two regions. Indeed, you may already see that the \emph{integrand} is not identical along those two lines. 

Construct the following contour integral:
 \begin{align}
 	\oint_C  dz\, z^{1/3} F(z) &= 
 	\int_0^\infty dx x^{1/3}F(x)
 	+ \int_{\text{arc}} dz \, z^1/3 F(z)
 	+ \int_\infty^0 dx \left(x e^{2\pi i}\right)^{1/3} F\left(e^{2\pi i}\right) \ .
 \end{align}
 The left-hand side is simple to evaluate with the residue theorem and will depend on the poles of the unspecified function $F(z)$. On the right-hand side, the integral along the arc vanishes by the assumptions we've made about $F(z)$ becoming exponentially small for large arguments, $|z|=R\gg 1$.  Finally, we notice that the third term on the right-hand side we've written $z=xe^{2\pi i}$; this is critical since it includes the phase that $z$ picks up when traversing the large circular path. Since $F(z)$ is meromorphic, we may simply replace $F(x\exp(2\pi i)) \to F(x)$. However, the $z^{1/3}$ needs some care. We find:
 \begin{align}
 	2\pi i \sum_{j=\text{poles}} z_j^{1/3} \text{Res}_F (z_j)
 	=
 	\left(1-e^{2\pi i/3}\right) \int_0^\infty dx\,  x^{1/3} F(x) \ .
 \end{align}
The $-e^{2\pi i/3}$ came from the $1/3$-power of the phase. The minus sign comes from $\int_{\infty}^0 dx = -\int_0^\infty dx$. Converting the difference of exponentials into a sine gives
\begin{align}
	\left(1-e^{2\pi i/3}\right) 
	= 2e^{i\pi/3} \cdot \frac{1}{2}\left(e^{-i\pi/3}-e^{i\pi/3}\right)
	= -2 e^{i\pi/3} \sin(\pi/3) \ .
\end{align}
We thus find an expression for the original integral:
\begin{align}
	\int_0^\infty dx\, x^{1/3}F(x) &=
	\sum_{j=\text{poles}}
	\frac{\pi i e^{i\pi/3} z_j^{1/3}\text{Res}_F(z_j)}{\sin(\pi/3)}
	\ .
\end{align}
We won't worry too much about branch cuts in this course. However, they do occasionally show up in physical manifestations, such as in dispersion relations\footnote{See, e.g., \url{https://arxiv.org/abs/1610.06090} for a peek at how it can show up in quantum field theory.}.





%!TEX root = P231_notes.tex

\section{The Harmonic Oscillator}
% \lecdate{lec~12}

Let's return to the problem of solving for Green's functions. Let's focus on our favorite example---arguably, the \emph{only} example\footnote{Upon generalizing to higher dimensions, curvilinear coordinates. Physicists have Harmonic Oscillators in different area codes.}---is the harmonic oscillator. The differential operator is
\begin{align}
	\mathcal O = \left(\frac{d}{dt}\right)^2 + \omega_0^2 \ .
\end{align}
The Green's function equation tells us the response $G(t,t_0)$ at time $t$ from a `unit displacement' at $t_0$:
\begin{align}
	G''(t,t_0) + \omega_0^2 G(t,t_0) = \delta(t-t_0) \ .
	\label{eq:HO:Greens:eqn}
\end{align}
Recall that the arguments $t$ and $t_0$ are analogous to the indices of a finite-dimensional matrix. For notational convenience, we will set $t_0=0$ and not list it explicitly. We are primarily concerned about the $t$-dependence of $G(t,t_0)$. 

\subsection{Fourier Transform}

The first thing we're going to do is write $G(t,t_0)$ as a Fourier transform with respect to $t$. Please refer to Appendix~\ref{app:Fourier} for our set of Fourier transform conventions. We can write $G(t)$ as an integral over Fourier modes with frequency $\omega$ and weight (Fourier transform) $\tilde G(\omega)$:
\begin{align}
	G(t) &= \int_{-\infty}^\infty\dbar \omega \, e^{-i\omega t} \tilde G(\omega) 
	&
	\dbar = \frac{d}{2\pi}
	\ .
	\label{eq:HO:Greens:Fourier}
\end{align}
We say that $\tilde G(\omega)$ is the Fourier transform of $G(t)$. The key point is that the $t$-dependence of $G(t)$ has been sequestered into the $e^{-i\omega t}$ plane waves. This is convenient since these plane waves are eigenfunctions of the derivative operator:
\begin{align}
	\frac{d}{dt} e^{-i\omega t} &= -i\omega e^{-i\omega t} \ .
\end{align}
The left-hand side of the Green's function equation \eqref{eq:HO:Greens:eqn} is
\begin{align}
	\mathcal O_t G(t,t_0) 
	&= 
	-
	\int_{-\infty}^\infty \dbar \omega \, 
	\left(\omega^2-\omega_0^2\right) e^{-i\omega t} \tilde G(\omega,t_0) \ .
\end{align}
The right-hand side is simply the Fourier transform of $\delta(t-t_0)$:
\begin{align}
	\delta(t-t_0)
	&=
	\int_{-\infty}^\infty \dbar \omega \, e^{-i\omega (t-t_0)} \ .
	\label{eq:delta:fourier}
\end{align}
\begin{exercise}
Use our conventions for the Fourier transform \eqref{eq:HO:Greens:Fourier} (see also Appendix~\ref{app:Fourier}) to confirm the Fourier representation of $\delta(t-t_0)$ in \eqref{eq:delta:fourier}. In our notation, the Fourier coefficients $\tilde f(\omega)$ of a function $f(t)$ is
\begin{align}
	\tilde f(\omega) &= 
	% \frac{1}{2\pi}
	\int_{-\infty}^\infty d t\, e^{i\omega t} f(t) \ .
\end{align}
\end{exercise}
So the Green's function equation for the 1D harmonic oscillator, \eqref{eq:HO:Greens:eqn}, tells us
\begin{align}
	-
	\int_{-\infty}^\infty \dbar \omega \, 
	\left(\omega^2-\omega_0^2\right) e^{-i\omega t} \tilde G(\omega)
	&=
	\int_{-\infty}^\infty \dbar \omega \, e^{i\omega t}
	\ .
	\label{eq:G:HO:Fourier:equation:integrals}
\end{align}
For simplicity we have set $t_0=0$ and don't write it explicitly. There's a rather unscrupulous\footnote{I don't think this is rigorously valid, but the result is true. The ends don't justify the means, but let's take this morally ambiguous shortcut to make the big picture clear. I encourage you to live the rest of your lives with virtue.} way to solve this equation for $\tilde G(\omega)$. Since the two sides of this expression are equal, they have the same Fourier expansion. This implies that the Fourier coefficients are equal. Since both sides are already written as Fourier expansions, we can just match the coefficients of the basis functions, $e^{-i\omega t}$. This gives us:
\begin{align}
	\tilde G(\omega) &= \frac{-1}{\omega^2-\omega_0^2}
	\label{eq:G:HO:Fourier:term}
\end{align}
\begin{exercise}
Prove \eqref{eq:G:HO:Fourier:term} honestly. {Hint}: Start with \eqref{eq:G:HO:Fourier:equation:integrals} and project out the Fourier coefficients. Recall that you do this by taking the inner product with one of the basis functions and then using the orthogonality of the eigenbasis. You may need to be careful with the normalization.
\end{exercise}
\begin{exercise}
In \eqref{eq:G:HO:Fourier:equation:integrals} we had already set $t_0 = 0$. What is the expression for $\tilde G(\omega)$ if we kept $t_0$ explicit?
\end{exercise}
That was the critical step: we have successfully solved for the Green's function Fourier coefficient. This means that we have a closed form expression for the Green's function by plugging $\tilde G(\omega)$ into \eqref{eq:HO:Greens:Fourier}:
\begin{align}
	G(t) &=  \int_{-\infty}^\infty \dbar \omega
	\, 
	\frac{-e^{-i\omega t}}{\omega^2-\omega_0^2} \ .
	\label{eq:G:HO:Fourier:Rep:t0:0}
\end{align}
All that's left is for us to actually \emph{do} this integral. Fortunately, this integral should look very similar. It seems to beg for us to solve using the residue theorem. 
\begin{exercise}
What are the poles of the integrand in \eqref{eq:G:HO:Fourier:Rep:t0:0}? What are their associated residues? What is the residue if $t_0\neq 0$?
\end{exercise}

\subsection{Contour Integral}

The integral \eqref{eq:G:HO:Fourier:Rep:t0:0} looks like it's perfect for contour integration. There's an exponential factor on top that will determine the convergence, and the denominator can be factored to see where the poles are. Except we notice something troubling:
\begin{align}
	\frac{-e^{-i\omega t}}{\omega^2-\omega_0^2} 
	&=
	\frac{-e^{-i\omega t}}{(\omega - \omega_0)(\omega + \omega_0)} \ .
\end{align}
The poles are located at $\omega = \pm \omega_0$. These are \emph{on the real axis}, precisely along the integration contour! How annoying!

Now you may want to have an existential moment. Remind yourself that all we're doing is solving for the behavior of the one-dimensional \emph{harmonic oscillator}. This is an eminently \emph{physical} system. We could have written this system as $\mathcal O f(t) = s(t)$ where $f(t)$ is the displacement of a harmonic oscillator and $s(t)$ is some driving function. The Green's function, $G(t,t_0)$ gives the response of the system to a `unit' driving function, $s(t) = \delta(t-t_0)$. The response of the system should be perfectly physical. And yet---\emph{and yet}---we now face an integral \eqref{eq:G:HO:Fourier:Rep:t0:0} that seems to run right into not only one, but \emph{two} singularities along the integration contour!

%% t0 general case




% %!TEX root = P231_notes.tex

\section{Kramers--Kr\"onig and Principal Value [optional]}
% \lecdate{lec~12}
% 2017 Lec 17ish

\subsection{Cauchy Principal Value}
% See Cohen, "Complex Analysis with Applications to Science and Engineering"
% p. 139

We have seen that the pole structure of an integrand has physical significance. When solving for the harmonic oscillator Green's function, we saw that we had to \emph{change} the nature of the harmonic oscillator differential operator in order to derive the \emph{retarded} (causal) Green's function rather than the \emph{advanced} (acausal) Green's function. In that case, we found a real integral with a pole along the integration axis and decided that the integrand itself was the problem.

What happens if \emph{we don't want to change the integrand?} For example, most physicists wouldn't object too much that the integral
\begin{align}
	\int_{-1}^4 dx\, \frac{1}{x-1}
\end{align}
should have a well defined value even though there is a simple pole at $x=1$. In fact, with a bit of thought (try sketching the integrand on a graph), one may argue that
\begin{align}
	\int_{-1}^4 dx\, \frac{1}{x-1}
	= 
	\int_{3}^4 dx\, \frac{1}{x-1}
\end{align}
because the integrand is antisymmetric about the pole. Sure, the integrand is undefined at $x=1$, but every contribution just to the right of the pole, say $x=1+\varepsilon$ is exactly canceled by a contribution just to the left of the pole, $x=1-\varepsilon$. This is purely \emph{real} calculus.

The \textbf{Cauchy principal value} of a real integral with a pole along the integration contour (the real axis) formalizes this idea. Suppose $f(x)$ is well behaved (analytic) along the real axis. For an integrand $f(x)/(x-x_0)$, the Cauchy principal value is defined to be
\begin{align}
	\mathcal P \int_{-\infty}^\infty dx\, \frac{f(x)}{x-x_0}
	= 
	\lim_{\varepsilon\to 0}
	\left[
	\int_{-\infty}^{x_0-\varepsilon} dx\, \frac{f(x)}{x-x_0}
	+\int_{x_0+\varepsilon}^\infty dx\, \frac{f(x)}{x-x_0}
	\right] \ .
\end{align}
In other words, you simply integrate everywhere except the pole along the real axis. Formally we cannot say anything about the integrand at $x=x_0$, but perhaps part of you secretly thinks this is not a big deal since the $\varepsilon\to 0$ limit is well behaved. Again, thus far we are only doing real calculus.

\flip{Include picture}

\paragraph{Relation to a contour.} We can now connect to complex analysis. Suppose $f(x)$ is well behaved enough that we can imagine integrating $f(z)/(z-x_0)$ along a closed contour with the integral along the ``arc at infinity'' (either in the upper or lower half plane) going to zero. Then we can write the following:
\begin{align}
	\int_{C(\gamma_\pm)} dz \; \frac{f(z)}{(z-x_0)} &= 
	% \int_{-\infty}^{x_0-\varepsilon} dx\, \frac{f(x)}{x-x_0}
	% +
	% \int_{x_0+\varepsilon}^\infty dx\, \frac{f(x)}{x-x_0}
	\left(\int_{-\infty}^{x_0-\varepsilon} 
	 	 	+
	 	 	\int_{x_0+\varepsilon}^\infty\right)  \frac{f(x) dx}{x-x_0}
	+
	\int_{\gamma_\pm} dz \; \frac{f(z)}{(z-x_0)}
	+
	\int_{\text{arc}} dz \; \frac{f(z)}{(z-x_0)} \ .
	\label{eq:cauchy:contour}
\end{align}
By assumption, the last term goes to zero. The first term is the Cauchy principal value when $\varepsilon\to 0$. The second term along contour $\gamma$ is a small semicircle of radius $\varepsilon$ that connects the two contours of the Cauchy principal value. It's a little semicircle that either goes just above ($\gamma_+$) or just below ($\gamma_-$) the real axis. Of course, you already realize the important point: the contour $C$ \emph{depends} on the choice of $\gamma_\pm$. This determines whether or not the point $z=x_0$ is inside or outside the contour.

\paragraph{Residue Theorem.} Let us assume that the integrand $f(z)/(z-x_0)$ is such that one can close the integral along the upper-half plane.\footnote{The argument is completely analogous if it were to be closed along the lower-half plane} This means that $f(z)/(z-x_0) \to 0$ fast enough as the radius of the semicircle goes to $R\to \infty$.\footnote{Remind yourself why `fast enough' is $1/R^2$ or faster.} Now the residue theorem tells us that
\begin{align}
	\int_{C(\gamma_\pm)} dz \; \frac{f(z)}{(z-x_0)} &= 2\pi i \sum_j \text{Res}_F(z_j)
	&
	F(z) = \frac{f(z)}{(z-x_0)} \ ,
\end{align}
where $j$ runs over the poles enclosed in $C(\gamma_\pm)$. When we take $\gamma_+$, this means that the poles do not include $x_0$. When we take $\gamma_-$, the poles do include $x_0$. The residue at $x_0$ is
\begin{align}
 	\text{Res}_F(x_0) &= f(x_0) \ .
\end{align}
In general, we expect $f(z)$ to have its own poles off the real axis; if any of those poles simple and are enclosed then they contribute to the sum. Just remember that if $z_j$ is a simple pole of $f(z)$, then the relevant residue is $(z_j-x_0)^{-1}\text{Res}_f(z_j)$.\footnote{Take the time to remind yourself why this is true if it is not obvious.}

\paragraph{Little semi-circles.}
The next step is to identify what the $\gamma_\pm$ integrals give. We may use the parameterization $z=x_0+\varepsilon e^{i\theta}$ to write
\begin{align}
	\int_{\gamma_+} dz \; \frac{f(z)}{(z-x_0)}
	&=
	\int_{\pi}^0 i\varepsilon e^{i\theta} d\theta \frac{f\left(x_0+\varepsilon e^{i\theta}\right)}{\varepsilon e^{i\theta}}
	=
	-i \pi f(x_0) \ .
\end{align}
We went ahead and took the $\varepsilon\to 0$ limit, implicitly using the analyticity of $f(z)$ at $z=x_0$. Similarly, if we used $\gamma_-$ as part of our contour,
\begin{align}
	\int_{\gamma_-} dz \; \frac{f(z)}{(z-x_0)}
	&=
	\int_{\pi}^{2\pi} i\varepsilon e^{i\theta} d\theta \frac{f\left(x_0+\varepsilon e^{i\theta}\right)}{\varepsilon e^{i\theta}}
	=
	+i \pi f(x_0) \ .
\end{align}

\paragraph{Putting it all together.}
We may then write \eqref{eq:cauchy:contour} as
\begin{align}
	\int_{C(\gamma_\pm)} dz \; \frac{f(z)}{(z-x_0)} &= 
	\mathcal P
	\int_{-\infty}^\infty dx\, \frac{f(x)}{x-x_0}
	+
	\int_{\gamma_\pm} dz \; \frac{f(z)}{(z-x_0)}
\end{align}
so that
\begin{align}
	\mathcal P
	\int_{-\infty}^\infty dx\, \frac{f(x)}{x-x_0}
	&=
	2\pi i \sum_j \left.\text{Res}_F(z_j)\right|_{C_\pm}
	\pm i \pi f(x_0) \ ,
	\label{eq:cauchy:principal:wrt:residues}
\end{align}
where we remind ourselves that the choice of contour ($\pm$) determines both the sign of the $\mp i \pi f(x_0)$ term \emph{and} whether or not the pole $z=x_0$ is enclosed by $C_\pm$. 
\begin{exercise}
Confirm that the right-hand side of \eqref{eq:cauchy:principal:wrt:residues} is independent of whether one closes the contour with $\gamma_+$ or $\gamma_-$.
\end{exercise}

\paragraph{Cauchy principal value and the $i\varepsilon$ notation.} %see Cohen.

Let us continue to assume that the contour $C$ is closed by an arc in the upper-half plane.\footnote{Recall that this amounts to an assumption about the convergence of $F(z)$ as $z\to Re^{i\theta}$ for large $R$ and $0 < \theta < \pi$.}
It is clear that closing the contour $C(\gamma_+)$ with $\gamma_+$ corresponds to excluding the pole at $z=x_0$ from the sum of residues. This means it is equivalent to a contour with the uninterrupted entire real axis $C_0$ for a modified integrand where the pole has been pushed downward:
\begin{align}
	\int_{C(\gamma_+)} dz \, \frac{f(z)}{z-x_0}
	=
	\lim_{\varepsilon\to 0}
	\int_{C_0} dz \, \frac{f(z)}{z-(x_0-i\varepsilon)} \ .
\end{align}
Similarly, for the contour $C(\gamma_-)$ with $\gamma_-$, the integral is equivalent to the contour that includes the uninterrupted real axis $C_0$ for a modified integrand where the pole is pushed upward into the contour:
\begin{align}
	\int_{C(\gamma_-)} dz \, \frac{f(z)}{z-x_0}
	=
	\lim_{\varepsilon\to 0}
	\int_{C_0} dz \, \frac{f(z)}{z-(x_0+i\varepsilon)} \ .
\end{align}
This gives a way to write \eqref{eq:cauchy:principal:wrt:residues} independently of the $\gamma_\pm$:
\begin{align}
	\mathcal P
	\int_{-\infty}^\infty dx\, \frac{f(x)}{x-x_0}
	&=
	% 2\pi i \sum_j \left.\text{Res}_F(z_j)\right|_{C_\pm}
	\lim_{\varepsilon\to 0}
	\int_{C_0} dz \, \frac{f(z)}{z-(x_0\mp i\varepsilon)} \ 
	% \mp i \pi \int dx\; f(x) \delta(x-x_0) \ .
	\pm i \pi f(x_0) \ .
\end{align}
Recalling that the contour integral over $C_0$ is really just the integral along the real line plus a vanishing integral over the arc, we may write the right-hand side as a real integral:
\begin{align}
	\mathcal P
	\int_{-\infty}^\infty dx\, \frac{f(x)}{x-x_0}
	&=
	% 2\pi i \sum_j \left.\text{Res}_F(z_j)\right|_{C_\pm}
	\lim_{\varepsilon\to 0}
	\int_{-\infty}^\infty dx\; f(x)\left[
		\frac{1}{x-(x_0\mp i\varepsilon)} \ 
		\pm i \pi   \delta(x-x_0) 
	\right] 
	 \ .
\end{align}
We see that a procedure for making sense of a real integral over a singularity gives us an expression that contains an imaginary part. This is often written at the level of \emph{distributions} as
\begin{align}
	\left.\frac{1}{x-x_0}\right|_P
	&= 
	\lim_{\varepsilon\to 0}
		\frac{1}{x-(x_0\mp i\varepsilon)} \ 
		\pm i \pi   \delta(x-x_0) \ .
		\label{eq:cauchy:principal:as:distribution}
\end{align}
By distribution we mean that the object \emph{only} makes sense in an integral, most likely being multiplied against another function. You already know that this expression only makes sense as a distribution because there's a `naked' $\delta$ function, and those things do not make sense outside of an integral. The $\left.\right|_P$ means principal value, that is: remember to put the $\mathcal P$ when you perform the integral. 

\subsection{Cauchy Principal Value + Cauchy Integral Representation}

Remember the Cauchy integral formula, \eqref{eq:cauchy:integral}? This told us that if a function $f$ is analytic around some point $z$, then we can express $f(z)$ as a contour integral around $z$:
\begin{align}
	f(z) = \frac{1}{2\pi i}\oint dz' \frac{f(z')}{(z'-z)} \ .
\end{align}
We first introduced this formula as a stepping stone to deriving the residue theorem. Now we have returned to the integral formula due to its striking resemblance to the integrals that popped up when describing principal values.

Suppose we want to take the limit where $z$ is the complexification of a real (physical) quantity. That is to say, we want to take the limit $z\to x$. For concreteness, let's assume that the physical limit is
\begin{align}
	z = \lim_{\varepsilon\to 0} x + i\varepsilon \ ,
\end{align}
so that $z$ approaches the real axis from above. This may be the case due to causality with our choice of sign conventions, as we saw for the harmonic oscillator Green's function.\footnote{You may be confused why we're approaching the real axis from above. We argued in the harmonic oscillator case that a causal theory has the poles pushed below the real axis---at least with our sign conventions for the Fourier transform. Because the physical pole approaches the real axis from below, we know that the function $f(x)$ is analytic in the region above (and in principle including) the real axis.} The relevant sign for the Cauchy principal value expression \eqref{eq:cauchy:principal:as:distribution} is
\begin{align}
	\lim_{\varepsilon\to 0} \frac{1}{x'-(x+i\varepsilon)}
	=
	\left.
	\frac{1}{x'-x}\right|_P
	+ i\pi \delta(x'-x) \ , 
\end{align}
where we've been careful with the choice of variable names.
\begin{exercise}
Write this expression for the case where the physically relevant limit approaches the real axis from above.
\end{exercise}
Let us now plug this into the Cauchy integral representation for $z = x-i\varepsilon$. Assume that $f(z)$ has all the usual convergence requirements\footnote{Pop quiz: what are these requirements? Ultimately there's a real integral that we have analytically continued into the complex plane. We want the integrand along the large arc around the lower-half plane to vanish.}
\begin{align}
	f(z) &= \frac{1}{2\pi i}\oint dz' \frac{f(z')}{z' - (x+i\varepsilon)}
	\\
	&=
	\frac{1}{2\pi i} 
	\left[	
		\mathcal P \int_{-\infty}^\infty dx \, 
		\frac{f(x')}{x'-x}
		-
		i\pi \int_{-\infty}^\infty dx \, f(x')\delta(x'-x)
	\right]
	\\
	&=
	\frac{1}{2\pi i} 
		\mathcal P \int_{-\infty}^\infty dx \, 
		\frac{f(x')}{x'-x}
		-\frac{1}{2}f(x) \ .
\end{align}
Simplifying this expression gives
\begin{align}
	f(x) &= \frac{1}{\pi i} 
		\mathcal P \int_{-\infty}^\infty dx 
		\, \frac{f(x')}{x'-x} \ .
\end{align}
This is now rather surprising! At this point, we have written everything in terms of the function $f(x)$ evaluated for \emph{real} arguments. Nothing in the integral along the real axis introduces any additional `imaginary-ness' to the right-hand side. However, there is this pernicious factor of $i$ on the right-hand side that seems to mix up the real and imaginary parts of $f$. To see this explicitly, write $f(x) = u(x) + i v(x)$. Then we have
\begin{align}
	u(x) &= \frac{1}{\pi}
	\mathcal P \int_{-\infty}^\infty dx 
		\, \frac{v(x')}{x'-x} 
		&
	v(x) &= -\frac{1}{\pi}
	\mathcal P \int_{-\infty}^\infty dx 
		\, \frac{u(x')}{x'-x} 
	\ .
\end{align}
To spell it out explicitly, we have the \textbf{Kramers--Kronig dispersion relations} (a name we justify below):
\begin{align}
	\text{Re}~f(x) &= \frac{1}{\pi}
	\mathcal P \int_{-\infty}^\infty dx 
		\, \frac{\text{Im}~f(x')}{x'-x} 
		&
	\text{Im}~f(x) &= -\frac{1}{\pi}
	\mathcal P \int_{-\infty}^\infty dx 
		\, \frac{\text{Re}~f(x')}{x'-x} 
	\ .
	\label{eq:KK:Cauchy:relations}
\end{align}
The remarkable observation is that the real and imaginary parts of the function $f$ are determined by one another through a principal value integral. The only assumption we made about $f$ is that it is is analytic just above the real line because the physical poles approach the real line from the lower half plane due to causality.

\begin{example}
The step-function ($\Theta$) is 1 for arguments larger than 0 and is otherwise zero. It is a convenient way to encode the causal properties of the retarded Green's function. \flip{FT of step...}
\end{example}

\subsection{Digression: Convolution Theorem}
\label{sec:convolution:theorem}

Suppose that the Fourier components of a function $f(t)$ is the product of the Fourier components of two other functions,
\begin{align}
	\tilde f(\omega) &= \tilde g(\omega) \tilde h(\omega) \ .
\end{align}
The convolution theorem relates the function $f(t)$ to $g(t)$ and $h(t)$. The trick is to insert 1 in a clever representation.\footnote{This may be the theme of the course: this is how you convert units, it is also how you project onto a different basis.} The particular representation of one is\footnote{We're breaking our conventions for taking a Fourier transform versus an inverse Fourier transform. However, because we end up doing both a Fourier and an inverse Fourier transform, the result is fully consistent.} 
\begin{align}
	1
	=\int d\omega' \delta(\omega'-\omega) 
	= \int \dbar t' e^{-i(\omega'-\omega)t'} \ .
\end{align}
Here's how it works:
\begin{align}
	f(t) &= \int \dbar \omega \, e^{-i\omega t}\, \tilde g(\omega) \tilde h(\omega)
	\\ 
	&= \int \dbar \omega  \, e^{-i\omega t} \tilde g(\omega)\, d\omega' \delta(\omega'-\omega) \tilde h(\omega')
	\\ 
	&= \int \dbar \omega d\omega' \dbar t'  \, e^{-i(\omega'-\omega)t'}e^{-i\omega t}\,  \tilde g(\omega)\tilde h(\omega')
	\\ 
	&= \int \dbar \omega  \dbar\omega' d t'  \, e^{-i\omega(t-t')}e^{-i\omega' t'}\,  \tilde g(\omega)\tilde h(\omega')
	\\ 
	&= \int dt'  \, g(t-t') h(t') \ .
	\label{eq:convolution:theorem:fourier}
\end{align}
You have already seen that Green's functions in time depend on the difference of the source and observation times, $G(t,t') = G(t-t')$.\footnote{Can you trace back why this is true?} The manipulation above shows that if $\tilde f(\omega) = \tilde g(\omega)\tilde h(\omega)$, then we may interpret one of $g$ or $h$ as a source and the other as a Green's function. Compare \eqref{eq:convolution:theorem:fourier} to our solution to $\mathcal O\psi = s$:
\begin{align}
	\psi(t) &= \int dt' \, G(t-t') s(t') \ .
\end{align}



\subsection{Example: Dielectric Media}

\paragraph{Brief review of electrostatics.}
A dialectric medium is one where there are bound charges that have some freedom to delocalize. When you subject the medium to an electric field, the charges separate a bit and polarize the medium.

\begin{center}
\includegraphics[width=.7\textwidth]{figures/Kramers_01.pdf}
\end{center}

We can split the charge density $\rho$ into free charges and bound charges so that the relevant Maxwell equation is:
\begin{align}
	\nabla \cdot \vec{E} &= 4\pi (\rho_\text{free} + \rho_\text{bound}) \ .
\end{align}
In turn, we may define the \textbf{polarization}, $\vec{P}$, with respect to the bound charge as the electric field sourced by the charge:
\begin{align}
	\rho_{\text{bound}} \equiv - \nabla \cdot\vec{P} \ .
\end{align}
This is a \emph{response} field to the initial electric field $\vec E$. Rearranging this gives
\begin{align}
	\nabla \cdot (\vec{E} + 4\pi \vec{P}) = 4\pi \rho_\text{free} \ .
\end{align}
We sometimes identify the dielectric displacement $\vec{D} \equiv \vec{E} + 4\pi \vec{P}$. For electric fields that are not too strong,\footnote{Can you explain on physical principles what `too strong' means? When does this model break down?}, many materials have the polarization parallel to the electric field:\footnote{You should appreciate why this is the `obvious' thing and reflect on what it would take for a material \emph{not} to obey this.}
\begin{align}
	\vec{P} \equiv \chi \vec{E} \ ,
	\label{eq:KK:polarization}
\end{align}
where we have defined the constant of proportionality to be $\chi$, the electric susceptibility. One may define the permittivity $\varepsilon \equiv 1+4\pi\chi$ so that it is the proportionality between the dielectric displacement and the electric field: $\vec{D} = \varepsilon \vec{E}$.

\paragraph{Waves through a medium.} Now let's imagine what happens when an electromagnetic wave passes through the medium. For our purposes, the wave is just a time dependence on the electric field, $\vec{E}(t)$.\footnote{And you should appreciate how much we're sweeping under the rug here.} The key point is that the susceptibility is \emph{frequency-dependent} $\chi(\omega)$. This should be clear to you intuitively: the microphysics of bound charges react to the changing electric field with respect to some characteristic time scale. It is a harmonic oscillator, after all.


In fact, we may write \eqref{eq:KK:polarization} as
\begin{align}
	\vec{P}(\omega) &= \chi(\omega) \vec E(\omega) \ .
	\label{eq:KK:P:XE}
\end{align}
The $\vec{P}(\omega)$, $\chi(\omega)$, and $\vec{E}(\omega)$ written here are the Fourier components of functions of time, $\vec{P}(t)$, $\chi(t)$, and $\vec{E}(t)$. Forgive me for not indicating tildes ($\tilde \chi$), hopefully making the arguments explicit will avoid confusion.


Let's be clear what we're doing: we want to treat the medium as some homogeneous background. In this background, the speed of light happens to be a bit slower: $v < c$. We define the ratio of the speed of light in vacuum to the speed of light in the homogeneous medium, $n = c/v$, the index of refraction. This is related to the microphysics by $n = \sqrt{\varepsilon\mu}$. For a pure dielectric (no magnetic funny business so $\mu=1$ in appropriate units), this is simply $n=\sqrt{\varepsilon} = \sqrt{1+4\pi\chi}$. What we're saying is that we \emph{know} that there's microphysics (bound charge) that does something weird. We also \emph{know} that fundamentally the speed of light is $c=1$. However, dealing with an Avogadro's number of bound charges is a pain. Instead, we realize that we can \emph{model} the complicated system of bound charges with respect to a susceptibility $\chi(\omega)$. This is an example of an effective theory, an idea that underscores everything that we do in physics.\footnote{Remember: we work with models. Good models are reasonable approximations with well defined limits where they are no longer good approximations. Do not confuse the model with nature, which is the complicated system that we're trying to approximate.}

Let's ignore polarization for this discussion. That's not where the key feature is. An electromagnetic wave of some polarization propagates as
\begin{align}
	\vec E(x,t) \sim \vec{E}_0 e^{ikx - i\omega t} \ .
\end{align}
For simplicity we assume a plane wave moving in the $x$ direction so we ignore the $y$ and $z$ dependence. The momentum/wave number $k$ is related to the angular frequency/energy by $k = \omega/v$, where $v$ is the velocity of the wave in units of $c$.
\begin{exercise}
Confirm for yourself that it is `obvious' that $v = \omega/k$ is the wave velocity. You can start with dimensional analysis, then remind yourself what $k$ and $\omega$ mean physically for a propagating wave.
\end{exercise}
Plugging in the susceptibility gives 
\begin{align}
	\vec E(x,t) \sim \vec{E}_0 e^{i\sqrt{1+4\pi \chi}x - i\omega t} \ .
\end{align}
Implicit here is the assumption that $\chi(\omega)$ is a real number. Perhaps it is not. However, it certainly has a real part. $\text{Re}[\chi(\omega)]$ is the \textbf{dispersion} of the medium. It gives the frequency-dependence of the index of refraction $n\sim v^{-1}$. This frequency dependence tells us how prisms and rainbows work according to Snell's law.

If there is an imaginary part, $\text{Im}[\chi(\omega)]$, it corresponds to \textbf{dissipation}. It's a damping of the wave at a given frequency. Writing
\begin{align}
	k = \frac{\omega}{v} = k_R + i \kappa \ ,
\end{align}
we have the expression for a dissipative wave:
\begin{align}
	\vec E(x,t) \sim \vec{E}_0 e^{-\kappa x} e^{ik_Rx - i\omega t} \ .
\end{align}

\paragraph{Kramers--Kr\"onig.} Let us now pull in two observations:
\begin{enumerate}
\item The susceptibility as a function of time, $\chi(t)$, is a Green's function. We can see this because \eqref{eq:KK:P:XE} tells us that $\vec{P}(\omega) = \chi(\omega \vec{E}(\omega))$ and we know from the convolution theorem in Section~\ref{sec:convolution:theorem} that this means $\vec{P}(t) = \int dt' \chi(t-t')\vec{E}(t)$, which is precisely the form of a Green's function solution to the polarization response to an external electric field $\vec{E}$.
\item We may invoke the Kramers--Kr\"onig relations \eqref{eq:KK:Cauchy:relations} to relate the real and imaginary parts of $\chi(\omega)$.
\end{enumerate}
Recall that the Kramers--Kr\"onig relations relate the real part of an analytic function on the real line to a principal value integral of the imaginary part of the function (and vice versa with `real' $\leftrightarrow$ `imaginary'). 

Is the susceptibility analytic? We know that physics is causal. So in the so-called time domain, $\chi(t)\sim \Theta(t)$. That is, $\chi(t<0)=0$. We want the retarded Green's function where the polarization is \emph{caused by} the external field $\vec{E}(t)$. The system may lag in its response to the external field---and indeed, this is the nature of the slower speed of light in the medium---but it may not \emph{anticipate} the electric field. This causality is imposed by an $i\varepsilon$ prescription on the poles of the Fourier integrand, just as we saw for the harmonic oscillator.\footnote{You can stop to think about why the bound charges really are harmonic oscillators.} Causality tells us that the poles are in the lower half plane (perhaps infinitesimally), so that the integral along and above the real line is analytic. Then the Kramers--Kr\"onig relations tell us
\begin{align}
	\text{Re}\left[\chi(\omega)\right]
	&=
	\frac{i}{\pi} \mathcal P \int d\omega' \frac{\text{Im}\left[\chi(\omega)\right]}{\omega'-\omega}
	\\
	\text{Im}\left[\chi(\omega)\right]
	&=
	\frac{-i}{\pi} \mathcal P \int d\omega' \frac{\text{Re}\left[\chi(\omega)\right]}{\omega'-\omega} \ .
\end{align}
Since $k=\sqrt{1+4\pi\chi}$, the real and imaginary parts of $\chi$ determine the dispersion (rainbows) and the dissipation (absorption) of the medium. Evidently, these two properties of a medium are not independent from one another. 

\begin{example}
You may wonder what this is good for. As a theorist, I certainly wonder this. It turns out that measuring the index of refraction is usually challenging. On the other hand, measuring absorption can be simpler: you just measure how much energy comes out of a material compared to how much energy went in. In this way, by measuring the absorption you can measure everything there is to know about the susceptibility of the material.
\end{example}

\begin{example}
This shows up in quantum mechanics and quantum field theory as the optical theorem. By calculating the amplitude for a particle to `fall out' of a particular Hilbert space due to decay into other particles, you also calculate corrections from virtual particles to its propagation. When you see this in quantum mechanics, you may stop a moment to think: ``it's odd that we're taking the imaginary part of this function.'' If you are too quick, you may dismiss this as quantum mechanics being strange because everything is complex. However, if you are cautious, you'll realize that we're really invoking cousins of the Kramers--Kr\"oning relations.
\end{example}

\paragraph{Why this has to be true.} He's a plausibility argument\footnote{Full disclosure: I learned this on \emph{Wikipedia}.} Imagine that you have some medium that you shine a Gaussian wavepacket of light into. By Gaussian I mean you shine a pulse of some duration that has the shape of a bell-curve in $t$:
\begin{center}
\includegraphics[width=.7\textwidth]{figures/Kramers_02.pdf}
\end{center}
The vertical axis is the field amplitude. As you know, the Fourier transform of a Gaussian is a Gaussian. So the pulse is a Gaussian in frequency space as well. 

Now suppose that this material had a magical property: it \emph{perfectly} absorbs \emph{one} frequency of light: $\omega_\times$. If you want to be realistic you can imagine some tiny-but-finite width around $\omega_\times$. Further, let's assume that there is no dispersion---it's a magical material for which all frequencies of light travel at the same velocity, except frequencies around $\omega_\times$ which do not propagate at all. This means that the electromagnetic wave \emph{after} passing through the material has a $\delta$-function gap in it at $\omega_\times$:
\begin{center}
\includegraphics[width=.5\textwidth]{figures/Kramers_03.pdf}
\end{center}
By linearity, we can think about this as a Gaussian plus a $\delta$-function spike:
\begin{center}
\includegraphics[width=.7\textwidth]{figures/Kramers_04.pdf}
\end{center}
\flip{The inverse Fourier transform of $\delta(\omega-\omega_\times)$ is a sine wave, not a straight line. Duh. The coefficient of $e^{-i\omega_\times t}$ is unity, but the function $1\times e^{-i\omega_\times t}$ is obviously not a flat line.}

Now we have a problem. The Fourier transform is also linear, so we can Fourier transform each component. The Gaussian in $\omega$ becomes another Gaussian in $t$. However, the $\delta$-function in $\omega$ becomes a \emph{flat distribution} in $t$! Remember Heisenberg's uncertainty relation? This is it: if you localize in one variable, you maximally delocalize in a conjugate variable. This should really bother you: this $-\delta(\omega-\omega_\times)$ spike gives contributions from $t<0$ to $\vec{E}_\text{after}(t)$! What do we make of this apparent paradox?


There is a nice resolution in terms of odd and even parts of $\chi$ with respect to time reversal. Let us define
\begin{align}
	\chi_+ &= \frac{1}{2}\left[\chi(t) + \chi(-t)\right]
	\\
	\chi_- &= \frac{1}{2}\left[\chi(t) - \chi(-t)\right] \ .
\end{align}
You already know that $\chi(t<0)$ is zero, but this separation is instructive since the above analysis seems to be probing $t<0$. Let's go to momentum space with the inverse Fourier transform:
\begin{align}
	\chi(\omega) &= \int dt\,  e^{i\omega t} \, \chi(t)
	\\
	&=
	\int dt\,  \left[\cos(\omega t)+i\sin(\omega t)\right] 
	\, \frac{1}{2}\left[\chi_+(t)+ \chi_-(t)\right] \ .
\end{align}
Now use the fact that the integral over all space of an odd times an even function vanishes. This means that the $\cos(\omega t)$ term only has non-zero integral when multiplying the $\chi_+(t)$ term, and similarly with $\sin(\omega t)$ with $\chi_-(t)$. Thus
\begin{align}
	\chi(\omega) &= 
	\frac{1}{2}\int dt\, 
	\cos(\omega t) \chi_+
	+
	\frac{i}{2}\int dt\, 
	\sin(\omega t) \chi_-
	\equiv
	\text{Re}[\chi(\omega)]
	+i
	\text{Im}[\chi(\omega)] \ ,
\end{align}
where we have explicitly identified the real and imaginary parts. That is to say: the real part is the piece that is \emph{even} in time and the odd part is the piece that is \emph{odd} in time. Both the even and odd pieces give acausal propagation, but when combined in equal parts, you have only causal propagation. The Kramers--Kr\"onig relation enforces causal propagation.

There's an excellent graphic from Wikipedia\footnote{``Kramers--Kronig relations,'' image by user FDominec, \acro{CC BY-SA 3.0}.}:
\begin{center}
\includegraphics[width=.9\textwidth]{figures/Kramers_05_Fdominec.pdf}
\end{center}
What we see in this figure is that the relation between the real and imaginary parts of $\chi(t)$ are precisely what forces $\chi(t<0) = 0$. In frequency space, one cannot simply posit that one frequency is absorbed without any implication on the dispersion of other frequencies. This would violate causality. The resolution to our paradox is that we cannot have a magical material that only absorbs in one frequency \emph{without} causing dispersion. We had stated this assumption with the [wrong] understanding that dispersion and disspation are independent properties that we can tune with microphysics. Evidently, one cannot construct a material whose polarizability is `stiff' to only one frequency without introducing a frequency-dependence to the speed of light across other frequencies.


\subsection{Fourier Transform of the Step Function}
% Following Appel section 11.1e

Our goal is to find the Fourier transform of the Heaviside step function, $\Theta(t)$. It turns out this has a nice representation with respect to principal values. A nice alternative derivation of the Kramers--Kr\"onig relations\footnote{"Kramers–Kronig in two lines," Ben Hu, \emph{American Journal of Physics} \textbf{57}, 821 (1989)} uses the Fourier transform of $\Theta(t)$, so it may be nice to see how one derives this.

\paragraph{A reminder about distributions.}
We've been a little glib about taking Fourier transforms of distributions. Remember that {distributions} are not quite functions. Just like relativists say that there's no such thing as a naked singularity and particle physicists say that there's no such thing as a free quark in a vacuum, there is no such thing as a $\delta$-function floating around without an integral. The integral may be implicit, but it is \emph{always} integrated over. 



\begin{example}
What are the Fourier coefficients of $\delta(t)$? Our inverse Fourier transform formula is \eqref{eq:inverse:fourier:convention}:
\begin{align}
	\tilde f(\omega) = \int dt\, e^{i\omega t} f(t) \ ,
\end{align}
so that we have $\tilde\delta(\omega) = 1$.
\end{example}

\begin{example}
What about the more nefarious object, the \emph{derivative} of a $\delta$-function? We simply have to integrate by parts to move that derivative off the $\delta$-function. 
\begin{align}
	\tilde \delta'(\omega) &= \int dt\, e^{i\omega t} \delta'(t)
	= -\int dt\, \left[\frac{d}{dt}e^{i\omega t}\right] \delta(t)
	= - i\omega \ .
\end{align}
\end{example}

One distribution that's particularly helpful for encoding causality is the Heaviside step function, $\Theta(t)$ which is zero for $t<0$ and one for $t>0$. It is convenient to further define that $\Theta(0) = 1/2$. It is useful to note that the derivative of the step function is the Dirac $\delta$ function: $\Theta'(t) = \delta(t)$. Can you see how to prove this?
\begin{example}
Because $\Theta(t)$ is a distribution, it only makes sense in an integral. The same holds for its derivative, $\Theta'(t)$. Imagine an integral over a test function $f(t)$ and integrate by parts with some modest assumptions about the boundary conditions of $f(t)$:
\begin{align}
	\int dt\, \Theta'(t) f(t) = - \int dt\, \Theta(t) f'(t)
	= f'(0) = \int dt\, \delta(t) f(t) \ . 
\end{align}
From this we identify $\Theta'(t) = \delta(t)$.
\end{example}

\paragraph{Fourier transform of the derivative of a distribution.} Given a distribution $p(t)$, one can find the Fourier components $\tilde p(\omega)$. What are the Fourier coefficients of the related distribution, $q(t) = p'(t)$? The Fourier transform of a distribution is, itself, a distribution. We may solve for $\tilde q(\omega)$ by integrating $\tilde q(\omega)$ over a test function:
\begin{align}
	\int d\omega\, \tilde q(\omega) \tilde f(\omega)
	&= 
	\int d\omega \left[dt\, q(t) e^{i\omega t}\right] \tilde f(\omega)
	\\	
	&= 
	\int d\omega dt\, q(t) \left[e^{i\omega t}  \tilde f(\omega)\right]
	\\	
	&= 
	\int d\omega dt\, \frac{d}{dt}p(t) \left[e^{i\omega t}  \tilde f(\omega)\right]
	\\
	&= 
	-\int d\omega dt\, p(t) (i\omega) e^{i\omega t}  \tilde f(\omega)
	\\
	&= 
	\int d\omega\, (-i\omega) \left[dt\,  p(t)  e^{i\omega t}\right]  \tilde f(\omega)
	\\
	&= 
	\int d\omega\, (-i\omega) \tilde p(\omega)  \tilde f(\omega) \ .
\end{align}
Comparing the first and last lines, we have
\begin{align}
	\tilde{p'}(\omega) = -i\omega \, \tilde p(\omega) \ .
	\label{eq:FT:Dist:derivative}
\end{align}




\paragraph{Principal values and distributions.}
What does all of this have to do with principal values? The Fourier transform of the step function contains a principal value distribution. To see this, we must build up some useful background.

First: if you know that $t\, p(t)=0$ for some distribution $p(t)$, then you must have
\begin{align}
	t \, p(t) = 0 \quad\Leftrightarrow \quad
	p(t) \propto \delta(t) \ .
\end{align}
Perhaps a hand-wavy intuitive explanation is useful here. If $p(t)=\delta(t)$, then the only non-zero point of the distribution is $t=0$. But multiplying by $t$ then forces the distribution to be zero at $t=0$ as well. Conversely, for any $t\neq 0$, the only way for $t\,p(t)=0$ is for $p(t)=0$ at that point. When $t=0$, $p(t)$ needn't be zero, and you may think that $p(t)=0$ is a valid solution. However, distributions are normalized so that $\int dt\, p(t) =1$, so $p(t)$ must be a $\delta$-function.

Next we note that a principal value integral can be interpreted as a distribution. Let us write
\begin{align}
	\mathcal P \int dx\, \frac{f(x)}{x-x_0} \equiv \int dx\, f(x) \left[\frac{1}{x-x_0}\right]_P \ ,
\end{align}
where the subscript $P$ on the right-hand side tells us that the bracketed term is understood to be a distribution that should be integrated over the real line \emph{except} around a ball of radius $\varepsilon\to 0$ about $x_0$. 

For simplicity, let us set $x_0=0$ and consider the principal value distribution $(1/x)_P$. The distribution $x (1/x)_P$ is, as one may expect from the notation, simply equal to one:
\begin{align}
	\int dx\, x\left( \frac{1}{x} \right)_P f(x)
	&= 
	\mathcal P \int dx\, \frac{x f(x)}{x}
	= 
	\mathcal P \int dx\, f(x)
	=
	\int dx\, f(x) \ .
\end{align}
This is obvious in that multiplying by $x$ has `plugged the simple pole' at $x=0$. 

We will also want to learn more about distributions $p(t)$ that satisfy $t\, p(t) = 1$. If this is the case, then we may define another distribution
\begin{align}
	q(t) &\equiv p(t) - (1/t)_P &
	\text{such that}&
	&
	t\,q(t) = t\, p(t) - t\left(\frac{1}{t}\right)_P = t\,p(t) - 1 = 0 \ .
\end{align}
Since $q(t)$ satisfies $t\,q(t)=0$, we use the prior observation to give
\begin{align}
	p(t) - \left(\frac{1}{t}\right)_P \propto \delta(t) \ .
	\label{eq:FT:Dist:PV:Delta}
\end{align}


\paragraph{Fourier transforming the step function.}
Finally, we can connect all of our observations this section to the Heaviside step function. First we use the fact that $\Theta'(t) = \delta(t)$. We invoke the rule for the Fourier coefficients of the derivative of a distribution, \eqref{eq:FT:Dist:derivative} to write
\begin{align}
	\tilde{\Theta'}(\omega) &= -i\omega \tilde \Theta(\omega) \ .
\end{align}
Inserting $\tilde{\Theta'}=\tilde\delta = 1$ gives
\begin{align}
	-i\omega\tilde \Theta(\omega) = 1 \ .
\end{align}
We know that distributions that satisfy $\omega p(\omega) =1$ have a nice property, \eqref{eq:FT:Dist:PV:Delta}, so that:
\begin{align}
	-i\tilde \Theta(\omega) - \left(\frac{1}{\omega}\right)_P \propto \delta(\omega) \ .
\end{align}
Define the constant of proportionality to be $-ic$ so that
\begin{align}
	\tilde \Theta(\omega) = 
	i\left(\frac{1}{\omega}\right)_P + c \delta(\omega) \ .
	\label{eq:FT:Dist:PV:Th:tilde}
\end{align}
To determine $c$, we invoke a bit of cleverness. The shifted step function $H(t)\equiv\Theta(t) - 1/2$ is real and odd in $t$. This means, in turn, that the Fourier transform of $H(t)$ is also odd. You can see this for any odd function $f(t)$ by examining the Fourier representation of $0=f(t)+f(-t)$ with a simple change in variables $\omega' = -\omega$ in the second term:
\begin{align}
	f(t)+f(-t) &= 
	\int \dbar \omega \,
 	e^{-i\omega t} \tilde f(\omega) 
	-
	\int \dbar \omega' \, e^{-i\omega'(-t)}  \tilde f(\omega') 
	\\&=
	\int \dbar\omega\, e^{i \omega t} 
	\left[\tilde f(\omega) - \tilde f(-\omega) \right] \ .
\end{align}
Since $f(t)-f(-t)=0$, we must have $\tilde f(\omega)- \tilde f(-\omega) =0$ and so the Fourier transform is also odd. We thus know that $\tilde H(t)$ is odd, which tells us that $\tilde\Theta(\omega) - \delta(\omega)/2$ must be odd. 

When we compare to the expression for $\tilde\Theta(\omega)$ in \eqref{eq:FT:Dist:PV:Th:tilde}, we see that the piece that is \emph{not} odd is $c\delta(\omega)$. We thus require $c$ to cancel the $\delta(\omega)$ piece in $\tilde\Theta(\omega) - \delta(\omega)/2$, which implies $c=1/2$. The result is a handy---if somewhat surprising---representation for the Fourier coefficients of the step function:
\begin{align}
	\tilde Theta(\omega) = i\left(\frac{1}{\omega}\right)_P + \frac{1}{2}\delta(\omega) \ .
\end{align}




%!TEX root = P231_notes.tex

\section{Green's functions in diverse dimensions}
% \lecdate{lec~12}
% 2018 Lec 27

Our usual approach is to take the simplest example and then focus on understanding the physics, while leaving each of you to generalize to the horrible real-world scenarios that you'll encounter in your other graduate courses and research. There's one generalization that we should take time to do properly: the shift from one dimension to multiple dimensions. This is the shift from treating time as a dependent variable to working on \emph{spacetime}. Equivalently, the harmonic oscillator converts into the wave equation.
%
Keep in mind that we're not doing any special relativity, even though we'll borrow notation from special relativity\footnote{There was once a professor teaching electrodynamics who used notation informed by relativity. The condensed matter students complained the class is about electricity and magnetism, not relativity. The professor replied: just where do you think magnetism comes from?}. 

\subsection{The `harmonic oscillator' in spacetime}
The second derivative, $d^2/dt^2$, is generalized in Euclidean space to the Laplacian, $\nabla^2$. But when you combine space and time (Minkowski space), there's the famous relative minus sign\footnote{At this level the minus sign is a convenient convention, but we know that the second derivative should really be relativistically invariant: $\partial^2 = \partial_\mu \partial^\mu$.}:
\begin{align}
	\frac{d^2}{dt^2}
	\to 
	\frac{1}{c^2}
	\frac{\partial}{\partial t}^2
	-
	\frac{\partial^2}
	{\partial \vec x^2} \ ,
\end{align}
where $\partial/\partial\vec x = \nabla$ is the gradient. Against all of my hard-developed instincts for natural units, I have replaced the explicit value of $c$ so that the operator makes sense dimensionally. Each term has powers of inverse length squared. In what follows I'm likely to make mistakes\footnote{My adviser used to say: if you think there may be a sign error or a factor of two error, then your homework is to fix those errors. In this case, if you think there's a missing factor of $c=1$, I'm not even sure if I'd even acknowledge that it's actually an error.} with the factors of $c$. 


In (1+1)-dimensions of spacetime\footnote{The (1+1) notation means one dimension of space, one dimension of time.} this is $\partial^2 = c^{-2} \partial_t^2 - \partial_x^2$. That looks familiar, doesn't it? The minus sign tells us that if we move forward in time a little bit, but look `backward' in space, then the function doesn't change. The description of the state at time $t-\delta t$ and position $x-\delta x$, subject to $\delta x = c\delta t$, is the same as the state at time $t$ and position $x$. 
%
This, in turn, means that the information was propagated \emph{forward} in space as time also moves forward. This is exactly what we expect from a wave. We recall that the solution to second derivative differential equations are usually trigonometric functions or their exponential counterparts. We further recall that `plane waves' are described with the same funny minus sign:
\begin{align}
	f(t) \sim \sin (\omega t - kx) \ .
\end{align}
From this you can read off that the plane wave velocity is $\omega/k$. Of course, you already knew that from dimensional analysis. 

\subsection{Green's functions for multidimensional spaces}
Our `harmonic oscillator' equation becomes:
\begin{align}
	\left[\frac{1}{c^2}
			\frac{\partial}{\partial t}^2
			-
			\frac{\partial^2}
			{\partial x^2}
			-
			\frac{\partial^2}
			{\partial y^2}
			-
			\frac{\partial^2}
			{\partial z^2}
		\right]
		\varphi(\vec{x},t) = \rho(\vec{x},t) \ .
		\label{eq:phi:wave:eq}
\end{align}
Huh, that looks familiar, doesn't it? We've chosen variables so that this looks just like the wave equation for the scalar potential $\varphi$ subject to a charged source $\rho$ in electrodynamics! In fact, you know how this works. There are three more equations that correspond to the vector potential:
\begin{align}
	\left[\frac{1}{c^2}
			\frac{\partial}{\partial t}^2
			-
			\frac{\partial^2}
			{\partial x^2}
			-
			\frac{\partial^2}
			{\partial y^2}
			-
			\frac{\partial^2}
			{\partial z^2}
		\right]
		\vec A(\vec{x},t) = \vec j(\vec{x},t) \ .
		\label{eq:A:wave:eq}
\end{align}
Now let's go through a series of questions about \eqref{eq:phi:wave:eq} and \eqref{eq:A:wave:eq} to make sure we're on the same page. 
\begin{enumerate}
\item \textbf{Are these differential equations still linear?} Yes. Remember what it means to be linear! $\mathcal O(f+g) = \mathcal Of +\mathcal Og$, for example.
\item \textbf{Are we worried that there are multiple arguments?} Not really. The functions now depend on $t$ and $\vec{x}=(x,y,z)$. You went from an infinite dimensional `vector space' to a still-infinite dimensional vector space\footnote{You can pontificate about whether or not this infinity has gotten `bigger.' It doesn't really make a difference.}. You can also mumble reassuring words to yourself, perhaps recalling how in prior encounters with the wave equation you've perhaps separated your functions into single-argument factors, $\Psi(t,x) = T(t)X(x)$, or something like that.
\item \textbf{How many equations are there?} There are \emph{four}\footnote{Obligatory TNG reference: \url{https://www.youtube.com/watch?v=wjKQQpPVifY}} equations. There is an equation for $\phi$ and one equation for each component of $\vec A$. In fact, we typically bundle this all up into a four-vector $A_\mu=(\varphi, \vec A)$. 
\item \textbf{How many differential operators are there?} Just one. There is only \emph{one} differential operator, $\partial^2$. It acts on different components of $A_\mu$, but it's the same wave operator acting on each component. 
\item \textbf{... so how many equations are there, really?} All four equations are essentially the same equation with different state functions and different sources. So it's really just one class of differential equation.
\end{enumerate}
Okay, here's the really important one. I'm going to put it in a box to make sure you pay really close attention. Please answer the following question before reading on:
\begin{framed}
\centering
How many Green's functions are there?
\end{framed}
Do we need a Green's function for each component of $A_\mu$? Do we need a Green's function for each dependent variable, $x^\mu = (ct,x,y,z)$? \emph{No}! There is only \emph{one} differential equation, and thus there is only \emph{one} Green's function. 




% A good exercise: phase and group velocity; see appell or cahill
%!TEX root = P231_notes.tex




\section{Coupled Harmonic Oscillators and Fields}
\label{sec:CHO:fields}

% CONNECT: initial value problem of Euler Lagrange vs path integral
% https://physics.stackexchange.com/questions/38348/is-the-principle-of-least-action-a-boundary-value-or-initial-condition-problem/38393#38393
% 

Let's take a break from Green's functions. In the previous section, we called the wave equation the \emph{harmonic oscillator in spacetime}. We proceeded to solve the wave equation for the electromagnetic field, $\partial^2 A_\mu = j_\mu$. We were very hand-wavy in arguing that the appropriate generalization of the harmonic oscillator operator $(d/dt)^2$ is the spacetime second derivative, $\partial^2 = c^{-2}\partial_t^2 - \nabla^2$. Furthermore, we noticed that our state functions went from being functions of only one variable, $\psi(t)$, to functions of space and time, $A_\mu(\vec{x},t)$. In other words, our states became \emph{fields}. It's instructive to see why a field is naturally understood as the continuum limit of a lattice of harmonic oscillators\footnote{I'm saying something `deep' here. Actual systems in condensed matter physics are atomic lattices with complicated electronic potentials. These are approximated by harmonic oscillators to leading order. At long wavelengths, the physics of this lattice often has a \emph{continuum limit} described by a field theory. The field theory is a model that has no underlying lattice, but whose long-wavelength predictions are designed to match that of a lattice with small spacing. In particle physics one usually assumes that nature is continuous so that the fundamental objects are actually fields. Whether or not this is literally true doesn't matter: it is sufficient that the range of phenomena that your theory hopes to describe are agnostic to whether or not there is a lattice. Sometimes particle physicists go the opposite way and use lattice techniques to make predictions where the continuum limit is difficult. The notion (and meaning) of a continuum limit is tied closely to the idea or renormalization group flow, which is one of the most elegant ideas in theoretical physics.}.


\subsection{A notational interlude}

Here's a bad habit: I often use the shorthand $x(t)$ to describe the position of a particle at time $t$. The particle doesn't need to literally be a particle, it could be the displacement of a spring from equilibrium. If the particle takes a position in three-space then we write $\vec{x}(t)$. Note that this is very different from a \emph{field}, where the spatial coordinates are a \emph{dependent} variable analogous to time. Just as $x(t)$ has a value for all $t$, a field $\varphi(x,t)$ takes some value for all $x$ and $t$. 

This is a potential source of confusion because we want to show the transition from a single harmonic oscillator to a field. The intermediate state is a lattice of coupled harmonic oscillators. To do this, let us break the bad habit of writing $x(t)$ and write the displacement of a harmonic oscillator to be $q(t)$. The displacement can be abstract, it doesn't have to be a distance separation; for example, it could be the value of a wavefunction at a point in space. If we have a bunch of harmonic oscillators evenly spaced on a line, we can index them $q_i(t)$. 

If you imagine that you have \emph{many} harmonic oscillators that are spaced very closely together, then rather then specifying an index you could just specify their position along the line, $x$. Note that $x$ is playing the role of an index, not as the state variable. In other words:
\begin{align}
	q_i(t) \to q(x,t) \ .
\end{align}
If I pick some point along the line, $x$, then the displacement of the harmonic oscillator at $x$ observed at time $t$ is $q(x,t)$. Congrats, $q(x,t)$ is now a field. You can see why we didn't want to use $x(t)$: it confuses the position along the line with the displacement of the harmonic oscillator at that position. 
%
The picture is as follows:
\begin{center}
\includegraphics[width=.5\textwidth]{figures/coupledHO.jpg}
\end{center}
The generalization to a dense three-dimensional lattice is straightforward, you end up with $q(x,y,z,t)$ or $q(\vec{x},t)$.


\subsection{Coupled Harmonic Oscillators}
\label{sec:CHO}

The Lagrangian for a single harmonic oscillator $q(t)$ is
\begin{align}
	L[q(t)] &= \frac{1}{2}m \dot q^2 - \frac{1}{2}k q^2 \ .
\end{align}
This assumes some reference equilibrium value $q=0$. Consider instead a series of identical beads of mass $m$ along a line that can move up and down freely, but are connected to their nearest neighbors by springs of uniform spring constant $k$. The Lagrangian for this system is
\begin{align}
	L[q_1(t), q_2(t), \cdots]
	&= 
	\sum_i \frac{1}{2} m \dot q_i^2 
	- 
	\sum_i \frac{1}{2}k (q_i - q_{i-1})^2 \ .
\end{align}
Because life is short, I'm not going to worry about the boundary conditions for the index $i$. You can assume that the system is periodic if you're nervous. Now let us pass into the continuum limit and consider some small separation $\Delta x$. We make the following replacement to continuum variables:
\begin{align}
	i & \rightarrow x
	\\
	q_i(t) & \rightarrow q(x,t)
	\\
	(q_i-q_{i-1})^2 &\rightarrow \Delta x^2 \left(\frac{\partial q}{\partial x}\right)^2
	\\
	\sum_i	&\rightarrow \int \frac{dx}{\Delta x}
	\\
	m &\rightarrow \rho \Delta x \ .
\end{align}
The factors of $\Delta x$ should be clear from dimensional analysis. You can think about them as some characteristic small length scale analogous to the lattice spacing. Observe that the mass of a single harmonic oscillator is replaced by the mass density $\rho$ multiplied by a `volume' ($\Delta x$ in one dimension). Once we write out the Lagrangian the $\Delta x$ factors almost cancel:
\begin{align}
	L 
	&= \int \frac{dx}{\Delta x} \, 
	\left[
		\frac{1}{2}\rho \Delta x 
		\left(\frac{\partial q}{\partial t}\right)^2 
		- 
		\frac{1}{2}k \Delta x^2
		\left(\frac{\partial q}{\partial x}\right)^2
	\right]
	\\
	&= \int dx\,
	\left[
		\frac{1}{2}\rho 
		\left(\frac{\partial q}{\partial t}\right)^2 
		- 
		\frac{1}{2} \left(k \Delta x\right)
		\left(\frac{\partial q}{\partial x}\right)^2
	\right] \ .
\end{align}
If we had a lattice in $D$ spatial dimensions, then this result generalizes to
\begin{align}
	L 
	&= \int d^Dx\,
	\left[
		\frac{1}{2}\rho 
		\left( \frac{\partial q}{\partial t} \right)^2 
		- 
		\frac{1}{2} \left(k \Delta x^{2-D}\right)
		\left(\frac{\partial q}{\partial \vec{x}}\right)^2
	\right] \ .
	\label{eq:HO:coupled:D}
\end{align}
Finally, let us rescale the state variable by defining a `better' state variable
\begin{align}
	\varphi(x,t) \equiv {\sqrt{\rho}}{q(x,t)} \ .
	\label{eq:field:normalization}
\end{align}
Further, let's go ahead and write the action $S=\int dt \, L$ with the idea of writing one big \emph{spacetime} integral over a Lagrangian \emph{denstiy} $S = \int dt\, d^Dx \, \mathcal L$:
\begin{align}
	S = \int dt \, d^Dx \,
	\frac{1}{2}
	\left[
	\left( 
		\frac{\partial \varphi}{\partial t}
		\right)^2
	- 
	c^2
	\left(\frac{\partial \varphi}{\partial \vec{x} }\right)^2
	\right] \ .
\end{align}
Observe that we have defined a speed
\begin{align}
	c^2 = \frac{k \Delta x^{2-D}}{\rho} \ .
	\label{eq:HO:coupled:D:c2}
\end{align}
If we take one more step and integrate by parts, we may write the action in the form:
\begin{align}
	S = \int dt \, d^Dx \,
	\frac{-c^2}{2}
	\varphi
	\left[
	\frac{1}{c^2}
	\left( 
		\frac{\partial}{\partial t}
		\right)^2
	- 
	\left(\frac{\partial }{\partial \vec{x} }\right)^2
	\right]
	\varphi 
	&=
	-c^2 \int d^{D+1}x \frac{1}{2} \varphi \partial^2 \varphi 
	\ .
\end{align}
Aha! Check it out, we have recovered the \emph{wave operator} $\partial^2$.  Now if you permit me some sloppy calculus, then if I squint at the action $S$ and vary with respect to $\varphi$, I get 
\begin{align}
	\delta S &=0 
	&\Rightarrow&&
	\partial^2 \varphi &= 0 \ .
	\label{eq:delta:S:operator:phi}
\end{align}
Let's be completely honest: we should do this variation more carefully... and we will, but just not right now; see Example~\ref{eg:varying:discrete:action}. But this result is rather compelling: $\partial^2\varphi(x,t) = 0$ is simply the wave equation.
\begin{exercise}
Confirm that \eqref{eq:HO:coupled:D} is the appropriate generalization to $D$ spatial dimensions.
\end{exercise}
\begin{exercise}
Confirm by dimensional analysis that $c$ in \eqref{eq:HO:coupled:D:c2} is a speed.
\end{exercise}

\subsection{Interpretation}

What we have described here is a lattice of harmonic oscillator states. The harmonic oscillator potential comes from each lattice point being connected `by a spring' to its nearest neighbor lattice points in each spatial direction. The propagation of a wave comes from a perturbation of single harmonic oscillator \emph{locally} affecting the harmonic oscillators nearby. Those oscillators, in turn, cause perturbations to \emph{their} neighbors. The speed $c$ is a characteristic speed of this propagation. In the continuum theory it's just some prefactor that is required by dimensional analysis. In the lattice theory it is related to the mass of each harmonic oscillator and the spring constant. 

Now that we've gone from the lattice to the continuum, you can just look at the Lagrangian for a field and see that it can be understood as each local piece of the field tugging at nearby parts of the field. 

\begin{example}
The relative minus sign between the time derivatives and the space derivatives in $\partial^2 = c^{-2}\partial_t^2 - \nabla^2$ is understood as the relative minus sign between the kinetic term and the potential term in the Lagrangian\footnote{You could, in turn, ask where that relative minus sign comes from. There are a few different ways to answer this, but I think the way that makes the most sense to me the one that is based on the path integral formulation of quantum mechanics.} 
\end{example}

\begin{exercise}
We assumed that each harmonic oscillator in the lattice has an identical mass $m$ and an identical spring constant $k$. What is the physical significance of this choice in the continuum representation?
\end{exercise}

\begin{exercise}
We assumed that each harmonic oscillator on the lattice only couples to its nearest neighbor. What would interactions with next-to nearest neighbors look like in the continuum representation? What is a good physical reason why you wouldn't have couplings to lattice points that are very far away\footnote{You can think about this problem the other way around and consider that the choice of the nearest neighbor couplings is \emph{defining} a sense of spacetime locality. Some theorists thinking about the nature of quantum mechanics propose that an analogous idea with quantum entanglement may be responsible for the macroscopic emergence of spacetime. See, e.g.~\texttt{arXiv:1606.08444}.}? \textsc{Hint}: we've already discussed this when we presented discretized function space... the only difference was when we did that, we started with a continuum and motivated a discrete representation. Now we're going in the other direction.
\end{exercise}


\subsection{A theoretical digression}
\label{sec:EFT:philosophy}

If you'll permit yet another digression, let me remind you that the reason why we spend so much time solving for the second derivative is that our models of physics tend to be local. Consider the action, $S = \int dt \, L$. If we have some state $\varphi$ that propagates in spacetime, then $\varphi$ is a \emph{field}. The natural form of the action is an integral with respect to a Lagrangian \emph{density},
\begin{align}
	S = \int d^4x \mathcal L = \int d^4x \left(\text{const.} + a \varphi + b \partial_\mu \varphi + \cdots\right) \ ,
\end{align}
where on the right-hand side we just started writing out \emph{all} possible polynomials of $\varphi(x,t)$ and the four-derivative $\partial_\mu$ with respect to coefficients $a, b,\cdots$. At this point we're not writing \emph{the} theory of the field $\varphi$, we are parameterizing \emph{all} theories of the field $\varphi$. Different choices for the infinite number of coefficients (typically called \emph{couplings}) correspond to different theories of the field $\varphi$.

We have already seen that the wave operator $\partial^2$ emerges from the kinetic term and the nearest-neighbor harmonic oscillator coupling on a lattice of spatial points. As we imagine the infinite sum of all possible terms in the action $S$, we write $\partial_\mu$ instead of $d/dt$ or $\nabla$ because we expect the theory to be Lorentz invariant\footnote{If your theory is not Lorentz invariant, then replace `Lorentz' with whatever symmetries your system does have. If it has no appreciable symmetries, then may Boltzmann have mercy on your soul.}.  In fact, we probably don't want to have any terms that have any free indices like $\partial_\mu \varphi$ because that means it transforms like a Lorentz vector and thus the term is not Lorentz invariant. Furthermore, we can use field redefinitions to remove linear terms\footnote{This is not an obvious statement in the action, but the variation of the action usually comes from a path integral, where on varies over $\varphi(x,t)$. Shifting $\varphi(x,t)$ is like integrating $dx$ versus $d(x+3) = dx$.}. Subject to symmetries, the action looks like:
\begin{align}
S = \int d^4x \left[\frac{1}{2}
%q\left(\partial^2 + \omega_0^2\right)q 
\left(\partial \varphi\right)^2
- \frac{1}{2}\omega_0^2 \varphi^2
+ c\varphi^3 + d\varphi^4 + e\partial^2 \varphi^2 + \cdots \right]	 \ .
\end{align}
You recognize that the term in parenthesis is simply the wave operator. It is usually convenient to lump together all of the quadratic terms
\begin{align}
	S_\text{quad.} 
	= 
	\int d^4x \, \left(\partial \varphi\right)^2
	- \frac{1}{2}\omega_0^2 \varphi^2
	=
	-
	\frac{1}{2}
	\int d^4x \, \varphi \,\mathcal{O}_\text{quad}\,\varphi \ .
\end{align}
When you vary this part of action and ignore the other terms, you end up with the appropriate \emph{linear} differential equation, $\mathcal O \varphi = 0$. This should make sense from basic calculus: if you have a quadratic function $f(x)=\frac{1}{2}ax^2$, then the first derivative is linear: $f'(x)=ax$ and the extrema satisfy $ax=0$. The wave equation comes from the same observation when you replace $x\to \varphi$ and $a\to \mathcal O_\text{quad.}$. Of course you're really going from a function $f(x)$ to a functional (function of functions) $S[\varphi]$, but this is a technical detail\footnote{You may want to check that this intuition is true by again imagining a discretized version of $\varphi$.}.


The operator $\mathcal{O}_\text{quad}$ includes the usual wave equation that represented a lattice of coupled harmonic oscillators, as well as the term $\omega_0^2 \varphi(x)^2$.  
\begin{exercise}
The $\omega_0^2$ term looks like the potential for a harmonic oscillator. How is it different from the $\nabla^2$ harmonic oscillator potential? How would you describe a lattice of harmonic oscillators with both the $\nabla^2$ and the $\omega_0^2$ potentials? \textsc{Hint}: what is the equilibrium value of the harmonic oscillators?
\end{exercise}

What about the other terms? First we should do a bit of dimensional analysis. For my own sanity, let's use natural units where we set $c=\hbar =1$ and all units are measure in mass dimension:
\begin{align}
	[x] = [t] &= -1 \ .
\end{align}
This means that $[d^4x] = -4$. Since the action is dimensionless in natural units---after all, it shows up in a $e^{iS}$---then we know the Lagrangian density has dimension $[\mathcal L] = +4$. Since $[\partial]=+1$, we deduce that the variable $q$ has mass dimension $[\varphi]=+1$. With that in mind, we can look at the `higher order' terms. The coefficients of these terms must have some \emph{negative} mass dimension:
\begin{align}
	c\, , d\, , e \, , \cdots = \left(\frac{1}{M}\right)^n \ ,
\end{align}
where $n$ is a positive integer. We've written the mass scale as $M$. Any mass scale in the theory must have physical significance. We could question whether the mass scale $M$ is big or small compared to either $\omega_0$ or the characteristic energies at which we are studying the system. If $M$ is small, then these coefficients are large, and their dynamics are important. However, if we include these terms in our dynamics, then the equations of motion become very nonlinear:
\begin{align}
	\mathcal O_\text{quad.} q + 3c\varphi^2 + 4d \varphi^3 + \cdots = 0 \ .
\end{align}
That would mean that the wave equation approximation is quite bad. Further, these additional terms are also clearly non-linear.  If we see physics that is approximately described by the wave equation, then these coefficients must be small. We thus expect $M$ to be large---it is an ultraviolet scale. 

What we intuit is that the scale $M$ is some energy scale where our theory is breaking down. If, for example, we were to \emph{experimentally}\footnote{In the \emph{gedanken} sense.} study this system at energies $E\sim M$, we expect that the effect of these non-linear terms are no longer small and become significant. In other words, the scale $M$ plays the role of a \emph{cutoff} at which our theory described by the linear operator $\mathcal O_\text{quad.}$ breaks down. For energies well below $M$, we can study the linear system and do \emph{perturbation theory} to study the first few non-linear coefficients that have coefficients that go like $(1/M)$ to a relatively small power. In the regime where our experiment has characteristic energy $E\ll M$, the in-principle infinite number of higher-order coefficients are negligible because we expect those effects to go like $(E/M)^n$; so keeping the first few should be an accurate description of the system.

This is the notion of an \textbf{effective theory}. All physical models are effective theories. We're simply parameterizing our ignorance about the universe and working in a regime where we are predictive. The effective theory philosophy explains why we are so obsessed with Green's functions of second-derivative operators:
\begin{itemize}
	\item Constant operators are trivial, they're not even differential.
	\item Operators with a single derivative are not Lorentz invariant. (The exceptions, like the diffusion operator, are only valid in the non-relativistic regime.)
	\item Operators with three, five, or any odd number of derivatives are not Lorentz invariant. They are also suppressed by inverse powers of the cutoff, $M$.
	\item Operators that are nonlinear, i.e.~terms in the Lagrangian density that have more than four powers of $q(x,t)$, are also suppressed by inverse powers of the cutoff $M$.
\end{itemize}
This tells you that in (3+1)-dimensions, the most interesting non-linear terms are $\mathcal L \supset c\varphi^3+ d\varphi^4$. If you assume that your theory has some $q\to -q$ symmetry, then you can even remove the $c$ term. By the way, this is how you build theories: you start with the most general $\mathcal L$ with an infinite number of terms. Then you argue based on symmetries and the relevance of high-dimensional operators which terms you can throw out. The result is that any reasonable theory in (3+1)-dimensions should be some perturbation of the harmonic oscillator/wave/Poisson system.

Let me layer onto this a bit more: implicit in writing out $S=\int d^4x\mathcal L$ is the idea that our theory should be manifestly \emph{local}. When we write terms like $q^3$, we really mean $\varphi(x,t)\varphi(x,t)\varphi(x,t)$ at the \emph{same} spacetime point. If this were not true, then the theory would be non-local and the causal structure of the theory would depend on the reference frame. This, in turn would put causality and Lorentz invariance at odds with one another and you'd have to pick one but not both. Some recent alternative formulation of quantum physics based on amplitudes (and not Lagrangians) have proposed that by giving up on \emph{manifest} locality, there may be more elegant descriptions of a theory. Those descriptions are local, they're just not obviously so.

\begin{exercise}
If we lived in Flatland, then the action would take the form $S = \int d^3x \mathcal L$. How does the power counting change for a state $q$? How does it change for a general number of spatial dimensions, $D$? Note: quantum effects can change the story quite a bit in $D=1$ and $D>3$. That's different story
\end{exercise}

\begin{exercise}
The Lagrangian density for electrodynamics is
\begin{align}
	\mathcal L = \frac{1}{4}F^{\mu\nu}F_{\mu\nu} \ ,
\end{align}
where $F_{\mu\nu}$ is the field strength tensor. Write out the corresponding equations of motion. Notice that you only get half of Maxwell's equations. Where does the other half come from? \textsc{Answer} (partial): the other half of Maxwell's equations come from the geometry of that system. It's most clear in the language of differential forms, but then one has to build up that mathematical machinery.
\end{exercise}

Finally, let me end this theoretical digression by re-emphasizing the effective theory philosophy. For the most part, a theory for the field $\varphi$ is an action with respect to this field. The equation of motion for the field come from $\delta S = 0$. The quadratic terms in $\varphi$ give a linear differential equation $\mathcal O_\text{quad} \varphi = 0$. In the presence of a source\footnote{One way to write the source is to put in a linear term in the Lagrangian density with the source $j(x)$ as a coefficient, $\Delta\mathcal L = -j(x)\varphi(x)$. The resulting differential equation is $\mathcal O_\text{quad.}\varphi =j(x)$.} there's some non-trivial right-hand side. This linear equation can be solved using Green's functions. The terms coming from higher powers of $\varphi$ are \emph{non-linear}, but they are also small as long as the theory is being studied at energies that are small compared to the dimensionful scale $M$ that suppresses those terms. This is the regime where the theory is under control (perturbative) and where it makes sense to talk about excitations of the $\varphi$ field. At much higher energies, this description may break down---for example, because it may become important to include the infinite number of non-linear terms. Or perhaps the underlying theory is a completely different theory where $\varphi$ were effective degrees of freedom\footnote{There is a lot of ongoing research in `pure' theoretical physics along these lines. One of the most interesting papers in the past 15 years has been \texttt{\href{https://arxiv.org/abs/hep-th/0602178}{hep-th/0602178}}, which pulls in many advanced ideas in quantum field theory, but is written in a way that anyone in this course can hopefully appreciate.}.
\begin{example}
Our theories of protons, neutrons, and electrons make sense until you study them at energies $E>\text{GeV}$ at which point the substructure of the nucleons becomes important. The protons and neutrons are no longer fundamental degrees of freedom and are replaced by a new description, quantum chromodynamics, that looks nothing like a theory of protons and neutrons. To say it differently, a chef doesn't need to know plasma physics when boiling a soup at 373 Kelvin. However, if you want to describe the phase transitions of early universe at $10^{13}$ Kelvin, then you really do need to know about subnuclear physics. The \emph{relevant} description of nature depends on the energy scale. Often times, the idea of a \emph{relevant} description is far more important than getting \emph{the right} description\footnote{One of the most intriguing ideas in theoretical physics is the existence of dualities that relate the mathematical description of two completely different theories. One of the implications of these observations is that there may not be a single `right' description of nature at a given energy scale. It may be that many completely different-looking theories make equivalent predictions about the same phenomena. Typically, though, only one of these descriptions is perturbative and easily calculable. Turning this observation upside down, one may use these dualities as a way to understand what happens to one description of nature when it breaks down---perhaps a dual description that was non-perturbative becomes perturbative in that break-down region. This turns out to be one of the main motivations to study extra dimensions.}.
\end{example}

If we only took the quadratic part of the action, then the theory is completely linear and every field looks like it's described by the wave equation. We call that case a \emph{free} field theory because it has no interactions and is completely solvable. Theories are distinguished by the kinds of interactions that \emph{perturb} the linear theory\footnote{The observation that all theories basically look alike up to a handful of parameters is called \textbf{universality}. This idea is formalized in theoretical physics in what is called the \textbf{renormalization group}.}. Suppose, for example, we included a $-\lambda \varphi^4$ term in the action. Now we have a theory that is not \emph{just} the wave equation---in fact, it is not even linear. Since our theory is non-linear, what good is any of the Green's function garbage we've learned in this class?! There are two standard options:
\begin{itemize}
	\item Numerical approach: simulate the system on a lattice and solve for the extrema of the action. This is called lattice QCD in the high-energy physics community. In the astronomy community this is just called `simulation.' 
	\item Feynman diagrams: do perturbation theory with respect to the couplings as long as they are small. 
\end{itemize}
I have little to say about the numerical approach other than that it is limited by computing power. There's always a handful of open questions in physics that one wonders: if we focused all of our supercomputing power on this problem for a week, maybe humanity would just know the answer.

The Feynman diagram approach is limited by perturbativity. The idea, though, is that you can use the Green's function from the linear (quadratic) part of the theory, but you have to \emph{perturb} about this Green's function by inserting powers of the non-linearity. The Green's function encodes the propagation of information from a source point in spacetime to the observation point in spacetime, hence the alternative name of `propagator'. When you include the non-linearities, you systematically deform the Green's function at intermediate points in spacetime. This is most easily represented graphically by diagrams where straight lines represent linear propagation through spacetime and vertices represent non-linear perturbations. For example, the $\varphi^4$ term is a vertex connecting four lines: this is because it's like four `harmonic oscillators' that are chained together; when one of them wiggles (because a wave hits it), it causes the other three to wiggle and leads to \emph{three} outgoing waves. We leave a careful exploration of this technique to a course on statistical field theory or quantum field theory. 

\begin{example}
If you have two fields, $\varphi$ and $\psi$, then the effective theory approach follows. You imagine writing down not only all possible polynomials in each field alone, but mixed polynomials like $\varphi^2\psi^2$ subject to the symmetries of your theory. The more symmetric your theory, the fewer terms are allowed. Higher-order terms that are suppressed by large powers of the heavy scale $M$ are negligible compared to terms with smaller $M$-suppression.
\end{example}


% potential future topic: fields from springs













%!TEX root = P231_notes.tex
\section{Green's Functions from Statistics and Gaussian Integrals}

In Section~\ref{sec:CHO:fields} we showed how fields emerge from a system of nearest-neighbor coupled harmonic oscillators. To leading order, fields are described by the wave operator. We remarked that the differential equation that governs a physical theory comes from a variation of the action functional $\delta S[\varphi]=0$. We even motivated how one builds theories: you start by imagining the most general action $S$ and then you remove terms because they violate symmetries or because they're suppressed by an energy scale $M$ much larger than the energy $E$ or physical phenomena you're examining. The latter point amounts to a $(E/M)^n$ suppression on physical observables. 

At this point, you may ask: well, where does $\delta S = 0$ come from? Curious, aren't we? It turns out that this connects to a few different ideas which we'll map out in this section. In case you haven't noticed, the latter part of these notes are much less technical than the first half. The point is no longer to teach you techniques. We are stepping back and seeing how these ideas interconnect with different theoretical structure that you may [soon] know. Don't worry if there are more results that are left to you to prove or look up, the goal isn't the technical derivation. The goal is to see the interconnection of these tools.

In this section, we'll make a connection to a bit of probability theory and a cute trick for solving Gaussian integrals. First, though, a quantum mechanical interlude.

\subsection{Quantum Mechanics: Green's Function from the Path Integral}

Time evolution by some amount of time $\Delta t$ in quantum mechanics is enacted by the Hamiltonian (energy) as an operator: $e^{i\hat H\Delta t}$. This time evolution may be described as a Green's function that takes an initial state $\Psi(q_0,t_0)$ and evolves it into a final state $\Psi(q, t)$. The wavefunctions are defined by a projection of a state $|\psi(t)\rangle$ onto the position basis $q$:
\begin{align}
	\Psi(q,t) = \langle q | \psi(t) \rangle \ .
\end{align}
Since $|\psi(t)\rangle$ is the time evolution of $|\psi(t_0)\rangle$ by a time $\Delta t = t-t_0$, we may continue:
\begin{align}
	\Psi(q,t) 
	&= \langle q | e^{-i\hat H \Delta t} |\psi(t_0) \rangle
	\\
	&= \int dq_0\langle q | e^{-i\hat H \Delta t} |q_0 \rangle\langle q_0 |\psi(t_0) \rangle 
	\\
	&= \int dq_0\,\langle q | e^{-i\hat H \Delta t} |q_0 \rangle \Psi(q_0, t_0)
	\ ,
\end{align}
where we have simply inserted a complete set of states (\emph{multiplied by one}) $\mathbbm{1} = \int dq_0 |q_0 \rangle\langle q_0 |$; this is simply the completeness relation that we recall from our review of linear algebra. We identify the quantity $\langle q | e^{-i\hat H \Delta t} |q_0 \rangle$ as the Green's function $G(q,t;q_0,t_0)$ since it now manifestly plays the role of a Green's function to solve for the state $\Psi(q,t)$ given the initial state $\Psi(q_0, t_0)$:
\begin{align}
	\Psi(q,t)\, 
	&= \int dq_0\, G(q,t;q_0,t_0) \Psi(q_0, t_0)
	\ . 
\end{align}
We didn't have to explicitly write the Schr\"odinger equation because that's already implicit in our starting point that $e^{-i\hat H \Delta t}$ is the time-evolution operator. 

If you go over the path integral formulation of quantum mechanics, one finds that by repeating this trick and inserting a complete set of states over \emph{many} time slices between $t_0$ and $t$ you end up with\footnote{You may refer to the first couple of lectures here: \url{https://sites.google.com/ucr.edu/p230b/}} a closed form expression for the Green's function:
\begin{align}
	G(q,t;q_0,t_0)  = \int \mathcal Dq(t) \, e^{iS[q]/\hbar} \ .
	\label{eq:G:QM}
\end{align}
\begin{exercise}
For those with some familiarity with quantum mechanics, derive the above result. Be sure to keep track of how $e^{i\hat H t}$ turns into $e^{iS[q]}$. I personally find this to be the most compelling motivation for defining the action $S$ as the difference of the kinetic and the potential energies. 
\end{exercise}
I've broken my usual convention of natural units and explicitly written out the $\hbar$ required to make $S[q]$ dimensionless. Recall that $[L]$ is energy and $S = \int dt \, L$ so that $[S] = E\times t$, which just happens to be the units of $\hbar$. The curious integration measure $\mathcal Dq(t)$ is an integral over different functions $q(t)$. The meaning is clear if we discretize in time:
\begin{align}
	\mathcal D q(t) = dq(t_0)\,dq(t_1)\,dq(t_2)\,\cdots dq(t_{N-1})\,dq(t_N = t) \ .
\end{align}
In other words, for each discrete time $t_i$, we vary the position $q(t_i)$ independently of the other positions. In this way, the integral over $\mathcal D q(t)$ is an integral over all possible functions $q(t)$ subject to the initial and final states.

You should interpret this expression as the famous quantum mechanical \emph{sum over histories}. The amplitude for a state to go from $\Psi(q_0,t_0)$ to $\Psi(q,t)$ includes a sum over the amplitude for each possible way to transition from the initial to the final state. At this point, you should recall the double slit experiment\footnote{The argument goes like this: imagine a double slit experiment. Now poke a third slit through the board; you sum over three paths. Now insert another board with two slits; you sum over $3\times 2 = 6$ paths. Now poke more holes, add more boards until you have an infinite number of boards each with an infinite number of holes. This corresponds to summing over all possible continuum paths from the initial to the final positions.}.
%
Evidently the quantity $e^{iS/\hbar}$ is the weight of each path. It corresponds to the amplitude of each path we're summing together. The total amplitude is given by the sum of these complex numbers with unit magnitude\footnote{You may want to look up `phasor' diagrams to see how to interpret this sum. Feynman describes this well in his for-the-public book \emph{QED: The Strange Theory of Light and Matter}, or you can see a more recent pop physics summary from PBS Spacetime: \url{https://youtu.be/vSFRN-ymfgE}.}.

$\hbar$ is also a measure of `quantumness'. Recall that it shows up in statistical mechanics due to the Gibbs paradox. The `quantum' of quantum mechanics refers to the fact that $\hbar\neq 0$ so that $[\hat p, \hat q] \neq 0$. As $\hbar\to 0$ one recovers the classical limit. We see that in the classical limit, the argument of the exponential $e^{iS/\hbar}$ becomes large and highly sensitive to variations of $S$. At this point one can make the \emph{saddle point approximation}. This is the observation that for $\hbar\to 0$, changes in $S[q]$ coming from nearby paths will rapidly change the phase of $e^{iS/\hbar}$ and cause the contributions of these nearby paths to cancel as one integrates over paths $q(t)$. There is one exception: when $S[q]$ is near an extremum (say, a minimum) then the contribution from nearby paths will change more slowly---this is obvious, you're at an extremum so the functional $S[q]$ is flat---and the those paths will dominate the integral. The result is that in the classical limit, the paths $q(t)$ that give the dominant contribution are simply those which realize an extremum of the action:
\begin{align}
	\delta S[q] = 0 \ .
\end{align}
And there you go, we have `derived' the principle of least action as the classical limit of time evolution in quantum mechanics. 

\begin{exercise}
How do you expect the Green's function \eqref{eq:G:QM} to change when we go from a \emph{particle} at $q(t)$ at time $t$ to a \emph{field} $\varphi(x,t)$ that takes in position $x$ and time $t$ as variables?  Congratulations, you're on your way to quantum field theory. 
\end{exercise}

\section{A probability refresher}

Now you may have a bit of mental whiplash as I change gears completely, but humor me a moment, we'll get back to the quantum mechanical picture shortly. 






\section{Closing Thoughts}

What I hope you've come to appreciate is that the mathematical machinery that you face in your first year of graduate school may appear daunting at first, but that you're never too far off from some variation of the harmonic oscillator. Green's functions are a powerful tool for solving differential equations, but we found that their \emph{analytic structure} tells is about the underlying physics of our theory. We waxed poetic about going from a harmonic oscillator to a field and how this is connected to the idea of extending time to spacetime. We danced carefully with the picture of functions as infinite-dimensional function spaces that could be understood in their discretized form as large-but-finite-dimensional vector spaces. We said a few words about the notion of effective theory and action principles. After arguing that everything really does reduce to something like a harmonic oscillator, we showed one way (perturbation theory) to deal with the \emph{non-linearities} that make \emph{actual} physics really interesting. There are any number of directions for you to go from here. Experimentalists and observers may want to dig into statistics and probability, condensed matter folks may flock to statistical mechanics while particle folks head to quantum field theory (only to discover that at the hearts of their respective fields they are speaking dialects of the same language\footnote{... that differ by an $i$}). From here on out, \emph{you} define your path through mathematical physics: what you need for your work, what tickles your fancy, and what gets a slice of your precious attention as you set forth in your scientific careers.



\section*{Acknowledgments}
%This work is supported in part by 
%the \textsc{nsf} grant \textsc{phy}-1316792. 
%
\textsc{p.t.}\ the students of Physics 231 (2016--2020) for their patience and feedback on on this course.

\appendix
\section{Method of Variations}
\label{app:method:of:variations}

The method of variations is a way to solve inhomogeneous differential equations starting from a basis of solutions to the homogeneous differential equation. We specialize to the case of second-order differential operators, though the technique is valid for higher-order operators.

\subsection{First order equation for coefficients}

Consider the second-order inhomogeneous linear differential equation
\begin{align}
	\mathcal Of(x) &= a_2(x)f''(x) + a_1(x) f'(x) + a_0(x) = g(x) \ .
\end{align}
One can determine the solution to this equation from varying with respect to the solutions to the homogeneous differential equation, $u(x)$ and $v(x)$. Write the solution to the inhomogeneous equation as
\begin{align}
	f(x) &= c(x)u(x) + d(x)v(x)
\end{align}
Make the ansatz that
\begin{align}
	c'(x) u(x) + d'(x) v(x) = 0 \ .
	\label{eq:method:of:variations:ansatz}
\end{align}
Then the inhomogeneous differential equation reduces to
\begin{align}
	\mathcal Of &= a_2(x)\left(c'(x) u'(x) + d'(x) v'(x)\right) =g(x) \ .
\end{align}
This gives expressions for $c'$ and $d'$:
\begin{align}
	c'(x) &= \frac{-v(x)g(x)/a_2(x)}{u(x) v'(x) - u'(x) v(x)}
	&
	d'(x) &= \frac{u(x)g(x)/a_2(x)}{u(x) v'(x) - u'(x) u(x)} \ ,
	\label{eq:method:of:variations:first:order}
\end{align}
where one may recognize the denominators to be the Wronksian, $W = uv' - u'v$. For simplicity, write $\tilde g(x) \equiv g(x)/a_2(x)$.
% Integrating these equations then gives the solution to the inhomogeneous equation. 

\subsection{Integrating and boundary conditions}

Suppose that the inhomogeneous differential equation has boundary conditions at $x_1$ and $x_2$ given by differential operators $\mathcal B_{1,2}$:
\begin{align}
	\mathcal B_1 f(x) &= \alpha_1 f'(x_1) + \beta_1 f(x_1) = 0
	&
	\mathcal B_2 f(x) &= \alpha_2 f'(x_2) + \beta_2 f(x_2) = 0 \ .
\end{align}
These boundary conditions are, in general, first order for a second order inhomogeneous differential equation. For example, these may come from integrating a second-order equation of motion by parts, leaving a first-order operator on either boundary.
%
The ansatz \eqref{eq:method:of:variations:ansatz} simplifies these boundary conditions:
\begin{align}
	\mathcal B_i f(x) &= 
	c(x_i) \mathcal B_i u (x) 
	+
	d(x_i) \mathcal B_i v (x) 
	= 0 
	\ .
	\label{eq:method:of:variations:BC}
\end{align}
For simplicity, let us write $(\mathcal B_i u) = \alpha_i u'(x_i) + \beta_i f(x_i)$ as a constant that depends on $u$ and $u'$ evaluated at the boundary $x_i$. We define $(\mathcal B_i v)$ similarly. 

Integrate the first-order differential equations for the coefficients \eqref{eq:method:of:variations:first:order} using these boundary conditions. The simplest approach is to integrate a convenient linear combinations:
\begin{align}
	\mathcal{I}_1 = 
	\int_{x_1}^x
	dy\, c'(y)(\mathcal B_1 u) + d'(y) (\mathcal B_1 v)
	&= c(x) (\mathcal B_1 u) + d(x) (\mathcal B_1 v) \ ,
	\label{eq:method:of:variations:int:1}
\end{align}
where the lower limit vanishes because $(\mathcal B_i f) \equiv 0$. Similarly, one finds
\begin{align}
	\mathcal{I}_2 = 
	\int_{x}^{x_2}
	dy\, c'(y)(\mathcal B_2 u) + d'(y) (\mathcal B_2 v)
	&= -c(x) (\mathcal B_2 u) - d(x) (\mathcal B_2 v) \ .
	\label{eq:method:of:variations:int:2}
\end{align}
The integrals in \eqref{eq:method:of:variations:int:1} and \eqref{eq:method:of:variations:int:2} are equivalently expressed using the expressions for $c'(x)$ and $d'(x)$ in \eqref{eq:method:of:variations:first:order}:
% \begin{align}
% 	\int_{x_1}^x
% 	dy\, c'(y)(\mathcal B_1 u) + d'(y) (\mathcal B_1 v)
% 	&= 
% 	\int_{x_1}^x dy \,
% 	\frac{-v(y)
% 	% g(x)/a_2(x)
% 	\tilde g(y)
% 	}{W(y)}
% 	(\mathcal B_1 u)
% 	+ 
% 	\frac{u(y)
% 	%g(x)/a_2(x)
% 	\tilde g(y)
% 	}{W(y)}
% 	(\mathcal B_1 v) 
% 	\\
% 	\int_{x}^{x_2}
% 	dy\, c'(y)(\mathcal B_2 u) + d'(y) (\mathcal B_2 v)
% 	&=
% 	\int_{x}^{x_2}
% 	dy\, \frac{-v(y)
% 	%g(x)/a_2(x)
% 	\tilde g(y)
% 	}{W(y)} 
% 	(\mathcal B_2 u) 
% 	+ \frac{u(y)
% 	%g(x)/a_2(x)
% 	\tilde g(y)
% 	}{W(y)} 
% 	(\mathcal B_2 v)
% 	\ .
% \end{align}
\begin{align}
	\mathcal{I}_1 
	%= 
	% \int_{x_1}^x
	% dy\, c'(y)(\mathcal B_1 u) + d'(y) (\mathcal B_1 v)
	&= 
	\int_{x_1}^x dy \,
	\left[
	- (\mathcal B_1 u) v(y)
	+ (\mathcal B_1 v) u(y)
	\right]
	\frac{
	\tilde g(y)
	}{W(y)}	
	\\
	\mathcal{I}_2 
	% = 
	% \int_{x}^{x_2}
	% dy\, c'(y)(\mathcal B_2 u) + d'(y) (\mathcal B_2 v)
	&=
	\int_{x}^{x_2}
	dy\, 
	\left[  - (\mathcal B_2 u) v(y) 
			+ (\mathcal B_2 v) u(y)
	\right]
	\frac{
	\tilde g(y)
	}{W(y)} 
	\ .
\end{align}
Combining this with the right-hand sides of \eqref{eq:method:of:variations:int:1} and \eqref{eq:method:of:variations:int:2} then gives the desired integral solutions for the coefficient functions:
\begin{align}
	\left[ \left(\mathcal B_1 u\right)\left(\mathcal B_2 v\right) - \left(\mathcal B_2 u\right)\left(\mathcal B_1 v\right)  \right]
	c(x)
	&= 
	% \phantom{+}
	\left(\mathcal B_2 v\right)
	\mathcal{I}_1 
	% \int_{x_1}^x dy \,
	% \left[
	% - (\mathcal B_1 u) v(y)
	% + (\mathcal B_1 v) u(y)
	% \right]
	% \frac{
	% \tilde g(y)
	% }{W(y)}	
	% \\
	% &\phantom{=}
	+
	\left(\mathcal B_1 v\right)
	% \int_{x}^{x_2}
	% dy\, 
	% \left[  - (\mathcal B_2 u) v(y) 
	% 		+ (\mathcal B_2 v) u(y)
	% \right]
	% \frac{
	% \tilde g(y)
	% }{W(y)} 
	\mathcal{I}_2 
	\\
	\left[ 
		\left(\mathcal B_2 u\right)\left(\mathcal B_1 v\right) 
		- 
		\left(\mathcal B_1 u\right)\left(\mathcal B_2 v\right)  
	\right]
	d(x)
	&= 
	\left(\mathcal B_2 u\right)
	\mathcal{I}_1 
	+ 
	\left(\mathcal B_1 u\right)
	\mathcal{I}_2 
	\ .
\end{align}
% Similarly for the $d(x)$ coefficient:
% \begin{align}
% 	\left[ 
% 		\left(\mathcal B_2 u\right)\left(\mathcal B_1 v\right) 
% 		- 
% 		\left(\mathcal B_1 u\right)\left(\mathcal B_2 v\right)  
% 	\right]
% 	d(x)
% 	&= 
% 	\phantom{+}
% 	\left(\mathcal B_2 u\right)
% 	\int_{x_1}^x dy \,
% 	\left[
% 	- (\mathcal B_1 u) v(y)
% 	+ (\mathcal B_1 v) u(y)
% 	\right]
% 	\frac{
% 	\tilde g(y)
% 	}{W(y)}	
% 	\\
% 	&\phantom{=}+
% 	\left(\mathcal B_1 u\right)
% 	\int_{x}^{x_2}
% 	dy\, 
% 	\left[  - (\mathcal B_2 u) v(y) 
% 			+ (\mathcal B_2 v) u(y)
% 	\right]
% 	\frac{
% 	\tilde g(y)
% 	}{W(y)} 
% 	\ .
% \end{align}
Define the left-hand side prefactors:
\begin{align}
	(\text{den}.) &\equiv
	\left(\mathcal B_1 u\right)\left(\mathcal B_2 v\right) - \left(\mathcal B_2 u\right)\left(\mathcal B_1 v\right)  \ ,
\end{align}
so that we have 
\begin{align}
	f(x) &= 
	\frac{\left(\mathcal B_2 v\right)
		\mathcal{I}_1 
		+
		\left(\mathcal B_1 v\right)
		\mathcal{I}_2}{(\text{den.})}
	u(x)
	-
	\frac{\left(\mathcal B_2 u\right)
		\mathcal{I}_1 
		+
		\left(\mathcal B_1 u\right)
		\mathcal{I}_2}{(\text{den.})}
	v(x)
\end{align}
Collecting the terms by the integration range gives:
\begin{align}
	(\text{den.})f(x)&=
	\int_{x_1}^x dy \frac{\tilde g(y)}{W(y)}
	\left[
		-(\mathcal B_1u)(\mathcal B_2u) v(x)v(y)
		+(\mathcal B_1v)(\mathcal B_2v) u(x)u(y)
		\right.
	\\
	&\phantom{=\int_{x_1}^x dy \frac{\tilde g(y)}{W(y)}[}
		\left.
		-(\mathcal B_1u)(\mathcal B_2v) u(x)v(y)
		+(\mathcal B_1v)(\mathcal B_2u) v(x)u(y)
	\right] + \cdots
	\\
	&=
	\int_{x_1}^x dy \frac{\tilde g(y)}{W(y)}
	\left[
		(\mathcal B_1u)v(y) - (\mathcal B_1v)u(y)
	\right] 
	\left[
		(\mathcal B_2u)v(x) - (\mathcal B_2v)u(x)
	\right] + \cdots \ .
\end{align}
The $\cdots$ represent the $\mathcal I_2$ terms. Performing the same grouping for the $\mathcal I_2$ terms then gives:
\begin{align}
	(\text{den.})f(x)&=
	\int_{x_1}^x dy \frac{\tilde g(y)}{W(y)}
	\left[
		(\mathcal B_1u)v(y) - (\mathcal B_1v)u(y)
	\right] 
	\left[
		(\mathcal B_2u)v(x) - (\mathcal B_2v)u(x)
	\right] + \\
	&\phantom{=}
	\int_{x}^{x_2} dy \frac{\tilde g(y)}{W(y)}
	\left[
		(\mathcal B_1v)u(x) - (\mathcal B_1u)v(x)
	\right] 
	\left[
		(\mathcal B_2v)u(y) - (\mathcal B_2u)v(y)
	\right] \ .
\end{align}
These two terms can be combined rather nicely:
\begin{align}
	f(x) &=
	\frac{1}{(\text{den.})}
	\int_{x_1}^{x_2} dy \,
	\frac{\tilde g(y)}{W(y)}
	\left[
		(\mathcal B_1u)v(x_<) - (\mathcal B_1v)u(x_<)
	\right] 
	\left[
		(\mathcal B_2u)v(x_>) - (\mathcal B_2v)u(x_>)
	\right]
\end{align}
where we introduce the convenient notation that
\begin{align}
	x_< &= \min(x,y)
	&
	x_> &= \max(x,y) \ .
\end{align}

%!TEX root = P231_notes.tex

\section{Fourier Conventions}
\label{app:Fourier}
% \lecdate{lec~12}
% 2017 Lec 15

This section is inspired by an excellent post on \texttt{physics.stackexchange}\footnote{\url{https://physics.stackexchange.com/a/308248}}. There are many different conventions for Fourier transforms. The danger is that you accidentally use one convention to do the Fourier transform and a different convention for the inverse transform. 

There are two choices one can make when defining a Fourier transform convention; we parameterize these choices by real numbers $a$ and $b$. The Fourier transform $\tilde f(\omega)$ of a function $f(t)$ is
\begin{align}
	\tilde f(\omega)
	&= 
	\sqrt{\frac{|b|}{(2\pi)^{1-a}}}
	\int_{-\infty}^\infty dt\, e^{ib\omega t} f(t) \ .
\end{align}
We see that $a$ tells us about the $(2\pi)$ factors and $b$ tells us about the argument of the basis function $e^{ib\omega t}$. With this basis, the inverse Fourier transform is 
\begin{align}
	f(t)&=
	\sqrt{\frac{|b|}{(2\pi)^{1+a}}}
	\int_{-\infty}^\infty d\omega\, e^{-ib\omega t} f(\omega) \ .
\end{align}

One may check that the inverse Fourier transform of a Fourier transform gives the original function:
\begin{align}
	\tilde{\tilde f} &=
	\frac{|b|}{2\pi}
	\int_{-\infty}^\infty d\omega\, e^{-ib\omega t}
	\int_{-\infty}^{\infty}
	ds\, e^{ib\omega s} f(s)
	\\
	&= 
	\frac{|b|}{2\pi}
	\int ds\, f(z) \int d\omega \, e^{ib\omega(s-t)}
	\\
	&= \int ds\, \delta(s-t) f(s) \ ,
\end{align}
where we have used $\int d\xi \exp(2\pi i x\xi) = \delta(x)$. 

The convention that we will choose for the \emph{time}--\emph{frequency} [inverse] Fourier transform is
\begin{align}
	f(t) &= \int_{-\infty}^{\infty} \dbar\omega e^{-i\omega t} \tilde f(\omega)
	&
	\dbar \omega &\equiv\frac{d\omega}{2\pi} \ .
\end{align}
This corresponds to $a=b=1$. The corresponding transform for the frequency-domain function is
\begin{align}
	\tilde f(\omega) &= 
	% \frac{1}{2\pi}
	\int_{-\infty}^\infty d t\, e^{i\omega t} f(t) \ .
\end{align}


All of this generalizes to higher dimensions: you simply Fourier transform each dimension. In fact, one is free to use a different Fourier transform convention for each direction. We can use this freedom to pick a convention that `automatically' fits our conventions for spacetime. In particular, given a four-vector $x=(t,\vec{x})$ and its conjugate four-momentum $p=(\omega, \vec{k})$, one may choose to Fourier transform as follows: 
\begin{align}
	f(x) &= \int \dbar\omega\dbar^3\vec{k} e^{-i(\omega t-\vec{k}\cdot\vec{x})} \tilde f(p)
	\ .
\end{align}
With this convention, the basis function is simply
\begin{align}
	e^{-i(\omega t-\vec{k}\cdot\vec{x})} 
	= e^{-ip\cdot x} \ , 
\end{align}
where $p\cdot x$ is the usual Minkowski dot product, $p_\mu x^\mu$. This makes it clear that the basis function is Lorentz invariant. The Fourier transform would still respect the spacetime symmetries even if we had not chosen a convenient notation---it just wouldn't be as simple to see.
%!TEX root = P231_notes.tex
%% From: 2917 Lec 22 and onward
\section{Nuts and Bolts Probability}
\label{sec:probability}

Let's forget Green's functions now and follow up on something to do with probability: Bayes' theorem. This is fully outside the scope of the class, but it is significant enough that it deserves to be mentioned in a `math methods' course. 

You are already familiar with \emph{conditional probability}. If you roll two 6-sided dice, the probability of each coming up with one is simply the product of each probability:
\begin{align}
	P(\text{both dice are 1}) = P(\text{one die is 1})^2 \ .
\end{align}
In this case we're asking about two events, $A$ (first die rolls 1) and $B$ (second die rolls 1). In general, though, these two events may not be independent. In this case, one has to deal with \emph{conditional probability}. We write $P(A|B)$ to mean the probability of $A$ assuming that $B$ is true. This, in turn, is related to the probability that $A$ \emph{and} $B$ are true divided by the probability that $B$ is true:
\begin{align}
	P(A|B) &= \frac{P(A\& B)}{P(B)} \ .
	\label{eq:P:A:given:B}
\end{align}
Stop to make sure this makes sense: if you write the probabilities as
\begin{align}
	P(X) = \frac{\text{number of times $X$ happens}}{\text{total number of samples}} \ ,
\end{align}
then the right-hand side of \eqref{eq:P:A:given:B} is simply
\begin{align}
	P(A|B) = \frac{\text{number of times $A$ and $B$ happen}}{\text{number of times $B$ happens}} \ , 
\end{align}
which is precisely what we want from $P(A|B)$: what is the probability that $A$ is true if we already know $B$ is true. Once we say $B$ is true, we can throw away all the samples where $B$ is not true.

Here's the key insight: $P(A|B)P(B) = P(A\& B)$. On the right hand side, $A\&B$ is completely symmetric in $A$ and $B$. That means we could swap them on the left-hand side as well: 
\begin{align}
	P(A\&B) = P(B\& A) = P(B|A)P(A) \ . 
\end{align}
We thus have $P(A|B)P(B)=P(B|A)P(A)$, which gives us the famous \textbf{Bayes theorem} that relates the conditional probabilities $P(A|B)$ and $P(B|A)$:
\begin{align}
	P(A|B) &= \frac{P(B|A) P(A)}{P(B)} \ .
\end{align}
Here's a useful graphical representation from Bob Cousins:
\begin{center}
\includegraphics[width=.8\textwidth]{figures/ConditionalProb.png}
\end{center}
What is important here is that $P(A|B)$ and $P(B|A)$ can differ significantly if $P(A)/P(B)$ is very big or very small. This can be problematic because human beings can be sloppy when distinguishing $A|B$ versus $B|A$. 
\begin{example}
The probability that someone knows how to use \texttt{ROOT}\footnote{\url{https://root.cern/}} given that they are a particle physicist is roughly $P(\text{\texttt{ROOT}}|\text{particle person}) \approx 50\%$. To rough approximation, particle experimentalists use \texttt{ROOT} and theorists avoid it like the plague. However, the probability that someone is a particle physicist given that they use \texttt{ROOT} is $P(\text{particle person}|\text{\texttt{ROOT}}) \approx 100\%$. If someone uses \texttt{ROOT}, then you can be pretty sure that they do particle physics.
\end{example}

There's a simple reason why this is important: usually $A$ is something that we can actually see and $B$ is something that we want to \emph{infer} but cannot directly test. Specifically, in physics we typically have the following case:
\begin{align}
	A &= \text{data} 
	&
	B &= \text{theory} \ .
\end{align}
$A$ is the particular data that an experimentalist or an observer actually records. $B$ roughly a statement about whether a theory is correct. 
% 
We've already made may caveats in Section~\ref{sec:EFT:philosophy} about the idea of effective theories and what it means for a theory to be `correct.' To summarize, usually $B$ is asking: is a given theory a good description of the underlying phenomenon related to the data $A$? 

At this point, I'll leave it to you to dig deeper into the formalism for probabilities\footnote{Some of my favorite references: \emph{Statistics: A Guide to the Use of Statistical Methods in the Physical Sciences}, the lectures by Bob Cousins \texttt{1807.05996} (and his 1995 \emph{American Journal of Physics} article), and Kyle Cranmer's statistics and data science textbook~\url{https://cranmer.github.io/stats-ds-book/}.}. In summary, you should be precise about your hypothesis tests and be able to clearly articulate the meaning of whatever figure of merit you're using to interpret your data. In what follows, let's just play with some silly manifestation of these ideas. 

\subsection{Were we lucky that the LHC didn't destroy the world?}
\label{sec:LHC:luck}

One of the favorite discussion topics when I was a graduate student was a funny clip from \emph{The Daily Show} about the Large Hadron Collider in 2009\footnote{\url{http://www.cc.com/video-clips/hzqmb9/the-daily-show-with-jon-stewart-large-hadron-collider}}
. At the time, there was speculation that the LHC could destroy the world. There were many fun and silly ideas for how this could come about, though there were many very good reasons why these ideas were more `silly science fiction' rather than environmental concern\footnote{\url{https://home.cern/science/accelerators/large-hadron-collider/safety-lhc}}. During the clip, John Oliver---then a correspondent for John Stewart's \emph{The Daily Show}---interviews some dude concerned about the possibility that the \acro{LHC} may destroy the world. The interview went something like this:
\begin{quote}
\textsc{Oliver}: What are the chances that the world is going to be destroyed?\\
\textsc{Dude}: It's about a one in two chance.\\
\textsc{Oliver}: 50--50?\\
\textsc{Dude}: It's either going to happen or not happen. So the best guess is 50--50.
\end{quote}
\begin{exercise}
What's wrong with that dude's argument?
\end{exercise}
 

\subsection{The Prosecutor's Fallacy}


This comes from Bergstrom and West's book \emph{Calling Bullshit}. Imagine some high-profile \emph{Ocean's Eleven} or \emph{Carmen Sandiego} style heist. The \acro{FBI} is able to recover a set of fingerprints which they run through their database of 50 million people and guess what: \emph{you're} a match. Against everyone's advice, you decide to represent yourself in court. When the \acro{FBI} agent explains that your fingerprint match makes is an open-and-shut case, you politely ask: \emph{what is the chance that the wrong fingerprint would be matched with a print in the database?}

``Hah!'' the prosecutor scoffs at you. ``The chance of this happening happens to be \emph{one in ten million}. That's beyond any reasonable doubt.'' Fortunately you know a thing or two about probability and there's a chalkboard in the courtroom. You draw the following table:
\begin{center}
\begin{tabular}{l|ll} \toprule % @{} removes space
		& Match & No Match
		\\ \hline
		Guilty & \phantom{1 person} & \phantom{0 people}
		\\
		Innocent & \phantom{5 person} & \phantom{50,000,000 people}
		\\ \bottomrule
\end{tabular}
\end{center}
You start by stating the obvious assumptions. There is \emph{one} guilty person---not you---whose fingerprints match the prints lifted from the crime scene. While it's neither here nor there, you also point out that there are no people who are both guilty but whose prints do not match those at the crime scene. You start to fill in the table:
\begin{center}
\begin{tabular}{l|ll} \toprule % @{} removes space
		& Match & No Match
		\\ \hline
		Guilty & {1 person} & {0 people}
		\\
		Innocent & \phantom{5 person} & \phantom{50,000,000 people}
		\\ \bottomrule
\end{tabular}
\end{center}
Next you say that there are approximately 50 million people in the fingerprint database. Almost all of theme are innocent and almost all of them are completely distinct from the evidence from the crime scene. You fill in the table a bit more:
\begin{center}
\begin{tabular}{l|ll} \toprule % @{} removes space
		& Match & No Match
		\\ \hline
		Guilty & {1 person} & {0 people}
		\\
		Innocent & \phantom{5 person} & {50,000,000 people}
		\\ \bottomrule
\end{tabular}
\end{center}
Finally, you point out: given a false-match rate of one in 10 million, you can estimate that around \emph{five} innocent people will have fingerprints that match those at the crime scene. 
\begin{center}
\begin{tabular}{l|ll} \toprule % @{} removes space
		& Match & No Match
		\\ \hline
		Guilty & {1 person} & {0 people}
		\\
		Innocent & {5 person} & {50,000,000 people}
		\\ \bottomrule
\end{tabular}
\end{center}
The prosecutor starts to sweat, they can see what's coming. You point out that the \emph{data} in the case is that you are one person whose fingerprints match those at the crime scene. That means we should focus only on the population of people whose fingerprints match those the crime scene: there are approximately six: five innocent and one guilty. You point to the numbers and say, ``I know that my \emph{Math Methods} class in graduate school wasn't \emph{especially} rigorous, but clearly there is a five in six chance that even though my fingerprint matches the one from the crime scene, I am in fact innocent.''
\begin{exercise}
What was the mistake that the prosecutor made?
\end{exercise}
Bergstrom and West made up the above story to explain the fallacy, but they point out that this scenario indeed played out in 2018 in the highly publicized case that led to the capture of the Golden State Killer using \acro{DNA} samples from a genealogy company. What was not shared widely in the press coverage was that the genetic screening alone initially led to the \emph{wrong} suspect. The true culprit was identified by combining the \acro{DNA} match with other evidence from more traditional detective work. Timely applications to medicine are all over\footnote{\url{https://youtu.be/lG4VkPoG3ko}}.

\subsection{Why we are not evidence for God}

Mathematician Jordan Ellenberg takes the above example a bit further to highlight another fallacy. This comes from his pop mathematics book \emph{How Not To Be Wrong}, which you're probably a bit advanced for, but is none-the-less a fun read and a lesson in effective technical communication to a lay audience. 

There is an argument that the universe we observe is evidence for God. After all, look how incredibly engineered and fine-tuned life is\footnote{You can also point to more physical arguments, like the apparent fine tuning of the cosmological constant.}. Indeed all this \emph{stuff} that we see around us does seem rather complicated---why isn't the universe more like the spherical cow\footnote{\url{https://en.wikipedia.org/wiki/Spherical_cow}} theories that we study in grad school? 

The argument is something like the one in Appendix~\ref{sec:LHC:luck}. Let's write down the \emph{likelihood} that we see this complicated universe. Ellenberg makes some very rough estimates that will be sufficient for our point. The existence of an omnipotent creator could seems to be far more likely to have created such a finely-tuned, complicated universe.
\begin{center}
\begin{tabular}{l|ll} \toprule % @{} removes space
		& God Exists & No God
		\\ \hline
		Simple universe &  & 
		\\
		Complicated universe & $10^{-6}$  & $10^{-18}$
		\\ \bottomrule
\end{tabular}
\end{center}
We don't bother to write out the first row---after all, that's not the case we care about. There you have it: it seems like the observation that our universe is complicated gives overwhelming evidence that God exists. 

Of course, the problem is that we have \emph{assumed} that God exists and God doesn't exist are the only possible outcomes. This is analogous to assuming that you have a theory for an exotic new particle and the results of an experiment either rejects existence of the exotic new particle in favor of the null hypothesis (no new particles) or confirms the existence of the exotic new particle (alternate hypothesis). You know this is too na\"ive. Sometimes there are other reasons why the data of your experiment looks funny---maybe you didn't plug in your cables properly\footnote{\url{https://en.wikipedia.org/wiki/Faster-than-light_neutrino_anomaly}}. So in our ontological argument, Ellenberg says that we should be sure to consider all possibilities. For example, often times complicated things come about by the dreaded `design by committee.' So maybe it's not that God exists, but that there's a whole pantheon of gods who designed the universe through a complex process of peer review and feedback. The result is \emph{this} ugly mess. Our table now looks like
\begin{center}
\begin{tabular}{l|lll} \toprule % @{} removes space
		& God Exists & No God & Many Gods
		\\ \hline
		Simple universe &  & &
		\\
		Complicated universe & $10^{-6}$  & $10^{-18}$ & $10^{-5}$
		\\ \bottomrule
\end{tabular}
\end{center}
The number we threw in here is fairly arbitrary, but the point is that you could make the argument that \emph{many gods} may be more likely than \emph{one god}. Fine, but for a physics class this is starting to look a little bit too much like religious studies, no? Well, as we ponder the origin of our universe, we go to knock on the door of our cosmology colleagues, only to find that they're busy running their cosmological simulation. Aha! You remember playing \emph{The Sims}, which is an odd knock-off experience compared to actual real life. However, it seems plausible that a sufficiently advanced civilization would have created a computer game to simulate their existence. Perhaps when that simulated reality is sufficiently advanced, it too will create a computer program that simulates \emph{their} reality. And so forth. You start to think that this, too, is a rather plausible origin for a strangely complicated universe. Ellenberg updates our table as follows
\begin{center}
\begin{tabular}{l|llll} \toprule % @{} removes space
		& God Exists & No God & Many Gods & Simulated reality
		\\ \hline
		Simple universe &  & & &
		\\
		Complicated universe & $10^{-6}$  & $10^{-18}$ & $10^{-5}$ & $10^{-1}$
		\\ \bottomrule
\end{tabular}
\end{center}
At the level of this crude example, it seems that the most likely possibility is that not only does any god \emph{not} exist, but it's unlikely that we exist. 

\subsection{The End of the World}

There's a cute example of Bayesian statistics that I read about in William Poundstone's \emph{The Doomsday Calculation.} I had the pleasure of meeting William Poundstone once and he seems like a nice and intelligent person. Unfortunately, I did not much enjoy the book. You can read the main argument in Poundstone's article for \emph{Vox}\footnote{\url{https://www.vox.com/the-highlight/2019/6/28/18760585/doomsday-argument-calculation-prediction-j-richard-gott}}.

The example comes from Richard Gott, a renowned astrophysicist. Gott visited the Berlin wall in the summer of 1969. The Berlin wall was built in 1961, so by then the wall was $t=8$ years old. Gott claims to have wondered how long the Berlin wall would stand and came up with the following argument.

Gott figured that there was nothing special about him visiting the Berlin wall in 1969. Assuming that the Berlin wall would be torn down, it had a finite lifetime $T$. He just happened to sample the existence of the Berlin wall at some moment of time within that lifetime. He assumed was no reason for him to be visiting the wall particularly early or particularly late in its lifetime; thus, there's a uniform `prior' probability that he'd visit the wall at any time during its lifetime. With this assumption, he said that there's a 50\% chance that his visit in 1969 happens to fall within the middle 50\% interval of $T$. In other words, there's a 50\% chance that $t=8$ is somewhere between $.25\times T$ and $.75\times T$. 
\begin{itemize}
\item If $t=8$ corresponds to $0.25 \times T$---that is, he happened to show up a little \emph{early} in the wall's lifetime---then $T=32$ and the wall would stand for another 24 years. 

\item Alternatively, if $t=8$ to $0.75 \times T$---that is, he happened to show up a little \emph{late} in the wall's lifetime---then $T=10.6$ and the wall would stand for only another 2.6 years. 
\end{itemize}
This brackets an anticipated lifetime for the Berlin wall to be between $T=10.6$ and $T=32$ years, or a range of dates 1972--1993. Spoiler alert: the Berlin wall was torn down in 1989, which is indeed in this range of years. 

Gott went on to speculate about the implications on how long humanity would survive\footnote{\url{https://www.nature.com/articles/363315a0.epdf}}.
\begin{exercise}
Using Gott's technique for estimating the lifespan of the Berlin wall and the approximation that humanity has existed for around 200,000 years, what is a plausible range of lifetimes for human existence?
\end{exercise}
One of my favorite essays that incorporates Gott's estimation is ``The Riemann Zeta Conjecture and the Laughter of the Primes,'' reprinted in \emph{When Einstein Walked with G\"odel} by Jim Holt. Holt starts the essay with the following sentences:
\begin{quote}
	What will civilization be like in a million years from now? Most of the things we're familiar with today will have disappeared. But some things will survive. And we can be pretty confident that among them will be numbers and laughter.
\end{quote}
\begin{exercise}
Holt's essay uses the following inputs: humans and chimpanzees split off evolutionarily about 5 million years ago. Since chimpanzees are known to laugh and are known to be able to do elementary arithmetic, we assume that laughter and mathematics is at least 5 million years old. Fill in the rest of the argument that `numbers and laughter' will still be around in the year one million.
\end{exercise}

\section{The Monty Hall Problem}

In 1975 Steve Selvin popularized the Monty Hall problem as a puzzle in probability. Consider the following game popularized in an old television show. There are three doors in front of a contestant. Behind one door is a prize, say a brand new car. 
\begin{center}
\includegraphics[width=.7\textwidth]{figures/Monty_01.pdf}
\end{center}
The contestant picks one of the three doors. Rather than opening that door immediately, the host offers a twist. The host---who has perfect knowledge of which door has the car---selects a second door and opens it, revealing no prize. 
\begin{center}
\includegraphics[width=.7\textwidth]{figures/Monty_02.pdf}
\end{center}
The host then offers the contestant the opportunity to either open the door they have currently selected, or to change their mind and instead select the remaining unopened door. There are now two perspectives, leading to what may appear to be a paradox:
\begin{enumerate}
	\item The contents of doors are equally unknown to the contestant, so it does not matter whether or not the contestant swaps choices.
	\item At the beginning of the game, the choice of a door has a 1/3 chance of winning the car. After the host has revealed one empty door, the choice if 1/2. If the contestant does not swap doors, then they are effectively locking in their 1-in-3 chance from the beginning of the game. Thus it makes sense for the contestant to swap choices since this locks in a 1-in-2 chance.
\end{enumerate}
Bayes theorem gives us the tools to quantify this puzzle. For concreteness, assume that the contestant initially chooses door 1, the host reveals door 2 to be empty, and then the host offers door 3. We would like to compare the following quantities
\begin{itemize}
	\item $P(1\,|\,\text{open}~2)$, the probability that the car is behind door 1 after the host has revealed that it is not in door 2. That is, the probability that the contestant wins by \emph{not} swapping doors.
	\item $P(3\,|\,\text{open}~2)$, the probability that the car is behind door 3 after the host has revealed that it is not in door 2. That is, the probability that the contestant wins by \emph{swapping} doors.
\end{itemize}
We are \emph{not} comparing either of these to $P(1)$, the probability that the car is behind door 1 \emph{prior} to any information about the other doors. In order to calculate the above conditional probabilities, we invoke Bayes' theorem.
\begin{align}
	P(1\,|\,\text{open}~2) = \frac{P(\text{open}~2\,|\,1)\, P(1)}{P(\text{open}~2)} \ .
\end{align}
The prior probability is $P(1)=1/3$. The first factor in the numerator is the probability that the host opens door 2 \emph{assuming} that the car is actually behind door 1. If the contestant were correct with their initial guess, then the host could have opened door 2 or 3. Thus $P(\text{open}~2\,|\,1)=1/2$. The denominator is the probability that the host would have revealed door 2 without any constraint from the contestant. Because there are two choices of empty doors, this is  $P(\text{open}~2)=1/2$. As a result, the \emph{posterior} probability is
\begin{align}
	P(1\,|\,\text{open}~2) = \frac{1}{3} \ .
\end{align}
We interpret this as follows: if the contestant does not to switch doors and remains with door 1, the odds of them winning is 1-in-3. 

What if the contestant switches? Bayes tells us
\begin{align}
	P(3\,|\,\text{open}~2) = \frac{P(\text{open}~2\,|\,3)\, P(3)}{P(\text{open}~2)} \ .
\end{align}
The denominator is unchanged. Similarly, the prior probability $P(3)=P(1)=1/3$ is unchanged. However, $P(\text{open}~2\,|\,3)$ is rather different. This is the probability that the host opens door 2 given that the actual car is behind door 3. However, what is not well conveyed in our notation---but nonetheless is true when applying Bayes' theorem---is that the host's choice is constrained by the fact that they \emph{cannot} reveal either (1) the door that the contestant has selected, or (2) the door that actually has the car. In this conditional probability, the host has no choice: they \emph{must} select door 2 since the contestant has selected door 1 and the car is behind door 3. This means that $P(\text{open}~2\,|\,3) = 1$. With that in mind, we have
\begin{align}
	P(3\,|\,\text{open}~2) = \frac{2}{3} \ .
\end{align}
Of course, we already knew this had to be true since conservation of probability tells us that $P(1\,|\,\text{open}~2)+P(3\,|\,\text{open}~2)=1$.

\section{Information Entropy}

In statistical mechanics we have a notion of \textbf{entropy},
\begin{align}
	S = -\sum_i p_i \ln p_i \ ,
\end{align}
where $p_i$ is the probability of a configuration. We have assumed $k_B=1$, a kind of `natural units' for statistical mechanics. 

\begin{example}
\textbf{Microcanonical ensemble}. For the microcanonical ensemble, all $N$ possible states have equal probability so that $p_i = 1/N$. Then the entropy is $S = -N^{-1}\sum_i\ln N^{-1} = \ln N$, which simply measure the logarithm of the number of microstates. 
\end{example}

There is a closely related idea in information theory called \textbf{information entropy} or Shannon entropy,
\begin{align}
	H = -\sim_i p_i \log_2 p_i \ ,
\end{align}
where the binary base tells you that we are counting bits of information. The Shannon entropy measures the information in a configuration. A bit is corresponds to reducing the uncertainty by a factor of two. Over the last ten years, information entropy has found its way into physics, so it is worth taking a moment to familiarize ourselves with the idea. 

There is an excellent pedagogical example by Aur\'elien G\'eron. Suppose we live somewhere where the weather has a 50\% chance each of being sunny or rainy. If the weather report says it will definitely rain tomorrow, then it has reduced the uncertainty relative to the baseline scenario by one bit: you went from two equal possibilities to one possibility. 

Similarly, if there were four equal probabilities (sun, rain, fog, snow), then a definite forecast reduces the uncertainty from 4 equally possible states to one. In other words, the forecast provides $2$ bits of information to reduce the number of possibilities by $1/2^2$. If there were eight equally possible weather possibilities, then a definite forecast provides $4$ bits of information that reduces the number of possibilities by $1/2^4$. 

The uncertainty reduction from a specification of a state is the inverse of the probability of that state. We find a convenient definition: when there are $N$ equal probabilities for all possible states, the \textbf{number of bits} encoded in a message that specifies a state is simply $\log_2 N$.

Of course, the weather analogy so far is pretty silly for us since we live in Riverside. In a climate slightly closer to Riverside, the odds of rain are only 25\% and the odds of a sunny day are 75\%. You expect it to be sunny tomorrow. This means that if the national weather service tells us that it will actually rain tomorrow, it has conveyed \emph{more information} than the case of equal probabilities. In other words, the information in a message can be interpreted as the degree of \emph{surprise} relative to the expected probability distribution. The uncertainty reduction coming from a rainy forecast is
\begin{align}
	\frac{1}{P(\text{rain})} = \frac{1}{25\%} = 4 \ .
\end{align}
The rainy forecast reduces our uncertainty by $4=2^2$, or $\log_2 4=-\log_2 P(\text{rain})=2$ bits. Alternatively, if the forecast is sunny, then we are not particularly impressed. That is what we expected. The number of bits is $-\log_2P(\text{sun}) = -\log_2(75\%) \approx 0.4$.

This gives a useful general definition: \textbf{information entropy} is the expected number of bits given a statement of the state:
\begin{align}
	H(p) = \langle -\log_2 p_i \rangle = \sum_i (-\log_2 p_i)\, p_i \ .
\end{align}
This explains how autocomplete algorithms work. Some letters are more frequent in English words than others. Saying that a word begins with `x' reduces the number of possible words much more than saying that the word begins with a much more common letter, like `e'.

\flip{To do: cross entropy, KL divergence.}




%
% \textsc{p.t.} thanks the Aspen Center for Physics (NSF grant \#1066293) for its hospitality during a period where part of this work was completed. \textsc{p.t.} is supported by the DOE grant \textsc{de-sc}/0008541.

%% Appendices
% \appendix


%% Bibliography
%\bibliographystyle{utcaps} 	% arXiv hyperlinks, preserves caps in title
%\bibliographystyle{utphys} 	% arXiv hyperlinks
% \bibliography{bib title without .bib}


\end{document}