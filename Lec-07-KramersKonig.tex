%!TEX root = P231_notes.tex

\section{Kramers--Kronig [optional]}
% \lecdate{lec~12}
% 2017 Lec 17ish

\subsection{Cauchy Principal Value}
% See Cohen, "Complex Analysis with Applications to Science and Engineering"
% p. 139

We have seen that the pole structure of an integrand has physical significance. When solving for the harmonic oscillator Green's function, we saw that we had to \emph{change} the nature of the harmonic oscillator differential operator in order to derive the \emph{retarded} (causal) Green's function rather than the \emph{advanced} (acausal) Green's function. In that case, we found a real integral with a pole along the integration axis and decided that the integrand itself was the problem.

What happens if \emph{we don't want to change the integrand?} For example, most physicists wouldn't object too much that the integral
\begin{align}
	\int_{-1}^4 dx\, \frac{1}{x-1}
\end{align}
should have a well defined value even though there is a simple pole at $x=1$. In fact, with a bit of thought (try sketching the integrand on a graph), one may argue that
\begin{align}
	\int_{-1}^4 dx\, \frac{1}{x-1}
	= 
	\int_{3}^4 dx\, \frac{1}{x-1}
\end{align}
because the integrand is antisymmetric about the pole. Sure, the integrand is undefined at $x=1$, but every contribution just to the right of the pole, say $x=1+\varepsilon$ is exactly canceled by a contribution just to the left of the pole, $x=1-\varepsilon$. This is purely \emph{real} calculus.

The \textbf{Cauchy principal value} of a real integral with a pole along the integration contour (the real axis) formalizes this idea. Suppose $f(x)$ is well behaved (analytic) along the real axis. For an integrand $f(x)/(x-x_0)$, the Cauchy principal value is defined to be
\begin{align}
	\mathcal P \int_{-\infty}^\infty dx\, \frac{f(x)}{x-x_0}
	= 
	\lim_{\varepsilon\to 0}
	\left[
	\int_{-\infty}^{x_0-\varepsilon} dx\, \frac{f(x)}{x-x_0}
	+\int_{x_0+\varepsilon}^\infty dx\, \frac{f(x)}{x-x_0}
	\right] \ .
\end{align}
In other words, you simply integrate everywhere except the pole along the real axis. Formally we cannot say anything about the integrand at $x=x_0$, but perhaps part of you secretly thinks this is not a big deal since the $\varepsilon\to 0$ limit is well behaved. Again, thus far we are only doing real calculus.

\flip{Include picture}

\paragraph{Relation to a contour.} We can now connect to complex analysis. Suppose $f(x)$ is well behaved enough that we can imagine integrating $f(z)/(z-x_0)$ along a closed contour with the integral along the ``arc at infinity'' (either in the upper or lower half plane) going to zero. Then we can write the following:
\begin{align}
	\int_{C(\gamma_\pm)} dz \; \frac{f(z)}{(z-x_0)} &= 
	\int_{-\infty}^{x_0-\varepsilon} dx\, \frac{f(x)}{x-x_0}
	+
	\int_{x_0+\varepsilon}^\infty dx\, \frac{f(x)}{x-x_0}
	+
	\int_{\gamma_\pm} dz \; \frac{f(z)}{(z-x_0)}
	+
	\int_{\text{arc}} dz \; \frac{f(z)}{(z-x_0)} \ .
	\label{eq:cauchy:contour}
\end{align}
By assumption, the last term goes to zero. The first two terms are the Cauchy principal value when $\varepsilon\to 0$. The third term along contour $\gamma$ is a small semicircle of radius $\varepsilon$ that connects the two contours of the Cauchy principal value. It's a little semicircle that either goes just above ($\gamma_+$) or just below ($\gamma_-$) the real axis. Of course, you already realize the important point: the contour $C$ \emph{depends} on the choice of $\gamma_\pm$. This determines whether or not the point $z=x_0$ is inside or outside the contour.

\paragraph{Residue Theorem.} Let us assume that the integrand $f(z)/(z-x_0)$ is such that one can close the integral along the upper-half plane.\footnote{The argument is completely analogous if it were to be closed along the lower-half plane} This means that $f(z)/(z-x_0) \to 0$ fast enough as the radius of the semicircle goes to $R\to \infty$.\footnote{Remind yourself why `fast enough' is $1/R^2$ or faster.} Now the residue theorem tells us that
\begin{align}
	\int_{C(\gamma_\pm)} dz \; \frac{f(z)}{(z-x_0)} &= 2\pi i \sum_j \text{Res}_F(z_j)
	&
	F(z) = \frac{f(z)}{(z-x_0)} \ ,
\end{align}
where $j$ runs over the poles enclosed in $C(\gamma_\pm)$. When we take $\gamma_+$, this means that the poles do not include $x_0$. When we take $\gamma_-$, the poles do include $x_0$. The residue at $x_0$ is
\begin{align}
 	\text{Res}_F(x_0) &= f(x_0) \ .
\end{align}
In general, we expect $f(z)$ to have its own poles off the real axis; if any of those poles simple and are enclosed then they contribute to the sum. Just remember that if $z_j$ is a simple pole of $f(z)$, then the relevant residue is $(z_j-x_0)^{-1}\text{Res}_f(z_j)$.\footnote{Take the time to remind yourself why this is true if it is not obvious.}

\paragraph{Little semi-circles.}
The next step is to identify what the $\gamma_\pm$ integrals give. We may use the parameterization $z=x_0+\varepsilon e^{i\theta}$ to write
\begin{align}
	\int_{\gamma_+} dz \; \frac{f(z)}{(z-x_0)}
	&=
	\int_{\pi}^0 i\varepsilon e^{i\theta} d\theta \frac{f\left(x_0+\varepsilon e^{i\theta}\right)}{\varepsilon e^{i\theta}}
	=
	-i \pi f(x_0) \ .
\end{align}
We went ahead and took the $\varepsilon\to 0$ limit, implicitly using the analyticity of $f(z)$ at $z=x_0$. Similarly, if we used $\gamma_-$ as part of our contour,
\begin{align}
	\int_{\gamma_-} dz \; \frac{f(z)}{(z-x_0)}
	&=
	\int_{\pi}^{2\pi} i\varepsilon e^{i\theta} d\theta \frac{f\left(x_0+\varepsilon e^{i\theta}\right)}{\varepsilon e^{i\theta}}
	=
	+i \pi f(x_0) \ .
\end{align}

\paragraph{Putting it all together.}
We may then write \eqref{eq:cauchy:contour} as
\begin{align}
	\int_{C(\gamma_\pm)} dz \; \frac{f(z)}{(z-x_0)} &= 
	\mathcal P
	\int_{-\infty}^\infty dx\, \frac{f(x)}{x-x_0}
	+
	\int_{\gamma_\pm} dz \; \frac{f(z)}{(z-x_0)}
\end{align}
so that
\begin{align}
	\mathcal P
	\int_{-\infty}^\infty dx\, \frac{f(x)}{x-x_0}
	&=
	2\pi i \sum_j \left.\text{Res}_F(z_j)\right|_{C_\pm}
	\pm i \pi f(x_0) \ ,
	\label{eq:cauchy:principal:wrt:residues}
\end{align}
where we remind ourselves that the choice of contour ($\pm$) determines both the sign of the $\mp i \pi f(x_0)$ term \emph{and} whether or not the pole $z=x_0$ is enclosed by $C_\pm$. 
\begin{exercise}
Confirm that the right-hand side of \eqref{eq:cauchy:principal:wrt:residues} is independent of whether one closes the contour with $\gamma_+$ or $\gamma_-$.
\end{exercise}

\paragraph{Cauchy principal value and the $i\varepsilon$ notation.} %see Cohen.

Let us continue to assume that the contour $C$ is closed by an arc in the upper-half plane.\footnote{Recall that this amounts to an assumption about the convergence of $F(z)$ as $z\to Re^{i\theta}$ for large $R$ and $0 < \theta < \pi$.}
It is clear that closing the contour $C(\gamma_+)$ with $\gamma_+$ corresponds to excluding the pole at $z=x_0$ from the sum of residues. This means it is equivalent to a contour with the uninterrupted entire real axis $C_0$ for a modified integrand where the pole has been pushed downward:
\begin{align}
	\int_{C(\gamma_+)} dz \, \frac{f(z)}{z-x_0}
	=
	\lim_{\varepsilon\to 0}
	\int_{C_0} dz \, \frac{f(z)}{z-(x_0-i\varepsilon)} \ .
\end{align}
Similarly, for the contour $C(\gamma_-)$ with $\gamma_-$, the integral is equivalent to the contour that includes the uninterrupted real axis $C_0$ for a modified integrand where the pole is pushed upward into the contour:
\begin{align}
	\int_{C(\gamma_-)} dz \, \frac{f(z)}{z-x_0}
	=
	\lim_{\varepsilon\to 0}
	\int_{C_0} dz \, \frac{f(z)}{z-(x_0+i\varepsilon)} \ .
\end{align}
This gives a way to write \eqref{eq:cauchy:principal:wrt:residues} independently of the $\gamma_\pm$:
\begin{align}
	\mathcal P
	\int_{-\infty}^\infty dx\, \frac{f(x)}{x-x_0}
	&=
	% 2\pi i \sum_j \left.\text{Res}_F(z_j)\right|_{C_\pm}
	\lim_{\varepsilon\to 0}
	\int_{C_0} dz \, \frac{f(z)}{z-(x_0\mp i\varepsilon)} \ 
	% \mp i \pi \int dx\; f(x) \delta(x-x_0) \ .
	\pm i \pi f(x_0) \ .
\end{align}
Recalling that the contour integral over $C_0$ is really just the integral along the real line plus a vanishing integral over the arc, we may write the right-hand side as a real integral:
\begin{align}
	\mathcal P
	\int_{-\infty}^\infty dx\, \frac{f(x)}{x-x_0}
	&=
	% 2\pi i \sum_j \left.\text{Res}_F(z_j)\right|_{C_\pm}
	\lim_{\varepsilon\to 0}
	\int_{-\infty}^\infty dx\; f(x)\left[
		\frac{1}{x-(x_0\mp i\varepsilon)} \ 
		\pm i \pi   \delta(x-x_0) 
	\right] 
	 \ .
\end{align}
We see that a procedure for making sense of a real integral over a singularity gives us an expression that contains an imaginary part. This is often written at the level of \emph{distributions} as
\begin{align}
	\left.\frac{1}{x-x_0}\right|_P
	&= 
	\lim_{\varepsilon\to 0}
		\frac{1}{x-(x_0\mp i\varepsilon)} \ 
		\pm i \pi   \delta(x-x_0) \ .
		\label{eq:cauchy:principal:as:distribution}
\end{align}
By distribution we mean that the object \emph{only} makes sense in an integral, most likely being multiplied against another function. You already know that this expression only makes sense as a distribution because there's a `naked' $\delta$ function, and those things do not make sense outside of an integral. The $\left.\right|_P$ means principal value, that is: remember to put the $\mathcal P$ when you perform the integral. 

\subsection{Cauchy Principal Value + Cauchy Integral Representation}

Remember the Cauchy integral formula, \eqref{eq:cauchy:integral}? This told us that if a function $f$ is analytic around some point $z$, then we can express $f(z)$ as a contour integral around $z$:
\begin{align}
	f(z) = \frac{1}{2\pi i}\oint dz' \frac{f(z')}{(z'-z)} \ .
\end{align}
We first introduced this formula as a stepping stone to deriving the residue theorem. Now we have returned to the integral formula due to its striking resemblance to the integrals that popped up when describing principal values.

Suppose we want to take the limit where $z$ is the complexification of a real (physical) quantity. That is to say, we want to take the limit $z\to x$. For concreteness, let's assume that the physical limit is
\begin{align}
	z = \lim_{\varepsilon\to 0} x + i\varepsilon \ ,
\end{align}
so that $z$ approaches the real axis from above. This may be the case due to causality with our choice of sign conventions, as we saw for the harmonic oscillator Green's function.\footnote{You may be confused why we're approaching the real axis from above. We argued in the harmonic oscillator case that a causal theory has the poles pushed below the real axis---at least with our sign conventions for the Fourier transform. Because the physical pole approaches the real axis from below, we know that the function $f(x)$ is analytic in the region above (and in principle including) the real axis.} The relevant sign for the Cauchy principal value expression \eqref{eq:cauchy:principal:as:distribution} is
\begin{align}
	\lim_{\varepsilon\to 0} \frac{1}{x'-(x+i\varepsilon)}
	=
	\left.
	\frac{1}{x'-x}\right|_P
	+ i\pi \delta(x'-x) \ , 
\end{align}
where we've been careful with the choice of variable names.
\begin{exercise}
Write this expression for the case where the physically relevant limit approaches the real axis from above.
\end{exercise}
Let us now plug this into the Cauchy integral representation for $z = x-i\varepsilon$. Assume that $f(z)$ has all the usual convergence requirements\footnote{Pop quiz: what are these requirements? Ultimately there's a real integral that we have analytically continued into the complex plane. We want the integrand along the large arc around the lower-half plane to vanish.}
\begin{align}
	f(z) &= \frac{1}{2\pi i}\oint dz' \frac{f(z')}{z' - (x+i\varepsilon)}
	\\
	&=
	\frac{1}{2\pi i} 
	\left[	
		\mathcal P \int_{-\infty}^\infty dx \, 
		\frac{f(x')}{x'-x}
		-
		i\pi \int_{-\infty}^\infty dx \, f(x')\delta(x'-x)
	\right]
	\\
	&=
	\frac{1}{2\pi i} 
		\mathcal P \int_{-\infty}^\infty dx \, 
		\frac{f(x')}{x'-x}
		-\frac{1}{2}f(x) \ .
\end{align}
Simplifying this expression gives
\begin{align}
	f(x) &= \frac{1}{\pi i} 
		\mathcal P \int_{-\infty}^\infty dx 
		\, \frac{f(x')}{x'-x} \ .
\end{align}
This is now rather surprising! At this point, we have written everything in terms of the function $f(x)$ evaluated for \emph{real} arguments. Nothing in the integral along the real axis introduces any additional `imaginary-ness' to the right-hand side. However, there is this pernicious factor of $i$ on the right-hand side that seems to mix up the real and imaginary parts of $f$. To see this explicitly, write $f(x) = u(x) + i v(x)$. Then we have
\begin{align}
	u(x) &= \frac{1}{\pi}
	\mathcal P \int_{-\infty}^\infty dx 
		\, \frac{v(x')}{x'-x} 
		&
	v(x) &= -\frac{1}{\pi}
	\mathcal P \int_{-\infty}^\infty dx 
		\, \frac{u(x')}{x'-x} 
	\ .
\end{align}
To spell it out explicitly, we have the \textbf{Kramers--Kronig dispersion relations} (a name we justify below):
\begin{align}
	\text{Re}~f(x) &= \frac{1}{\pi}
	\mathcal P \int_{-\infty}^\infty dx 
		\, \frac{\text{Im}~f(x')}{x'-x} 
		&
	\text{Im}~f(x) &= -\frac{1}{\pi}
	\mathcal P \int_{-\infty}^\infty dx 
		\, \frac{\text{Re}~f(x')}{x'-x} 
	\ .
\end{align}
The remarkable observation is that the real and imaginary parts of the function $f$ are determined by one another through a principal value integral. The only assumption we made about $f$ is that it is is analytic just above the real line because the physical poles approach the real line from the lower half plane due to causality.

\begin{example}
The step-function ($\Theta$) is 1 for arguments larger than 0 and is otherwise zero. It is a convenient way to encode the causal properties of the retarded Green's function. \flip{FT of step...}
\end{example}

\subsection{Digression: Convolution Theorem}


j
Suppose $f(x) = g(x)h(x)$. Then the Fourier transform is
% https://doi.org/10.1119/1.15901
\subsection{Example: Dielectric Media}




\flip{In Progress.}



